\documentclass[]{article}
\usepackage{graphicx}
\usepackage[svgnames]{xcolor} 
\usepackage{fancyhdr}
\usepackage{fancyvrb}
\usepackage{tocloft}
\usepackage[hidelinks]{hyperref}
\usepackage{enumitem}
\usepackage[many]{tcolorbox}
\usepackage{listings }
\usepackage[a4paper, total={6in, 8in} , top = 2cm,bottom = 4cm]{geometry}
%\usepackage[a4paper, total={6in, 8in}]{geometry}
\usepackage{afterpage}
\usepackage{amssymb}
\usepackage{pdflscape}
\usepackage{textcomp}
\usepackage{xecolor}
\usepackage{rotating}
\usepackage[Kashida]{xepersian}
\usepackage[T1]{fontenc}
\usepackage{tikz}
\usepackage[utf8]{inputenc}
\usepackage{PTSerif} 
\usepackage{seqsplit}
\usepackage{changepage}


\usepackage{listings}
\usepackage{xcolor}
\usepackage{sectsty}

\setcounter{secnumdepth}{0}
 
\definecolor{codegreen}{rgb}{0,0.6,0}
\definecolor{codegray}{rgb}{0.5,0.5,0.5}
\definecolor{codepurple}{rgb}{0.58,0,0.82}
\definecolor{backcolour}{rgb}{0.95,0.95,0.92}
\definecolor{blanchedalmond}{rgb}{1.0, 0.92, 0.8}
\definecolor{brilliantlavender}{rgb}{0.96, 0.73, 1.0}
 
\NewDocumentCommand{\codeword}{v}{
\texttt{\textcolor{blue}{#1}}
}
\lstset{language=java,keywordstyle={\bfseries \color{blue}}}

\lstdefinestyle{mystyle}{
    backgroundcolor=\color{backcolour},   
    commentstyle=\color{codegreen},
    keywordstyle=\color{magenta},
    numberstyle=\tiny\color{codegray},
    stringstyle=\color{codepurple},
    basicstyle=\ttfamily\normalsize,
    breakatwhitespace=false,         
    breaklines=true,                 
    captionpos=b,                    
    keepspaces=true,                 
    numbers=left,                    
    numbersep=5pt,                  
    showspaces=false,                
    showstringspaces=false,
    showtabs=false,                  
    tabsize=2
}

\lstset{style=mystyle}

 \settextfont[BoldFont={XB Zar bold.ttf}]{XB Zar.ttf}


\setlatintextfont[Scale=1.0,
 BoldFont={LiberationSerif-Bold.ttf}, 
 ItalicFont={LiberationSerif-Italic.ttf}]{LiberationSerif-Regular.ttf}





\newcommand{\inputsample}[1]{
    ~\\
    \textbf{ورودی نمونه}
    ~\\
    \begin{tcolorbox}[breakable,boxrule=0pt]
        \begin{latin}
            \large{
                #1
            }
        \end{latin}
    \end{tcolorbox}
}

\newcommand{\outputsample}[1]{
    ~\\
    \textbf{خروجی نمونه}

    \begin{tcolorbox}[breakable,boxrule=0pt]
        \begin{latin}
            \large{
                #1
            }
        \end{latin}
    \end{tcolorbox}
}

\newtcolorbox{mybox}[2][]{colback=red!5!white,
colframe=red!75!black,fonttitle=\bfseries,
colbacktitle=red!85!black,enhanced,
attach boxed title to top center={yshift=-2mm},
title=#2,#1}

\newenvironment{changemargin}[2]{%
\begin{list}{}{%
\setlength{\topsep}{0pt}%
\setlength{\leftmargin}{#1}%
\setlength{\rightmargin}{#2}%
\setlength{\listparindent}{\parindent}%
\setlength{\itemindent}{\parindent}%
\setlength{\parsep}{\parskip}%
}%
\item[]}{\end{list}}


\definecolor{foldercolor}{RGB}{124,166,198}
\definecolor{sectionColor}{HTML}{ff5e0e}
\definecolor{subsectionColor}{HTML}{008575}

\definecolor{listColor}{HTML}{00d3b9}

\definecolor{umlrelcolor}{HTML}{3c78d8}

\definecolor{subsubsectionColor}{HTML}{3c78d8}

\defpersianfont\authorFont[Scale=0.9]{XB Zar bold.ttf}

\defpersianfont\titr[Scale=1.5]{Lalezar-Regular.ttf}

\defpersianfont\fehrest[Scale=1.2]{Lalezar-Regular.ttf}

\defpersianfont\fehrestTitle[Scale=3.0]{Lalezar-Regular.ttf}

\defpersianfont\fehrestContent[Scale=1.2]{XB Zar bold.ttf}


\sectionfont{\color{sectionColor}}  % sets colour of sections
\subsectionfont{\color{subsectionColor}}  % sets colour of sections
\subsubsectionfont{\color{subsubsectionColor}}


\renewcommand{\labelitemii}{$\circ$}


\renewcommand{\baselinestretch}{1.1}


\renewcommand{\contentsname}{فهرست}

\renewcommand{\cfttoctitlefont}{\fehrestTitle}


\renewcommand\cftsecfont{\color{sectionColor}\fehrestContent\selectfont}
\renewcommand\cftsubsecfont{\color{subsectionColor}\fehrestContent\selectfont}
\renewcommand\cftsubsubsecfont{\color{subsubsectionColor}\fehrestContent\selectfont}
%\renewcommand{\cftsecpagefont}{\color{sectionColor}}

\setlength{\parskip}{1.2pt}

\begin{document}


%%% title pages
\begin{titlepage}
\begin{center}

\textbf{ \Huge{به نام خدا} }
        
\vspace{0.2cm}

\includegraphics[width=0.4\textwidth]{sharif1.png}\\
\vspace{0.2cm}
\textbf{ \Huge{\emph درس برنامه‌سازی پیشرفته} }\\
\vspace{0.25cm}
\textbf{ \Large{ فاز اول پروژه - ضمیمه توضیحات کارت} }
\vspace{0.2cm}
       
 
      \large \textbf{دانشکده مهندسی کامپیوتر}\\\vspace{0.1cm}
    \large   دانشگاه صنعتی شریف\\\vspace{0.2cm}
       \large   ﻧﯿﻢ سال دوم 00-99 \\\vspace{0.10cm}
      \noindent\rule[1ex]{\linewidth}{1pt}
استاد:\\
    \textbf{{دکتر محمدامین فضلی}}



    \vspace{0.20cm}

   مهلت ارسال:\\
    \textbf{{۲۳ اردیبهشت - }}
    \textbf{{ساعت 23:59:59}}

    \vspace{0.10cm}
مسئول پروژه:\\
    \textbf{\authorFont{امیرمهدی نامجو}}
    
        \vspace{0.10cm}
مسئول فاز اول:\\
    \textbf{\authorFont{عرشیا اخوان}}
    
        \vspace{0.10cm}
طراحان فاز اول:\\
    \textbf{\authorFont{سروش جهان‌زاد، متین شجاع، امیرصدرا عبدالهی، امیرمهدی کوششی، یاسمین گلزار، امیرحسین هادیان }}
    
        \vspace{0.05cm}
مسئولین تنظیم مستند:\\
    \textbf{\authorFont{سروش جهان‌زاد و پارسا محمدیان}}
    

\end{center}
\end{titlepage}
%%% title pages


%%% header of pages
\newpage
\pagestyle{fancy}
\fancyhf{}
\fancyfoot{}
\cfoot{\thepage}
\lhead{فاز اول}
\rhead{\includegraphics[width=0.1\textwidth]{sharif.png}\\
دانشکده مهندسی کامپیوتر
}
\chead{پروژه برنامه‌سازی پیشرفته}
%%% header of pages
\renewcommand{\headrulewidth}{2pt}

\KashidaOff



\tableofcontents

\newpage

 \Large \textbf{\\\\
}


\section*{{\titr نکات قابل توجه}}
\addcontentsline{toc}{section}{{\fehrestContent نکات قابل توجه}}
\begin{itemize}
\item
پس از اتمام این فاز، در گیت خود یک تگ با ورژن \lr{"v1.0.0"} بزنید. در روز تحویل حضوری این tag بررسی خواهد شد و کدهای پس از آن نمره‌ای نخواهد گرفت. برای اطلاعات بیش‌تر در مورد شیوه ورژن‌گذاری، می‌توانید به
 \href{https://semver.org/}{\textcolor{blue}{\underline{این لینک}}}
 مراجعه کنید. البته برای این پروژه صرفا رعایت کردن همان ورژن گفته شده کافیست، اما خوب‌ است که با منطق ورژن‌بندی هم آشنا بشوید.

\item
در روز تحویل حضوری مشارکت تمام اعضای تیم در پروژه بررسی خواهد‌ شد و در صورت عدم مشارکت بعضی از اعضا، نمره‌ی ایشان برای آن فاز پروژه "صفر" لحاظ می‌گردد. مشارکت، با توجه به \lr{commit} های افراد تیم در مخزن گیت‌هاب پروژه بررسی می‌شود.

\item
در هر فاز می‌توانید سه روز تاخیر به ازای کسر نمره داشته‌ باشید که به ازای هر روز آن، ۱۰ درصد از نمرهٔ آن فاز را از دست خواهید‌ داد. در مجموع سه‌فاز پروژه، سه روز تاخیر نیز بخشیده خواهد‌ شد.

\item
به ازای هر ساعتی که پروژه را زودتر تحویل دهید، ۱۵ دقیقه به مهلت تاخیر بدون کسر نمره شما اضافه خواهد‌ شد. این مقدار حداکثر یک روز خواهد‌ بود که در صورت ارسال ۴ روز زودتر از ددلاین به شما تعلق خواهد گرفت. \textbf{بنابراین ددلاین‌های پروژه تحت هیچ شرایطی تمدید نخواهد‌ شد.} توصیه می‌شود با برنامه‌ریزی مناسب به ددلاین‌های درس پایبند باشید.

\item
در صورت کشف تقلب از هریک از تیم‌ها، برای بار اول منفی نمرهٔ آن فاز برای آن تیم ثبت می‌شود و برای بار دوم، نمرهٔ منفی کل پروژه برای تیم لحاظ خواهد‌ شد که معادل مردود شدن در درس است.
\end{itemize}
\newpage



\section*{{\titr معرفی کارت‌ها }}
\addcontentsline{toc}{section}{{\fehrestContent معرفی کارت‌ها}}
همان‌طور که می‌دانید هیولاها، افسون‌ها و تله‌ها جز مهمی از این بازی هستند. در این جا می‌خواهیم تعدادی
از این کارت‌ها و ویژگی‌های آن‌ها را به شما معرفی کنیم.
دقت کنید که خصوصیات و اثرات بعضی کارت‌ها برای راحت‌تر شدن پیاده‌سازی شما، تغییر کرده و ساده‌تر شده‌اند؛ 
بنابراین برای ارزیابی پایانی هم همین توضیحات مبنا قرار خواهند گرفت. \\
برای آشنایی بیشتر با اصطلاحات، روند بازی و شرایط مورد نیاز برای اتفاقات گفته شده، اکیدا توصیه می‌شود که
 مستندات مربوط به بازی را که در فایل ضمیمه‌ی دیگری در اختیارتان قرار داده شده است مطالعه کنید.


\tcbset{skin=bicolor,colback=LightGreen,colframe=DarkGreen,
	colbacklower=LimeGreen!75!LightGreen,
	width=1\textwidth,before=,after=\hfill,
	left=1mm,right=1mm,top=1mm,bottom=1mm,middle=1mm   , varwidth boxed title=center , halign lower=center , halign title=center , halign upper=center}








\newcommand{\spell}[5]{
	\begin{tcolorbox}[adjusted title=\lr{#1}]
		\includegraphics[width = 40mm]{#2}
		
		افسون یا تله: #3
		
		نوع: #4
		
		\tcblower
		
		#5
		
	\end{tcolorbox}
	
}


\newcommand{\monster}[7]{
	\begin{tcolorbox}[adjusted title=\lr{#1}]
		\includegraphics[width = 40mm]{#2}
		
		نوع هیولا: \lr{#3} \quad سطح: {#4}
		
		قدرت حمله: #5 \quad قدرت دفاع: #6
		\tcblower
		
		#7
		
	\end{tcolorbox}
	
}





\subsection*{{\titr کارت‌های هیولا}}
\addcontentsline{toc}{subsection}{{\fehrestContent کارت‌های هیولا}}

\monster{Command knight}{Resources/monsters/CK.jpg}{Warrior/Effect}{4}{1000}{1000}{
	این هیولا یک هیولا با اثر ادامه‌دار (\lr{continuous effect monster}) است که هنگامی که در زمین به رو (\lr{face up}) قرار گرفته باشد به قدرت حمله‌ی همه ی کارت‌هایی که روی زمین قرار گرفته‌اند 400 واحد اضافه می‌کند و تا زمانی که کارت هیولای دیگری در زمین باشد نمی‌توان به این کارت حمله کرد (اگر به رو باشد و اثر (\lr{effect}) آن در حال اجرا باشد).
	
}



\monster{Battle Ox}{Resources/monsters/BO.jpg}{Beast-Warrior}{4}{1700}{1000}{
	بدون توضیحات اضافه
}

\monster{Axe Raider}{Resources/monsters/AxeWarrior.jpg}{Warrior}{4}{1700}{1150}{
	بدون توضیحات اضافه
}



\monster{Horn Imp}{Resources/monsters/HI.jpg}{Fiend}{4}{1300}{1000}{
	بدون توضیحات اضافه
}

\monster {Yomi Ship}{Resources/monsters/YS.jpg}{Aqua/Effect}{3}{800}{1400}{
	وقتی یک هیولا به آن حمله کند و نابودش کند، خود آن هیولا نیز نابود می‌شود.
}


\monster{Silver Fang}{Resources/monsters/SF.jpg}{Beast}{3}{1200}{800}{
	بدون توضیحات اضافه
}

\monster{Suijin}{Resources/monsters/suijin.jpg}{Aqua/Effect}{7}{2500}{2400}{
	 اگر یک هیولا به این کارت حمله کند، می‌توانید امتیاز حمله آن هیولا را فقط برای همان نوبت صفر کنید. (این اثر پیش از محاسبه‌ی آسیب انجام می‌شود). از وقتی که این کارت به رو باشد، فقط برای یک بار می‌توانید این اثر را فعال کنید.
}


\monster{Fireyarou}{Resources/monsters/FireY.jpg}{Pyro}{4}{1300}{1000}{
	بدون توضیحات اضافه
}


\monster{Curtain of Dark Ones}{Resources/monsters/CODO.jpg}{Spellcaster}{2}{600}{500}{
	بدون توضیحات اضافه
}


\monster{Feral Imp}{Resources/monsters/FI.jpg}{Fiend}{4}{1300}{1400}{
	بدون توضیحات اضافه
}

\monster{Dark Magician}{Resources/monsters/DM.jpg}{Spellcaster}{7}{2500}{2100}{
	بدون توضیحات اضافه
}

\monster{Wattkid}{Resources/monsters/WK.jpg}{Thunder}{3}{1000}{500}{
	بدون توضیحات اضافه
}

\monster{Baby Dragon}{Resources/monsters/BabyDragon.jpg}{Dragon}{3}{1200}{700}{
	بدون توضیحات اضافه
}

\monster{Hero of the East}{Resources/monsters/hote.jpg}{Warrior}{3}{1100}{1000}{
	بدون توضیحات اضافه
}

\monster{Battle Warrior}{Resources/monsters/BW.jpg}{Warrior}{3}{700}{1000}{
	بدون توضیحات اضافه
}


\monster{Crawling dragon}{Resources/monsters/CrawlingDragon.jpg}{Dragon}{5}{1600}{1400}{
	بدون توضیحات اضافه
}

\monster{Flame Manipulator}{Resources/monsters/FM.jpg}{Spellcaster}{3}{900}{1000}{
	بدون توضیحات اضافه
}

\monster{Blue-Eyes White Dragon}{Resources/monsters/BEWD.jpg}{Dragon}{8}{3000}{2500}{
	بدون توضیحات اضافه
}

\monster{Crab Turtle}{Resources/monsters/crab.jpg}{Aqua/Ritual}{7}{2550}{2500}{
	قابل فراخوانی با کارت مخصوص فراخوانی آیینی (Ritual)
}

\monster{Skull Guardian}{Resources/monsters/Skull.jpg}{Warrior/Ritual}{7}{2050}{2500}{
	قابل فراخوانی با کارت مخصوص فراخوانی آیینی (Ritual)
}

\monster{Slot Machine}{Resources/monsters/sm.jpg}{Machine}{7}{2000}{2300}{
	بدون توضیحات اضافه
}


\monster{Haniwa}{Resources/monsters/Haniwa.jpg}{Rock}{2}{500}{500}{
	بدون توضیحات اضافه
}

\monster{Man-Eater Bug}{Resources/monsters/MEB.jpg}{Insect/Effect}{2}{450}{600}{
	هنگامی که از پشت به رو برگردد می‌تواند یکی از هیولا های حریف را نابود کند
}


\monster{Gate Guardian}{Resources/monsters/GG.jpg}{Warrior/Effect}{11}{3750}{3400}{
	این هیولا را نمی‌توانید به صورت عادی در زمین قرار دهید اما می‌توانید با قربانی کردن ۳ هیولای دلخواه در زمین،
	مستقیما از دست آن را احضار ویژه کنید.
}

\monster{Scanner}{Resources/monsters/Scanner.jpg}{Machine/Effect}{1}{?}{?}{
	در هر نوبت، می‌توانید یکی از هیولاهای حریف که از بازی خارج شده است را انتخاب کنید تا برای آن نوبت به آن هیولا تبدیل شود.
	در این حالت حمله و دفاع و اسم و تمامی خاصیت‌های این شخصیت تا آخر همان نوبت (تا مرحله پایان) با کارت انتخاب شده حریف یکی می‌شود.
}

\monster{Bitron}{Resources/monsters/Bitron.jpg}{Cyberse}{2}{200}{2000}{
	بدون توضیحات اضافه
}


\monster{Marshmallon}{Resources/monsters/Marshmallon.jpg}{Fairy/Effect}{3}{300}{500}{
	این کارت نمی‌تواند در نبرد عادی نابود شود.
	زمانی که به این کارت حمله می‌شود، پس از محاسبه آسیب وارد شده به صاحب کارت، اگر کارت به پشت (\lr{face down}) بود، 
	از جان بازیکن حمله کننده ۱۰۰۰ واحد کم خواهد شد.

}



\monster{Beast King Barbaros}{Resources/monsters/BKB.jpg}{Beast-Warrior/Effect}{8}{3000}{1200}{
می‌توانید این کارت را بدون قربانی کردن دو هیولای دیگر به صورت عادی احضار کنید اما قدرت حمله‌ی آن به ۱۹۰۰ کاهش خواهد یافت. همچنین می‌توانید این کارت را با قربانی سه هیولا فقط احضار کنید (نه اینکه به پشت در زمین بگذارید)، در این صورت تمام کارت‌هایی که حریف کنترل می‌کند نابود خواهد شد.
}



\monster{Texchanger}{Resources/monsters/Texchanger.jpg}{Cyberse/Effect}{1}{100}{100}{
	تنها یکبار در هر نوبت، اگر این کارت مورد حمله واقع شود، حمله خنثی می‌شود و صاحب کارت می‌تواند یک هیولای عادی (بدون قابلیت خاص) از نوع \lr{Cyberse} را از دست، دک و یا گورستان خود به صورت ویژه احضار کند.
}


\monster{Leotron}{Resources/monsters/Leo.jpg}{Cyberse}{4}{2000}{0}{
	بدون توضیحات اضافه
}

\monster{The Calculator}{Resources/monsters/Calc.jpg}{Thunder/Effect}{2}{?}{0}{
	قدرت حمله‌ی این کارت برابر مجموع سطح هیولا‌های رو به بالای‌ صاحب کارت ضربدر ۳۰۰ است.
}


\monster{Alexandrite Dragon}{Resources/monsters/Alex.jpg}{Dragon}{4}{2000}{100}{
	بدون توضیحات اضافه
}


\monster{Mirage Dragon}{Resources/monsters/MD.jpg}{Dragon/Effect}{4}{1600}{600}{
زمانی که این کارت به رو در زمین باشد، حریف نمی‌تواند هیچ کارت تله‌ای را فعال کند.
}


\monster{Herald of Creation}{Resources/monsters/HOC.jpg}{Spellcaster/Effect}{4}{1800}{600}{
تنها یکبار در هر نوبت، می‌توانید با حذف یک کارت از دستتان، یک هیولای سطح ۷ و یا بالاتر را از گورستان خود به دستتان منتقل کنید.

}

\monster{Exploder Dragon }{Resources/monsters/ED.jpg}{Dragon/Effect}{3}{1000}{0}{
اگر این کارت در نبرد کشته شود و به گورستان منتقل شود، کارتی را که او را نابود کرده است نابود می‌کند و در این میان، جان هیچ‌کدام از بازیکنان کم نخواهد شد و تنها کارت‌ها از بین می‌روند.
}

\monster{Warrior Dai Grepher }{Resources/monsters/WDG.jpg}{Warrior}{4}{1700}{1600}{
	بدون توضیحات اضافه
}


\monster{Dark Blade }{Resources/monsters/DB.jpg}{Warrior}{4}{1800}{1500}{
	بدون توضیحات اضافه
}


\monster{Wattaildragon  }{Resources/monsters/wat.jpg}{Dragon}{6}{2500}{1000}{
	بدون توضیحات اضافه
}


\monster{Terratiger, the Empowered Warrior}{Resources/monsters/TEMW.jpg}{Warrior/Effect}{4}{1800}{1200}{
زمانی که این کارت به صورت عادی احضار شده باشد، صاحب کارت می‌تواند از دستش یک هیولای عادی سطح ۴ یا پایین‌تر را در حالت تدافعی احضار کند.	
}


\monster {The Tricky}{Resources/monsters/TT.jpg}{Spellcaster/Effect}{5}{2000}{1200}{
اگر این کارت در دست شما موجود باشد، می‌توانید آن را با حذف یک کارت از دستتان، به صورت ویژه احضار کنید.
	
}




\monster{Spiral Serpent}{Resources/monsters/SS.jpg}{Sea Serpent}{8}{2900}{2900}{
	بدون توضیحات اضافه	
}











\subsection*{{\titr کارت‌های \lr{Spell} و \lr{Trap}}}
\addcontentsline{toc}{subsection}{{\fehrestContent کارت‌های \lr{Spell} و 
\lr{Trap}}}

\spell{Monster Reborn}{Resources/spells/MR.jpg}{افسون}{عادی - محدود}{
	با این کارت شما می‌توانید یک از هیولاها را از گورستان خود یا حریف انتخاب کنید و آن را به صورت احضار ویژه به بازی برگردانید.
}


\spell{Terraforming}{Resources/spells/T.jpg}{افسون}{عادی - محدود}{
یک عدد کارت افسون میدانی(\lr{Field Spell}) را از دکی که در اختیار دارید، به دست خود اضافه کنید.
}

\spell{Pot of Greed}{Resources/spells/POG.jpg}{افسون}{عادی - محدود}{
	۲ کارت از بالای دک بردارید.
}



\spell{Raigeki}{Resources/spells/R.jpg}{افسون}{عادی - محدود}{
	تمام هیولاهایی را که حریفتان کنترل می‌کند نابود کنید.
}


\spell{Change of Heart}{Resources/spells/coh.jpg}{افسون}{عادی - محدود}{
یکی از هیولاهایی که حریفتان کنترل می‌کند را مورد هدف قرار دهید و تا انتهای این نوبت از بازی آن را به کنترل خود درآورید.
}



\spell{Harpie's Feather Duster}{Resources/spells/hf.jpg}{افسون}{عادی - محدود}{
	تمام کارت‌های افسون و تله که حریفتان کنترل می‌کند را نابود کنید.
}



\spell{Swords of Revealing Light}{Resources/spells/sor.jpg}{افسون}{عادی - نامحدود}{
	پس از آنکه این کارت را فعال کردید، به مدت سه نوبت حریف در زمین بازی باقی خواهد ماند و سپس نابود می‌شود. زمانی که این کارت فعال می‌شود:
	
	هیولاهای حریف که به پشت قرار دارند را به رو بچرخانید. تا زمانی که این کارت به رو در زمین بازی قرار دارد، کارت‌های هیولای حریف نمی‌توانند اعلام حمله کنند.
	
}



\spell{Dark Hole}{Resources/spells/DH.jpg}{افسون}{عادی - نامحدود}{
تمام کارت‌های هیولای موجود در زمین بازی را نابود کنید.
}



\spell{Supply Squad}{Resources/spells/SS.jpg}{افسون}{عادی - نامحدود}{
	در هر نوبت اگر حداقل یکی از هیولاهایتان در نبرد یا به خاطر اثر یک کارت نابود شود، ۱ کارت از بالای دک بردارید.
}


\spell{Spell Absorption}{Resources/spells/sa.jpg}{افسون}{عادی - نامحدود}{
هر زمانی که یک کارت افسون فعال شد، به محض اعمال شدنش ۵۰۰ واحد جان (LP) دریافت کنید.
}



\spell{Messenger of peace}{Resources/spells/mop.jpg}{افسون}{Continuous  - نامحدود}{
هیولا هایی که ATK معادل 1500 یا بیشتر را دارند نمی‌توانند اقدام به حمله کنند. یکبار در هر نوبت از بازی، در طول فاز \lr{Standby}  به میزان ۱۰۰ واحد از جان (LP) خود را بپردازید یا این کارت را نابود کنید.
}



\spell{Twin Twisters}{Resources/spells/tt.jpg}{افسون}{\lr{Quick Play} - نامحدود}{
یکی از کارت‌های دستتان را دور بریزید؛ سپس حداکثر 2 کارت افسون یا تله موجود در زمین بازی را مورد هدف قرار دهید و نابودشان کنید.
}



\spell{Mystical space typhoon}{Resources/spells/ms.jpg}{افسون}{\lr{Quick Play} - نامحدود}{
یک کارت افسون یا تله موجود در زمین بازی را مورد هدف قرار داده و سپس نابودش کنید.
}

\spell{Ring of Defense}{Resources/spells/rod.jpg}{افسون}{\lr{Quick Play} - نامحدود}{
	هرگاه یک کارت تله با اثر تخریب کننده فعال شود، آن آسیب را به 0 برسانید.
}

\spell{Yami}{Resources/spells/Y.jpg}{افسون}{\lr{Field} - نامحدود}{
تمام  هیولاهای موجود در زمین بازی که از نوع  \lr{Fiend}  و  \lr{Spellcaster}  هستند، 200 واحد ATK/DEF  دریافت می‌کنند؛ همچنین تمام هیولاهای  از نوع   \lr{Fairy} که در زمین بازی قرار دارند، 200 واحد ATK/Def  از دست می‌دهند.
}

\spell{Forest}{Resources/spells/F.jpg}{افسون}{\lr{Field} - نامحدود}{
تمام  هیولاهای موجود در زمین بازی که از نوع  \lr{Insect/Beast/Beast-Warrior}  هستند، 200 واحد ATK/Def  دریافت می‌کنند.
}


\spell{Closed Forest}{Resources/spells/CF.jpg}{افسون}{\lr{Field} - نامحدود}{
تمام  هیولاهای \lr{Beast-Type}  که خودتان کنترل می‌کنید، به ازای هر هیولای موجود در گورستان،  100 واحد ATK  دریافت می‌کنند.
}

\spell{UMIIRUKA}{Resources/spells/U.jpg}{افسون}{\lr{Field} - نامحدود}{
	برای تمام کارت‌های هیولا از نوع  Aqua به میزان 500 امتیاز به ATK اضافه کرده و 400 امتیاز از DEF کم کنید.
}

\spell{Sword of Dark Destruction}{Resources/spells/sd.jpg}{افسون}{\lr{Equip} - نامحدود}{
	کارت‌های هیولای مجهز شده (equipped) با این کارت که از نوع Fiend یا Spellcaster باشند، 400 واحد ATK دریافت می‌کنند و 200 واحد از DEF آن‌ها کم می‌شود.
}

\spell{Black Pendant}{Resources/spells/bp.jpg}{افسون}{\lr{Equip} - نامحدود}{
کارت‌های هیولای مجهز شده (equipped) با این کارت، 500 واحد ATK دریافت می‌کنند.
}

\spell{United We Stand}{Resources/spells/uws.jpg}{افسون}{\lr{Equip} - نامحدود}{
کارت‌های هیولای مجهز شده (equipped) به این کارت، به ازای هر هیولایی که توسط شما کنترل می‌شود و در زمین بازی به  رو قرار گرفته است 800 واحد ATK/DEF دریافت می‌کنند. 
}

\spell{Magnum Shield}{Resources/spells/msh.jpg}{افسون}{\lr{Equip} - نامحدود}{
فقط هیولاهایی از نوع Warrior  را مجهز می‌کند. تاثیر این کارت با توجه به حالت قرار گرفتن کارت هیولای مورد نظر در زمین بازی است:

اگر کارت هیولا در حالت حمله باشد:  آن کارت دقیقا معادل مقدار DEF خود، امتیاز ATK دریافت می‌کند.

اگر کارت هیولا در حالت دفاع باشد:  آن کارت دقیقا معادل مقدار ATK خود، امتیاز DEF دریافت می‌کند.

}

\spell{Advanced Ritual Art}{Resources/spells/ARA.jpg}{افسون}{\lr{Ritual} - نامحدود}{
با این کارت می‌توانید یک هیولای آیینی (\lr{Ritual Monster}) را احضار آیینی (\lr{Ritual Summon}) کنید.

علاوه بر این شما باید تعدادی کارت هیولای عادی از دک انتخاب کنید که مجموع سطحشان برابر با سطح کارت هیولای آیینی که قصد احضارش را دارید باشد. سپس این کارت‌هارا از دک به گورستان بفرستید و هیولای آیینی مدنظر را احضار کنید.

	
}

\subsection*{{\titr کارت‌های تله}}
\addcontentsline{toc}{subsection}{{\fehrestContent کارت‌های تله}}


\spell{Magic Cylinder}{Resources/traps/MC.jpg}{تله}{عادی - نامحدود}{
	زمانی که کارت هیولای حریف، اعلام به حمله می‌کند: هیولای مهاجم را هدف قرار دهید، حمله را خنثی کنید و در صورت انجام موفق این کار، معادل ATK هیولای حمله‌کننده به حریف آسیب وارد کنید.
}

\newpage


\spell{Mirror Force}{Resources/traps/mf.jpg}{تله}{عادی - نامحدود}{
زمانی که هیولای حریف، اعلام حمله می‌کند: تمام هیولا‌های حریف را که در حالت حمله هستند نابود کنید.
}





\spell{Mind Crush}{Resources/traps/mcr.jpg}{تله}{عادی - نامحدود}{
نام یک کارت را اعلام کنید؛ اگر آن کارت در دست حریف بود، باید تمام نسخه های آن را کنار بگذارد. 
در غیر این صورت شما باید یک کارت تصادفی را از دستتان دور بیندازید.
}



\spell{Trap Hole}{Resources/traps/th.jpg}{تله}{عادی - نامحدود}{
زمانی که حریف شما یک هیولا با امتیاز حمله ۱۰۰۰ یا بیشتر را به صورت عادی یا چرخشی (\lr{Flip}) احضار کند، 
با فعال کردن این تله آن هیولا را نابود کنید.
}


\spell{Torrential Tribute}{Resources/traps/tt.jpg}{تله}{عادی - نامحدود}{
	زمانی که هیولایی احضار می‌شود، با فعال کردن این کارت تمام هیولا‌های زمین را نابود کنید.
}



\spell{Time Seal}{Resources/traps/ts.jpg}{تله}{عادی - محدود}{
	حریف در نوبت بعدی نمی‌تواند در مرحله برداشت کارتی بردارد.
}

\spell{Negate Attack}{Resources/traps/na.jpg}{تله}{عادی - نامحدود}{
هنگامی که هیولای حریف حمله خود را اعلام می‌کند، با فعال کردن این کارت، حمله را خنثی کنید و مرحله نبرد را خاتمه دهید.
}


\spell{Solemn Warning }{Resources/traps/sw.jpg}{تله}{عادی - نامحدود}{
هنگامی که یک هیولا احضار می‌شود، یا وقتی یک احضار ویژه بر اثر  افسون، تله یا اثر یک هیولا رخ می‌دهد، 2000 واحد جان (LP) پرداخت کنید؛ احضار یا فعال سازی را منتفی کنید، و اگر این کار را با موفقیت انجام دادید، آن کارت را نابود کنید.
}


\spell{Magic Jammer }{Resources/traps/mj.jpg}{تله}{عادی - نامحدود}{
هنگامی که کارت افسون فعال می‌شود، 1 کارت را از دستتان دور بیندازید. فعال سازی را منتفی کنید و در صورت انجام موفق این کار، آن کارت افسون را از بین ببرید.
}



\spell{Call of the Haunted }{Resources/traps/coh.jpg}{تله}{عادی - نامحدود}{
یک کارت را از گورستان خود در حالت حمله به زمین بازی احضار کنید.
}




\end{document}







