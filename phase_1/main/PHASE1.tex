\documentclass[]{article}
\usepackage{graphicx}
\usepackage[svgnames]{xcolor} 
\usepackage{fancyhdr}
\usepackage{fancyvrb}
\usepackage{tocloft}
\usepackage[hidelinks]{hyperref}
\usepackage{enumitem}
\usepackage[many]{tcolorbox}
\usepackage{listings }
%\usepackage[a4paper, total={6in, 8in} , top = 2cm,bottom = 4cm]{geometry}
\usepackage[a4paper, total={6in, 8in}]{geometry}
\usepackage{afterpage}
\usepackage{amssymb}
\usepackage{pdflscape}
\usepackage{textcomp}
\usepackage{xecolor}
\usepackage{rotating}
\usepackage[Kashida]{xepersian}
\usepackage[T1]{fontenc}
\usepackage{tikz}
\usepackage[utf8]{inputenc}
\usepackage{PTSerif} 
\usepackage{seqsplit}
\usepackage{changepage}


\usepackage{listings}
\usepackage{xcolor}
\usepackage{sectsty}

\setcounter{secnumdepth}{0}
 
\definecolor{codegreen}{rgb}{0,0.6,0}
\definecolor{codegray}{rgb}{0.5,0.5,0.5}
\definecolor{codepurple}{rgb}{0.58,0,0.82}
\definecolor{backcolour}{rgb}{0.95,0.95,0.92}
\definecolor{blanchedalmond}{rgb}{1.0, 0.92, 0.8}
\definecolor{brilliantlavender}{rgb}{0.96, 0.73, 1.0}
 
\NewDocumentCommand{\codeword}{v}{
\texttt{\textcolor{blue}{#1}}
}
\lstset{language=java,keywordstyle={\bfseries \color{blue}}}

\lstdefinestyle{mystyle}{
    backgroundcolor=\color{backcolour},   
    commentstyle=\color{codegreen},
    keywordstyle=\color{magenta},
    numberstyle=\tiny\color{codegray},
    stringstyle=\color{codepurple},
    basicstyle=\ttfamily\normalsize,
    breakatwhitespace=false,         
    breaklines=true,                 
    captionpos=b,                    
    keepspaces=true,                 
    numbers=left,                    
    numbersep=5pt,                  
    showspaces=false,                
    showstringspaces=false,
    showtabs=false,                  
    tabsize=2
}

\lstset{style=mystyle}

 \settextfont[BoldFont={XB Zar bold.ttf}]{XB Zar.ttf}


\setlatintextfont[Scale=1.0,
 BoldFont={LiberationSerif-Bold.ttf}, 
 ItalicFont={LiberationSerif-Italic.ttf}]{LiberationSerif-Regular.ttf}





\newcommand{\inputsample}[1]{
    ~\\
    \textbf{ورودی نمونه}
    ~\\
    \begin{tcolorbox}[breakable,boxrule=0pt]
        \begin{latin}
            \large{
                #1
            }
        \end{latin}
    \end{tcolorbox}
}

\newcommand{\outputsample}[1]{
    ~\\
    \textbf{خروجی نمونه}

    \begin{tcolorbox}[breakable,boxrule=0pt]
        \begin{latin}
            \large{
                #1
            }
        \end{latin}
    \end{tcolorbox}
}

\newtcolorbox{mybox}[2][]{colback=red!5!white,
colframe=red!75!black,fonttitle=\bfseries,
colbacktitle=red!85!black,enhanced,
attach boxed title to top center={yshift=-2mm},
title=#2,#1}

\newenvironment{changemargin}[2]{%
\begin{list}{}{%
\setlength{\topsep}{0pt}%
\setlength{\leftmargin}{#1}%
\setlength{\rightmargin}{#2}%
\setlength{\listparindent}{\parindent}%
\setlength{\itemindent}{\parindent}%
\setlength{\parsep}{\parskip}%
}%
\item[]}{\end{list}}


\definecolor{foldercolor}{RGB}{124,166,198}
\definecolor{sectionColor}{HTML}{ff5e0e}
\definecolor{subsectionColor}{HTML}{008575}

\definecolor{listColor}{HTML}{00d3b9}

\definecolor{umlrelcolor}{HTML}{3c78d8}

\definecolor{subsubsectionColor}{HTML}{3c78d8}

\defpersianfont\authorFont[Scale=0.9]{XB Zar bold.ttf}

\defpersianfont\titr[Scale=1.5]{Lalezar-Regular.ttf}

\defpersianfont\fehrest[Scale=1.2]{Lalezar-Regular.ttf}

\defpersianfont\fehrestTitle[Scale=3.0]{Lalezar-Regular.ttf}

\defpersianfont\fehrestContent[Scale=1.2]{XB Zar bold.ttf}


\sectionfont{\color{sectionColor}}  % sets colour of sections
\subsectionfont{\color{subsectionColor}}  % sets colour of sections
\subsubsectionfont{\color{subsubsectionColor}}


\renewcommand{\labelitemii}{$\circ$}


\renewcommand{\baselinestretch}{1.1}


\renewcommand{\contentsname}{فهرست}

\renewcommand{\cfttoctitlefont}{\fehrestTitle}


\renewcommand\cftsecfont{\color{sectionColor}\fehrestContent\selectfont}
\renewcommand\cftsubsecfont{\color{subsectionColor}\fehrestContent\selectfont}
\renewcommand\cftsubsubsecfont{\color{subsubsectionColor}\fehrestContent\selectfont}
%\renewcommand{\cftsecpagefont}{\color{sectionColor}}

\setlength{\parskip}{1.2pt}

\begin{document}


%%% title pages
\begin{titlepage}
\begin{center}

\textbf{ \Huge{به نام خدا} }
        
\vspace{0.2cm}

\includegraphics[width=0.4\textwidth]{sharif1.png}\\
\vspace{0.2cm}
\textbf{ \Huge{\emph درس برنامه‌سازی پیشرفته} }\\
\vspace{0.25cm}
\textbf{ \Large{ فاز اول پروژه} }
\vspace{0.2cm}
       
 
      \large \textbf{دانشکده مهندسی کامپیوتر}\\\vspace{0.1cm}
    \large   دانشگاه صنعتی شریف\\\vspace{0.2cm}
       \large   ﻧﯿﻢ سال دوم 00-99 \\\vspace{0.10cm}
      \noindent\rule[1ex]{\linewidth}{1pt}
استاد:\\
    \textbf{{دکتر محمدامین فضلی}}



    \vspace{0.20cm}

   مهلت ارسال:\\
    \textbf{{۲۳ اردیبهشت - }}
    \textbf{{ساعت 23:59:59}}

    \vspace{0.10cm}
مسئول پروژه:\\
    \textbf{\authorFont{امیرمهدی نامجو}}
    
        \vspace{0.10cm}
مسئول فاز اول:\\
    \textbf{\authorFont{عرشیا اخوان}}
    
        \vspace{0.10cm}
طراحان فاز اول:\\
    \textbf{\authorFont{سروش جهان‌زاد، متین شجاع، امیرصدرا عبدالهی، امیرمهدی کوششی، یاسمین گلزار، امیرحسین هادیان }}
    
        \vspace{0.05cm}
مسئولین تنظیم مستند:\\
    \textbf{\authorFont{سروش جهان‌زاد و پارسا محمدیان}}
    

\end{center}
\end{titlepage}
%%% title pages


%%% header of pages
\newpage
\pagestyle{fancy}
\fancyhf{}
\fancyfoot{}
\cfoot{\thepage}
\lhead{فاز اول}
\rhead{\includegraphics[width=0.1\textwidth]{sharif.png}\\
دانشکده مهندسی کامپیوتر
}
\chead{پروژه برنامه‌سازی پیشرفته}
%%% header of pages
\renewcommand{\headrulewidth}{2pt}

\KashidaOff



\tableofcontents

\newpage

 \Large \textbf{\\\\
}


\section*{{\titr نکات قابل توجه}}
\addcontentsline{toc}{section}{{\fehrestContent نکات قابل توجه}}
\begin{itemize}
\item
پس از اتمام این فاز، در گیت خود یک تگ با ورژن \lr{"v1.0.0"} بزنید. در روز تحویل حضوری این tag بررسی خواهد شد و کدهای پس از آن نمره‌ای نخواهد گرفت. برای اطلاعات بیش‌تر در مورد شیوه ورژن‌گذاری، می‌توانید به
 \href{https://semver.org/}{\textcolor{blue}{\underline{این لینک}}}
 مراجعه کنید. البته برای این پروژه صرفا رعایت کردن همان ورژن گفته شده کافیست، اما خوب‌ است که با منطق ورژن‌بندی هم آشنا بشوید.

\item
در روز تحویل حضوری مشارکت تمام اعضای تیم در پروژه بررسی خواهد‌ شد و در صورت عدم مشارکت بعضی از اعضا، نمره‌ی ایشان برای آن فاز پروژه "صفر" لحاظ می‌گردد. مشارکت، با توجه به commit های افراد تیم در مخزن گیت‌هاب پروژه بررسی می‌شود.

\item
در هر فاز می‌توانید سه روز تاخیر به ازای کسر نمره داشته‌ باشید که به ازای هر روز آن، ۱۰ درصد از نمرهٔ آن فاز را از دست خواهید‌ داد. در مجموع سه‌فاز پروژه، سه روز تاخیر نیز بخشیده خواهد‌ شد.

\item
به ازای هر ساعتی که پروژه را زودتر تحویل دهید، ۱۵ دقیقه به مهلت تاخیر بدون کسر نمره شما اضافه خواهد‌ شد. این مقدار حداکثر یک روز خواهد‌ بود که در صورت ارسال ۴ روز زودتر از ددلاین به شما تعلق خواهد گرفت. \textbf{بنابراین ددلاین‌های پروژه تحت هیچ شرایطی تمدید نخواهد‌ شد.} توصیه می‌شود با برنامه‌ریزی مناسب به ددلاین‌های درس پایبند باشید.

\item
در صورت کشف تقلب از هریک از تیم‌ها، برای بار اول منفی نمرهٔ آن فاز برای آن تیم ثبت می‌شود و برای بار دوم، نمرهٔ منفی کل پروژه برای تیم لحاظ خواهد‌ شد که معادل مردود شدن در درس است.
\end{itemize}

\newpage

\section*{{\titr مقدمه}}
\addcontentsline{toc}{section}{{\fehrestContent مقدمه}}

\subsection*{{\titr اهداف پروژه}}

\addcontentsline{toc}{subsection}{{\fehrestContent اهداف قابل توجه}}

\begin{itemize}

\item
هدف این پروژه، طراحی یک بازی کارتی مشابه عنوان \lr{Yu-Gi-Oh!} است. فاز اول به شکل عمده به منطق بازی اختصاص دارد.


\item
در این فاز از پروژه، طراحی شی‌ءگرای بازی و جداسازی صحیح منطق بخش‌های مختلف از یکدیگر مورد نظر است.

\item
یکی از اهداف پروژه، آشنایی با برخی ابزارها و Pattern های استاندارد برنامه‌نویسی، مثل ابزار مدیریت و تعریف پروژه‌ٔ 
\href{https://en.wikipedia.org/wiki/Apache_Maven}{\textcolor{blue}{\underline{\lr{Apache Maven}}}}، 
است.

\item
آشنایی با سیستم مدیریت نسخه \lr{Git} و کار تیمی بر روی پروژه بر بستر یک مخزن \lr{Github}، یکی از اهداف مهم پروژه است. در این مورد توصیه می‌شود تغییرات خود را در دوره‌های کوتاه مدت \lr{commit} کنید.

\end{itemize}

\subsection*{{\titr کلیات پروژه}}
\addcontentsline{toc}{subsection}{{\fehrestContent کلیات پروژه}}

در این فاز، صرفا منطق پروژه، بدون پیاده‌سازی گرافیک یا معماری شبکهٔ آن، باید پیاده‌سازی شود. نحوهٔ ارتباط با کاربر نیز از طریق واسط کاربری کنسول است. توجه داشته باشید که در فاز سوم پروژه، باید سیستم را طبق یک معماری سرور-کلاینت طراحی کنید. در معماری اکثر بازی‌هایی که به این شکل انجام می‌شوند، منطق بازی در سرور و مستقل از واسط کاربری سمت کلاینت و کاربر است. هر چند برای فاز اول نیاز به این مسئله ندارید ولی خوب‌ است که از الآن طراحی خوبی داشته باشید که قسمت‌های مختلف پروژه نظیر منطق اصلی انجام بازی، وابستگی اساسی به بخش‌هایی نظیر واسط کاربری نداشته باشد.

در ادامهٔ مستند، موجودیت‌ها، نمای کلی رابط کاربری سیستم، نقش‌ها و دستورات لازم شرح داده‌شده است.

\begin{enumerate}[label={نکته \arabic*:}]
\item
 تمامی اطلاعات، اعم از اطلاعات کاربران، کارت‌ها و... باید در خارج از برنامه (مثلا روی فایل) ذخیره شوند و پس از \lr{terminate} شدن برنامه و اجرای مجدد آن، بصورت خودکار اطلاعات قبلی خوانده شود و قابل دسترسی باشد. برای این کار می‌توانید از ابزارهای کار با \lr{Json} در جاوا، مثل
  \href{https://www.tutorialspoint.com/gson/gson_quick_guide.htm}{\textcolor{blue}{\lr{Gson}}}
   استفاده‌ کنید.

\item
شما باید پروژه‌ی خود را بر بستر ابزار \lr{Apache Maven} پیاده‌سازی کنید. همچنین برای اضافه کردن کتابخانه‌های مورد نیاز، \lr{dependency} های مربوطه را به فایل \lr{pom.xml} اضافه کنید.




   

\item
در هر جایی از پروژه می‌توانید هرگونه خلاقیتی را به‌کار ببرید. با این حال توجه کنید که خواسته‌های واضح پروژه بایستی انجام شوند و سیستم ورودی گرفتن و خروجی دادن شما باید مطابق جزییات گفته شده در این مستند باشد.


\item
در مستند بعضی از دستور‌هایی که مشاهده می‌کنید فرمتی به شکل زیر دارند:

\begin{mybox}[colback=yellow]{دستور}
	
	
	\begin{latin}
		
	user login -{}-username <username> -{}-password <password>
		
	\end{latin}
	
\end{mybox}

در چنین مواردی که شامل پارامتر‌هایی هستند که با
\lr{-{}-}
یعنی دو کاراکتر دش، مشخص شده‌اند، باید بتوانیم از دستور با هر ترتیبی استفاده کنیم. یعنی دستور زیر

\begin{mybox}[colback=yellow]{دستور}
	
	
	\begin{latin}
		
		user login -{}-password <password> -{}-username <username> 
		
	\end{latin}
	
\end{mybox}

هم دستور معتبری است.

همچنین به عنوان نمره امتیازی، می‌توانید حالت مخفف شده را هم برای آنان پیاده‌سازی کنید. نحوه مخفف سازی به صورت تک حرفی با و با یک نماد -
خواهد بود و نحوه انتخاب حروف برعهده خودتان است. مثلا برای نمونه بالا چنین عبارتی می‌تواند پیشنهاد خوبی باشد:

\begin{mybox}[colback=yellow]{دستور}
	
	
	\begin{latin}
		
		user login -p <password> -u <username> 
		
	\end{latin}
	
\end{mybox}


\end{enumerate}




\newpage


\section*{{\titr معرفی بازی}}
\addcontentsline{toc}{section}{{\fehrestContent معرفی بازی}}


\textbf{این دو عبارت زیر مربوط به داک سال قبلن. صرفا برای این که قالب نوشتن این مدل دستورات که label خاص کنارشون هست یا توی حاشیه خاص هستن دستتون باشه، گذاشتم. این خط رو پاک کنید بعدا. }TODO

\begin{itemize}[label=$\blacksquare$]
	\item
	نام کاربری، نام، نام‌خانوادگی، ایمیل، شماره تلفن، رمز عبور 
\end{itemize}



\begin{mybox}[colback=yellow]{دستور}
	
	
	\begin{latin}
		
		create account [type] [username]
		
	\end{latin}
	
\end{mybox}

\section*{{\titr توضیح بخش‌های مختلف پروژه}}
\addcontentsline{toc}{section}{{\fehrestContent توضیح بخش‌های مختلف پروژه}}



\subsection*{{\titr کاربر، منوها و موارد عمومی}}
\addcontentsline{toc}{subsection}{{\fehrestContent کاربر، منوها و موارد عمومی
}}

در این بازی مانند تمام بازی‌های دیگر یک عده کاربر وجود دارند که بازی می‌کنند. 
هر کاربر باید فیلد‌های زیر را داشته باشد:
\begin{itemize}[label=$\blacksquare$]
	\begin{latin}
		\item username
		\item password
		\item nickname
		\item score
	\end{latin}
\end{itemize}
در ابتدای بازی هر فرد باید ثبت نام کند و سپس در دفعات بعدی فقط وارد شود و 
بازی کند.
\\
همچنین یک تابلو امتیازات وجود دارد که کاربران را بر اساس امتیازی که دارند، 
رتبه‌بندی می‌کند و با نام مستعارشان نمایش می‌دهد.
\\
در فاز‌ آخر پروژه نیز باید امکان چت (\lr{global chat}) میان کاربران فراهم 
شود؛ اما برای این فاز با توجه به نبود شبکه، نیازی به پیاده‌سازی آن نیست.
\\
پس از ورود به بازی، وارد منوی ورود و ثبت‌نام (\lr{Login Menu}) می‌شویم و پس از 
ساخت اکانت و وارد شدن کاربر به منوی اصلی (\lr{Main Menu}) می‌رویم که راه 
ارتباطی میان تمام اجزای مختلف بازی است. لیست منو‌هایی که باید در منوی اصلی 
پشتیبانی شوند به شرح زیر است:

\begin{itemize}[label=$\blacksquare$]
	\begin{latin}
		\item New Game
		\item Deck
		\item Shop
		\item Profile
		\item Scoreboard
		\item Exit
	\end{latin}
\end{itemize}
دستورات هر منو فقط داخل آن‌ها معتبر است و اگر در منوی مربوطه صدا زده نشوند، 
باید خطای مناسب چاپ شود.

\subsubsection*{{\titr دستورات مرتبط با کاربر و منو}}
\addcontentsline{toc}{subsubsection}{{\fehrestContent دستورات مرتبط با کاربر و 
منو}}
\vspace{.5cm}
\textbf{ورود به یک منو:}
\begin{mybox}[colback=yellow]{دستور}
	\begin{latin}	
		menu enter <menu name>
	\end{latin}
\end{mybox}
در صورتی که کاربر در منوی اصلی باشد و وارد منوی \lr{Exit} شود، بازی باید 
خاتمه یابد.
\\
در صورتی که کاربر داخل یک منوی دیگر باشد باید خطای زیر چاپ شود:
\\
\begin{mybox}[colback=yellow]{پیغام به کاربر}
	\begin{latin}	
		menu navigation is not possible
	\end{latin}
\end{mybox}

\vspace{.5cm}
\textbf{خروج از یک منو:}
\begin{mybox}[colback=yellow]{دستور}
	\begin{latin}	
		menu exit
	\end{latin}
\end{mybox}
در صورتی داخل یک منو باشیم، این دستور ما را به منوی بالاتر می‌برد و اگر در 
منوی اصلی باشیم وارد منوی ورود و ثبت‌نام خواهیم شد. در صورتی که در منوی 
ورود 
و ثبت‌نام بودیم، بازی خاتمه خواهد یافت.

\vspace{.5cm}
\textbf{منوی فعلی:}
\begin{mybox}[colback=yellow]{دستور}
	\begin{latin}	
		menu show-current 
	\end{latin}
\end{mybox}
این دستور نام منوی فعلی را نشان می‌دهد و اگر در منوی اصلی باشیم 
\lr{Main Menu} و اگر در منوی ورود و ثبت‌نام باشیم \lr{Login Menu} چاپ می‌شود.

\vspace{.5cm}
\textbf{ساخت کاربر جدید:}
\begin{mybox}[colback=yellow]{دستور}
	\begin{latin}	
		user create -{}-username <username> -{}-nickname <nickname> 
		-{}-password 
		<password>
	\end{latin}
\end{mybox}
پیغام موفقیت:
\begin{mybox}[colback=yellow]{پیغام به کاربر}
	\begin{latin}	
		user created successfully!
	\end{latin}
\end{mybox}
خطا‌های زیر در صورت وجود به همین ترتیب بررسی شوند:
\\
نام کاربری تکراری:
\begin{mybox}[colback=yellow]{پیغام به کاربر}
	\begin{latin}	
		user with username <username> already exists
	\end{latin}
\end{mybox}
نام مستعار تکراری:
\begin{mybox}[colback=yellow]{پیغام به کاربر}
	\begin{latin}	
		user with nickname <nickname> already exists
	\end{latin}
\end{mybox}

\vspace{.5cm}
\textbf{ورود کاربر:}
\begin{mybox}[colback=yellow]{دستور}
	\begin{latin}	
		user login -{}-username <username> -{}-password <password>
	\end{latin}
\end{mybox}
پیغام موفقیت:
\begin{mybox}[colback=yellow]{پیغام به کاربر}
	\begin{latin}	
		user logged in successfully!
	\end{latin}
\end{mybox}
خطا‌های زیر در صورت وجود به همین ترتیب بررسی شوند:
\\
عدم وجود کاربر با این نام کاربری:
\begin{mybox}[colback=yellow]{پیغام به کاربر}
	\begin{latin}	
		Username and password didn’t match!
	\end{latin}
\end{mybox}
رمز اشتباه:
\begin{mybox}[colback=yellow]{پیغام به کاربر}
	\begin{latin}	
		Username and password didn’t match!
	\end{latin}
\end{mybox}
توجه کنید که از دیدگاه امنیت، یک تلاش ناموفق برای ورود هیچ‌گاه نباید دارای 
اطلاعاتی باشد که فرایند ورود را برای بار دوم راحت‌تر کند. مثلا اگر فردی 
صرفا 
قصد دزدیدن یکسری اکانت بدون توجه به صاحب آن‌ها را داشته باشد و در نتیجه 
یکسری نام‌کاربری تصادفی را امتحان کند، در صورتی که نام‌کاربری و پسورد مطابقت 
نداشت، نباید از پیام ما متوجه بشود که نام‌کاربری که وارد کرده، واقعا در 
سیستم وجود دارد. در نتیجه در هر دو حالت بالا پیغام خطای یکسانی چاپ می‌شود.

\vspace{.5cm}
\textbf{خروج کاربر:}
\begin{mybox}[colback=yellow]{دستور}
	\begin{latin}	
		user logout
	\end{latin}
\end{mybox}
این دستور تنها در منوی اصلی معتبر است و در صورت موفقیت، وارد منوی ورودی و 
ثبت‌نام می‌شود.
\\
پیغام موفقیت:
\begin{mybox}[colback=yellow]{پیغام به کاربر}
	\begin{latin}	
		user logged out successfully!
	\end{latin}
\end{mybox}

\vspace{.5cm}
\textbf{جدول امتیازات:}
\begin{mybox}[colback=yellow]{دستور}
	\begin{latin}	
		scoreboard show
	\end{latin}
\end{mybox}
پس از ورود به منوی \lr{scoreboard} با اجرای دستور باید لیست کاربران که بر 
اساس امتیازشان به صورت نزولی مرتب شده است به همراه نام مستعارشان نمایش داده 
شود. اگر دو کاربر امتیاز برابر داشتند، نام مستعارشان را به ترتیب حروف الفبا 
اما با رتبه یکسان چاپ کنید. هر کاربر در یک خط و با فرمت زیر چاپ شود:
\begin{mybox}[colback=yellow]{پیغام به کاربر}
	\begin{latin}	
		rank- <nickname>: <score>
	\end{latin}
\end{mybox}
نمونه‌ای از خروجی به شکل زیر است:
\begin{mybox}[colback=yellow]{نمونه خروجی}
	\begin{latin}	
		1- shahin: 5000 \\
		2- hovakhshatara: 4000 \\
		3- ebrahim\_1379: 3000 \\
		3- the-ultimate-mahD: 3000 \\
		5- rostam dastan: 1000
	\end{latin}
\end{mybox}

\vspace{.5cm}
\textbf{پروفایل:}
\\
پس از ورود به منوی profile دستورات زیر معتبر هستند:
\\
\vspace{.5cm}
\textbf{تغییر نام مستعار:}
\begin{mybox}[colback=yellow]{دستور}
	\begin{latin}	
		profile change -{}-nickname <nickname>
	\end{latin}
\end{mybox}
پیغام موفقیت:
\begin{mybox}[colback=yellow]{پیغام به کاربر}
	\begin{latin}	
		nickname changed successfully!
	\end{latin}
\end{mybox}
در صورتی که کاربری با این نام مستعار وجود داشت خطای زیر چاپ شود:
\begin{mybox}[colback=yellow]{پیغام به کاربر}
	\begin{latin}	
		user with nickname <nickname> already exists
	\end{latin}
\end{mybox}

\vspace{.5cm}
\textbf{تغییر رمز عبور:}
\begin{mybox}[colback=yellow]{دستور}
	\begin{latin}	
		profile change -{}-password -{}-current <current password> -{}-new <new 
		password>
	\end{latin}
\end{mybox}
پیغام موفقیت:
\begin{mybox}[colback=yellow]{پیغام به کاربر}
	\begin{latin}	
		password changed successfully!
	\end{latin}
\end{mybox}
خطا‌های زیر در صورت وجود به همین ترتیب بررسی شوند:
\\
نادرست بودن رمز فعلی:
\begin{mybox}[colback=yellow]{پیغام به کاربر}
	\begin{latin}	
		Current password is invalid
	\end{latin}
\end{mybox}
یکی بودن رمز قدیم و جدید:
\begin{mybox}[colback=yellow]{پیغام به کاربر}
	\begin{latin}	
		please enter a new password
	\end{latin}
\end{mybox}

\subsection*{{\titr معرفی ساختار کارت‌ها}}
\addcontentsline{toc}{subsection}{{\fehrestContent معرفی ساختار کارت‌ها}}

به طور کلی سه دسته کارت در بازی داریم:
\begin{itemize}
	\item
	هیولا
	\item
	جادو
	
	\item
	تله
\end{itemize}

در ادامه به معرفی این کارت‌ها می‌پردازیم.

\subsubsection*{{\titr هیولا:}}
\addcontentsline{toc}{subsubsection}{{\fehrestContent هیولا}}
کارت‌های اصلی بازی هیولاها هستند. هیولاها دو مشخصه اصلی حمله و دفاع دارند که در مستندات قوانین بازی و بخش گیم‌پلی به شکل مفصل‌تر بررسی شده‌اند. بعضی از هیولاها صرفا برای حمله یا دفاع استفاده می‌شوند و ویژگی خاصی همراه خود ندارند اما بعضی دیگر، ویژگی‌های خاصی دارند که در شرایط مشخصی فعال شده و برای هر کارت به طور جداگانه مشخص شده است.

هیولاها مشخصه‌ای به نام سطح یا Level دارند که عموما براساس سطح حمله و دفاع آن‌ها مشخص شده است. هیولاهایی با سطح بالاتر از $5$ معمولا به شکل عادی قابل احضار به داخل زمین نیستند و باید از طریق قربانی کردن یک کارت دیگر و یا روش‌های دیگری که در ادامه مستند به آن‌ها اشاره شده است، وارد بازی شوند.

همچنین هیولاها مشخصه‌ای به نام نوع هم دارند. بعضی از کارت‌ها وجود دارند که قابلیت‌های نوع خاصی از هیولاها را در تقویت یا تضعیف می‌کنند و از این رو نوع هیولا هم اهمیت دارد.




\subsubsection*{{\titr معرفی کلی جادو و تله:}}
\addcontentsline{toc}{subsubsection}{{\fehrestContent معرفی کلی جادو و تله}}

هر کدام از کارت های Spell یا Trap، ویژگی برای معرفی دارند که در ادامه به هرکدام از این ویژگی‌ها می پردازیم: (امکان زوم کردن روی تصویر وجود دارد)



\begin{center}
	\includegraphics[width = 0.25 \textwidth]{Resources/st.png}
	
\end{center}


\begin{enumerate}
	
	\item
	نام کارت: کارت هایی با نام مشابه، به عنوان کارت‌هایی یکسان  در نظر گرفته می شوند. توجه داشته باشید که از هر نوع کارت، می‌توان حداکثر 3 نمونه در مجموع دک اصلی و دک فرعی قرار داد.
	
	\item
	نوع: کارت‌های اسپل با نماد  \lr{SPELL}  و کارت‌های ترپ با نماد \lr{TRAP}  در بازی شناخته می‌شوند.
	
	
	\item
	نماد: در کل 6 نماد وجود دارد که هر کدام، خواص ویژه ای از کارت مورد نظر را نشان می‌دهند. اگر کارتی هیچ کدام از این نمادها را نداشت، کارت عادی (\lr{Normal}) محسوب خواهد شد. دربخش بعدی به توضیح هرکدام از این نوع کارت‌ها خواهیم پرداخت.
	
	\begin{center}
		\includegraphics[width =1.0 \textwidth]{Resources/spelltrap.png}
		
	\end{center}
	
	
	
	\item
	 توضیحات کارت: شرایط فعال شدن و اثرات هر کارت در این قسمت آورده می‌شود. برای استفاده از هر کارت، باید توضیحات و دستور عملی که در این قسمت آمده است را با دقت بخوانید.
	
	\item
شماره ی کارت: این شماره کارت در طول بازی کارآمد نخواهد بود  و صرفا برای شناسایی هر کارت، از شماره‌ی درج شده روی آن استفاده می‌کنیم.
	
	
	
\end{enumerate}


\subsubsection*{{\titr جادو:}}
\addcontentsline{toc}{subsubsection}{{\fehrestContent جادو}}

در حالت کلی، کارت‌های اسپل یا جادو می‌توانند در فاز اصلی بازی فعال شوند و به شما برای پیشرفت در بازی کمک کنند. این کارت‌ها می توانند تاثیر بسزایی در پیشبرد هرچه بهتر بازی شما داشته باشند. از جمله مزایای کارت‌های اسپل می‌توان به نابود کردن کارت‌های دیگر یا قوی کردن مانستر‌ها اشاره کرد؛ پس توجه داشته باشید که از این کارت‌ها، در بهترین موقعیت‌های ممکن استفاده کنید!

\textbf{انواع کارت‌های جادو}

\begin{itemize}
	\item \lr{Normal}:
این کارت‌ها یکبار مصرف هستند. برای به کارگیری آنها باید به رقیب اعلام کنید که می خواهید از اسپل کارت خود استفاده کنید و سپس، آن را رو به بالا در فیلد بازی جایگذاری کنید. اگر کارت با موفقیت فعال شد، باید کارهایی که در بخش توضیحات کارت نوشته شده را انجام دهید و سپس آن را به Graveyard منتقل کنید.
	
	 \item \lr{Ritual}:
	 
	 از این کارت ها برای احضار آیینی (\lr{ritual summons})  استفاده می‌کنیم. نحوه ی به کارگیری این نوع اسپل کارت‌ها دقیقا مشابه اسپل کارت‌های Normal است.
	 
	 \item \lr{Continuous}: 
	 
	 این کارت‌ها تنها بعد از فعال‌سازی وارد زمین بازی می‌شوند و تاثیر آن‌ها تا زمانی ادامه خواهد داشت که کارت همچنان رو به بالا در زمین بازی باشد.
	 
	 \item \lr{Equip}:
	 
	 این کارت‌ها می‌توانند به یک مانستر به انتخاب خودتان (که رو به بالا در زمین بازی جایگذاری شده است) یک اثر اضافه کنند (که این تاثیر طبق توضیحات نوشته شده روی خود کارت اسپل می‌تواند روی مانستر‌های خودتان و یا حتی رقیبتان اعمال شود).
	 پس از فعال‌سازی، این کارت‌ها در زمین بازی قرار می‌گیرند و تنها روی \lr{Equipped Monster} ها می‌توانند تاثیر بگذارند، اما همچنان بخش \lr{Spell and Trap}  زمین بازی شما را اشغال می‌کنند.
	 اگر امکانش فراهم بود، برای راحت‌تر شدن فرایند بازی، بهتر است این کارت را دقیقا در ناحیه‌ی پشت مانستر مربوطه قرار دهید. زمانی که مانستر مربوط به این اسپل کارت نابود شود یا رو به پایین برگردانده و یا حتی از زمین بازی برداشته شود، این اسپل کارت هم نابود خواهد شد.
	 
	 
	 \item \lr{Field}:
	 
	 این کارت‌ها در قسمت \lr{Field Zone} قرار می‌گیرند. برای آنکه بتوانید از یک اسپل کارت \lr{Field} دیگر استفاده کنید، باید ابتدا کارت قبلی را به \lr{Graveyard} منتقل کنید.
	 بسیاری از این نوع اسپل کارت‌ها روی بازی هردو بازیکن تاثیر خواهند گذاشت. ممکن است که این نوع اسپل کارت رو به پایین نیز در زمین بازی قرار داده شود اما تنها زمانی که رو به بالا باشد، فعال خواهند شد.
	 
	
	\item \lr{Quick-Play}:
	
	این اسپل کارت‌ها نوع خاصی هستند که در هر یک از فازهای نوبت شما می‌توانند فعال شوند و  اگر از قبل آن را رو به پایین در زمین بازی قرار داده باشید، قابلیت فعال سازی این اسپل کارت را حتی در نوبت حریف هم دارید. البته توجه داشته باشید، در همان دور که کارت را در زمین بازی قرار می دهید، نمی‌توانید همزمان آن را فعال کنید.
	
	
\end{itemize}




\subsubsection*{{\titr تله:}}
\addcontentsline{toc}{subsubsection}{{\fehrestContent تله}}

این کارت‌ها مشابه کارت‌های اسپل، با اثر‌های متفاوتی که دارند به شما در طول بازی کمک خواهند کرد؛ اما تفاوت اصلی آنها با کارت‌های اسپل این است که شما می‌توانید کارت‌های ترپ را در نوبت حریف فعال کنید.

بسیاری از کارت‌های اسپل، آثار مثبتی برای حمله کردن دارند اما کارت‌های ترپ هنگامی که حریف در حال اجرا و بکارگیری استراتژی‌هایش است، می‌توانند او را غافلگیر کنند. استفاده از کارت‌های ترپ گاه می‌تواند چالش برانگیز باشد، چرا که شاید حریف شما قبل از اتمام اثر کارت ترپ، مجبور به انجام یک سری حرکات خاص شود.


\textbf{انواع کارت‌های تله}

\begin{itemize}
	\item \lr{Normal}:
پیش از آنکه قادر به فعال سازی یک کارت ترپ باشید، باید آن را در زمین بازی جایگذاری کنید. توجه داشته باشید، در همان دور که کارت را در زمین بازی قرار می دهید، نمی توانید همزمان آن را فعال کنید اما پس از اتمام آن دور، فعال سازی آن کارت ترپ در هر زمانی امکان پذیر خواهد بود.

ترپ کارت های \lr{Normal} یکبار مصرف هستند و پس از اثرگذاری در بازی به \lr{Graveyard} منتقل می شوند.(دقیقا مثل اسپل کارت‌های \lr{Normal})

زمانی که این ترپ کارت‌ها فعال شوند، اثر آن ها غیر قابل توقف است؛ اگرچه ترپ کارتی که رو به پایین در زمانی بازی قرار داشته باشد، می تواند در همان دور که جایگذاری شده است و  تا قبل از فعال سازی، توسط حریف نابود شود. پس با این حساب باید در استفاده از ترپ کارت‌های خود زیرکانه عمل کنید!


	\item \lr{Continuous}: 
	
درست مشابه اسپل کارت‌های \lr{Continuous}، این نوع ترپ کارت‌ها نیز زمانی که در زمین بازی قرار گیرند، فعال خواهد شد و تا زمانی که رو به بالا در زمین بازی پابرجا هستند، اثر آن‌ها در جریان بازی اعمال خواهد شد.

قابلیت برخی از ترپ کارت‌های \lr{Continuous}  مشابه  \lr{Effect Monster}  هاست. همچنین این کارت ها می‌توانند حرکات حریف شما را محدود کرده و یا به مرور امتیاز حریف را کاهش دهند.

	
	\item \lr{Counter}:
	
این کارت‌ها معمولا در واکنش به فعال سازی دیگر کارت‌ها فعال می شوند و تاثیراتی روی کارت اولیه خواهند داشت.
کارت‌های ترپ \lr{Counter} آثاری بر علیه کارت‌های اسپل  \lr{Normal}  یا کارت‌های ترپ \lr{Normal}  دارند که اغلب به سختی متوقف می‌شوند.

آگاه باشید که معمولا استفاده از این نوع کارت‌های ترپ برای شما هزینه بر خواهد بود.
	
\end{itemize}


\subsubsection*{{\titr تفاوت اساسی جادو و تله}}

\addcontentsline{toc}{subsubsection}{{\fehrestContent تفاوت اساسی جادو و تله}}

کارت‌های اسپل می‌توانند مانند کارت‌های ترپ رو به پایین در زمین بازی قرار گیرند اما قوانین اجرایی برای هر کدام متفاوت است. کارت‌های اسپل فقط در فاز اصلی بازی می‌توانند فعال شوند؛ حتی امکان فعال سازی آنها در همان دور که در زمین جایگذاری شده‌اند نیز وجود دارد.(به جز کارت های اسپل \lr{Quick-Play}). در نتیجه  برخلاف کارت‌های ترپ، شما اجازه‌ی فعال‌سازی آنها را هنگام نوبت حریف ندارید.


توجه داشته باشید که در طول بازی شما می‌توانید تنها برای فریب دادن طرف مقابل، یک کارت اسپل را رو به پایین در زمین بازی قرار دهید و او را با bluff قرار دادن یک کارت ترپ در زمین بازی به فکر وا دارید!



\subsubsection*{{\titr نحوه نمایش متنی کارت‌ها}}
\addcontentsline{toc}{subsubsection}{{\fehrestContent نحوه نمایش متنی کارت‌ها}}

در قسمت‌هایی از بازی لازم است که کارت‌ها را با جزییات کامل نمایش بدهید. نمونه‌ای از آن چه که به عنوان جزییات کامل یک کارت در نظر گرفته می‌شود در زیر آمده است.

\begin{itemize}
	\item کارت هیولا:
\begin{mybox}[colback=yellow]{نمونه خروجی}
	\begin{latin}	
		Name: Card Name
		Level: 4
		Type: Warrior
		ATK: 1500
		DEF: 1500
		Description: Some Text
	\end{latin}
\end{mybox}	
	
	
		\item کارت جادو:
	\begin{mybox}[colback=yellow]{نمونه خروجی}
		\begin{latin}	
			Name: Card Name
			Spell
			Type: Normal
			Description: Some Text
		\end{latin}
	\end{mybox}	


	\item کارت تله:
\begin{mybox}[colback=yellow]{نمونه خروجی}
	\begin{latin}	
		Name: Card Name
		Trap
		Type: Normal
		Description: Some Text
	\end{latin}
\end{mybox}	
	
\end{itemize}




\subsection*{{\titr دک و دسته‌کارت‌ها}}
\addcontentsline{toc}{subsection}{{\fehrestContent دک و دسته‌کارت‌ها}}
دسته کارت یا به اختصار د.ک. (\lr{Deck}) در واقع مخزنی از کارت‌هاست که شما با 
آن به مصاف حریف خود می‌روید. هر دک به دو دسته تقسیم می‌شود:
\begin{itemize}
	\item
	دک اصلی
	\item
	دک جانبی
\end{itemize}

\subsubsection*{{\titr دک اصلی (\lr{Main Deck}):}}
\addcontentsline{toc}{subsubsection}{{\fehrestContent دک اصلی (\lr{Main 
Deck}):}}
دک اصلی شامل حداقل ۴۰ کارت و حداکثر ۶۰ کارت است. 
\\
از هر کارت تنها ۳ عدد می‌تواند در یک دک باشد و بعد از اضافه شدن آن کارت به 
دک، آن کارت از موجودی کارت‌های بازیکن حذف می‌شود، به عبارتی اگر بخواهید از یک 
کارت در چند دک استفاده کنید، باید به تعدادی که از آن کارت در دک می‌خواهید 
قرار دهید، در موجودی خود کارت داشته باشید.

\subsubsection*{{\titr دک فرعی (\lr{Side Deck}):}}
\addcontentsline{toc}{subsubsection}{{\fehrestContent دک فرعی (\lr{Side 
Deck}):}}
دک جانبی شامل حداقل صفر و حداکثر ۱۵ کارت است.
\\
کاربرد دک جانبی این است که شما می‌توانید در حین بازی، یک کارت از دک جانبی را 
با دک اصلی جا‌به‌جا کنید تا استراتژی خود را بهبود ببخشید. بنابراین تعداد 
کارت‌های دک جانبی در بازی همیشه ثابت است و تنها یک کارت میان دک جانبی و اصلی 
جا‌به‌جا می‌شود.

\subsubsection*{{\titr نکات کلی:}}
\addcontentsline{toc}{subsubsection}{{\fehrestContent نکات کلی:}}
\begin{itemize}
	\item
	هر بازیکن برای بازی‌، باید حتما یک دک معتبر داشته باشد و دکی معتبر است 
	که دک اصلی‌آن حداقل ۴۰ کارت داشته باشد.
	\item
	هر بازیکن می‌تواند به تعداد دلخواه دک داشته باشد اما تنها یک دک به عنوان 
	دک فعال (\lr{Active Deck}) انتخاب شده و با آن بازی می‌شود.
	\item
	نام دک‌ها نباید تکراری باشد.
	\item
	از آن‌جایی که تعداد مجاز یک کارت در دک ۳ است، دک جانبی نیز نباید این 
	قانون را نقض کند، به عبارتی مجموع یک کارت در دک اصلی و جانبی ۳ است و 
	اگر از 
	یک کارت ۳ عدد در دک اصلی وجود داشت، دیگری نمیتوان آن کارت را به دک 
	جانبی 
	اضافه کرد.
\end{itemize}

\subsubsection*{{\titr دستورات مرتبط با دک:}}
\addcontentsline{toc}{subsubsection}{{\fehrestContent دستورات مرتبط با دک:}}
پس از ورود به منوی \lr{Deck}، دستورات زیر معتبر هستند:
\vspace{.5cm}
\textbf{ساخت دک جدید:}
\begin{mybox}[colback=yellow]{دستور}
	\begin{latin}	
		deck create <deck name>
	\end{latin}
\end{mybox}
توجه: نام دک می‌تواند چند بخشی باشد. (به طور مثال: \lr{my first deck})
\\
پیغام موفقیت:
\begin{mybox}[colback=yellow]{پیغام به کاربر}
	\begin{latin}	
		deck created successfully!
	\end{latin}
\end{mybox}
خطا در صورتی که دک با این نام وجود داشته باشد:
\begin{mybox}[colback=yellow]{پیغام به کاربر}
	\begin{latin}	
		deck with name <deck name> already exists
	\end{latin}
\end{mybox}

\vspace{.5cm}
\textbf{حذف دک:}
\begin{mybox}[colback=yellow]{دستور}
	\begin{latin}	
		deck delete <deck name>
	\end{latin}
\end{mybox}
توجه: زمانی که یک دک حذف می‌شود، تمام کارت‌های موجود در آن به موجودی کارت‌های 
بازیکن باز می‌گردد.
\\
پیغام موفقیت:
\begin{mybox}[colback=yellow]{پیغام به کاربر}
	\begin{latin}	
		deck deleted successfully
	\end{latin}
\end{mybox}
خطا در صورتی که دک با این نام وجود نداشته باشد:
\begin{mybox}[colback=yellow]{پیغام به کاربر}
	\begin{latin}	
		deck with name <deck name> does not exist
	\end{latin}
\end{mybox}

\vspace{.5cm}
\textbf{انتخاب دک به عنوان دک فعال:}
\begin{mybox}[colback=yellow]{دستور}
	\begin{latin}	
		deck set-activate <deck name>
	\end{latin}
\end{mybox}
توجه: معتبر نبودن دک در هنگام شروع بازی بررسی خواهد شد.
\\
پیغام موفقیت:
\begin{mybox}[colback=yellow]{پیغام به کاربر}
	\begin{latin}	
		deck activated successfully
	\end{latin}
\end{mybox}
خطا در صورتی که دک با این نام وجود نداشت:
\begin{mybox}[colback=yellow]{پیغام به کاربر}
	\begin{latin}	
		deck with name <deck name> does not exist
	\end{latin}
\end{mybox}

\vspace{.5cm}
\textbf{اضافه کردن کارت به دک:}
\begin{mybox}[colback=yellow]{دستور}
	\begin{latin}	
		deck add-card -{}-card <card name> -{}-deck <deck name> 
		-{}-side(optional)
	\end{latin}
\end{mybox}
توجه کنید که تنها برای اضافه کردن کارت به \lr{side deck} از پرچم \lr{side-
{}-} 
استفاده می‌کنیم. اگر از آن استفاده نکینم، به معنی اضافه شدن به دک اصلی است.
\\
پیغام موفقیت:
\begin{mybox}[colback=yellow]{پیغام به کاربر}
	\begin{latin}	
		card added to deck successfully
	\end{latin}
\end{mybox}
خطاهای زیر در صورت وجود باید به همین ترتیب بررسی شوند:
\\
عدم وجود کارت با این نام در موجودی کارت‌های بازیکن:
\begin{mybox}[colback=yellow]{پیغام به کاربر}
	\begin{latin}	
		card with name <card name> does not exist
	\end{latin}
\end{mybox}
عدم وجود دک با این نام:
\begin{mybox}[colback=yellow]{پیغام به کاربر}
	\begin{latin}	
		deck with name <deck name> does not exist
	\end{latin}
\end{mybox}
پر بودن دک اصلی یا جانبی:
\begin{mybox}[colback=yellow]{پیغام به کاربر}
	\begin{latin}	
		<main/side> deck is full
	\end{latin}
\end{mybox}
اگر در حال حاضر از یک کارت ۳ تا در دک موجود باشد:
\begin{mybox}[colback=yellow]{پیغام به کاربر}
	\begin{latin}	
		there are already three cards with name <card name> in deck <deck name>
	\end{latin}
\end{mybox}

\vspace{.5cm}
\textbf{حذف کارت از دک:}
\begin{mybox}[colback=yellow]{دستور}
	\begin{latin}	
		deck rm-card -{}-card <card name> -{}-deck <deck name> 
		-{}-side(optional)
	\end{latin}
\end{mybox}
توجه کنید که تنها برای حذف کردن کارت از side deck از پرچم side-{}- استفاده 
می‌کنیم.
\\
توجه: زمانی که یک کارت از دک حذف می‌شود، به موجودی کارت‌های بازیکن برمی‌گردد.
\\
پیغام موفقیت:
\begin{mybox}[colback=yellow]{پیغام به کاربر}
	\begin{latin}	
		card removed form deck successfully
	\end{latin}
\end{mybox}
خطاهای زیر در صورت وجود باید به همین ترتیب بررسی شوند:
\\
عدم وجود دک با این نام:
\begin{mybox}[colback=yellow]{پیغام به کاربر}
	\begin{latin}	
		deck with name <deck name> does not exist
	\end{latin}
\end{mybox}
عدم وجود کارت با این نام در دک اصلی یا جانبی:
\begin{mybox}[colback=yellow]{پیغام به کاربر}
	\begin{latin}	
		card with name <card name> does not exist in <main/side> deck
	\end{latin}
\end{mybox}

\textbf{نمایش دک‌های بازیکن:}
\begin{mybox}[colback=yellow]{دستور}
	\begin{latin}	
		deck show -{}-all
	\end{latin}
\end{mybox}
خروجی باید با فرمت زیر باشد:
\\
هر دک با فرمت زیر نمایش داده می‌شود که ابتدا نام دک، سپس تعداد کارت‌های موجود 
در دک اصلی و سپس دک جانبی و در انتها معتبر بودن یا نبودن دک می‌آید.
\begin{mybox}[colback=yellow]{پیغام به کاربر}
	\begin{latin}	
		<deck name>: main deck <number of main deck cards>, side deck <number 
		of side deck cards>, <valid/not valid>
	\end{latin}
\end{mybox}
به طور مثال:
\begin{mybox}[colback=yellow]{خروجی نمونه}
	\begin{latin}	
		My best: main deck 50, side deck 5, valid
	\end{latin}
\end{mybox}
خروجی کلی به شکل زیر است:
\\
ابتدا در صورتی که یک دک به عنوان دک فعال انتخاب شده باشد، نمایش داده خواهد 
شد و در صورتی که دک فعالی وجود نداشته باشد چیزی زیر Active deck چاپ نمی‌شود، 
سپس بقیه دک‌ها زیر Other decks به ترتیب حروف الفبا چاپ می‌شوند:
\begin{mybox}[colback=yellow]{پیغام به کاربر}
	\begin{latin}	
		Decks: \\
		Active deck: \\
		<active deck> \\
		Other decks: \\
		<other decks>
	\end{latin}
\end{mybox}
اگر کاربر هیچ دکی نداشته باشد خروجی به شکل زیر خواهند بود:
\begin{mybox}[colback=yellow]{خروجی نمونه}
	\begin{latin}	
		Decks: \\
		Active deck: \\
		Other decks:
	\end{latin}
\end{mybox}

\vspace{.5cm}
\textbf{نمایش یک دک:}
\begin{mybox}[colback=yellow]{دستور}
	\begin{latin}	
		deck show -{}-deck-name <deck name> -{}-side(Opt)
	\end{latin}
\end{mybox}
توجه: برای نمایش دک فرعی از پرچم \lr{side-{}-} استفاده می‌کنیم.
\\
خروجی:
\begin{mybox}[colback=yellow]{پیغام به کاربر}
	\begin{latin}	
		Deck: <deck name> \\
		Side/Main deck: \\
		Monsters: \\
		<card name>: <card description> \\
		Spell and Traps: \\
		<card name>: <card description> \\
	\end{latin}
\end{mybox}
دقت شود که در هر قسمت ترتیب کارت‌ها بر اساس ترتیب الفبایی نامشان است.

\subsection*{{\titr فروشگاه}}
\addcontentsline{toc}{subsection}{{\fehrestContent فروشگاه}}
در این قسمت باید فروشگاه را پیاده‌سازی کنید.
\\
همانطور که از اسمش مشخص است باید بازیکنان در این قسمت بتوانند کارت مورد نظر 
خود را خریداری کنند.
\\
هر کارت یک قیمتی دارد که برای خریداری آن باید پرداخت شود. قیمت ها نیز بر 
اساس قدرت هر کارت تعیین شده است.
\\
قیمت هر کارت در فایل اکسل موجود است.
\\
در فروشگاه باید لیست تمام کارت ها موجود باشد و کاربر می تواند با دستور زیر، 
کارت مورد نظر را خریداری کند.
\begin{mybox}[colback=yellow]{دستور}
	\begin{latin}	
		shop Buy <card name>
	\end{latin}
\end{mybox}
اگر کاربر  نام اشتباهی را وارد کرده بود پیغام زیر را چاپ کنید:
\begin{mybox}[colback=yellow]{پیغام به کاربر}
	\begin{latin}	
		There is no card with this name
	\end{latin}
\end{mybox}
اگر کاربر سکه کافی برای خرید کارت را نداشت پیغام زیر را چاپ کنید:
\begin{mybox}[colback=yellow]{پیغام به کاربر}
	\begin{latin}	
		Not enough money
	\end{latin}
\end{mybox}
توجه داشته باشید که هیچ محدودیتی در خرید تعداد زیادی کارت یکسان ندارید.
\\
برای نمایش کارت‌های موجود در \lr{shop} دستور زیر را به کار ببرید.
\begin{mybox}[colback=yellow]{دستور}
	\begin{latin}	
		shop show -{}-all
	\end{latin}
\end{mybox}
خروجی دستور بالا به شکل زیر است:
\begin{mybox}[colback=yellow]{پیغام به کاربر}
	\begin{latin}	
		<card name>:<card description>
	\end{latin}
\end{mybox}
که ترتیب نمایش کارت‌ها بر اساس ترتیب الفبایی نام‌هایشان است.

\subsection*{{\titr گیم پلی بازی (قسمت اصلی انجام بازی)}}
\addcontentsline{toc}{subsection}{{\fehrestContent گیم پلی بازی (قسمت اصلی 
انجام بازی)}}
در ابتدا از منوی اصلی باید به منوی بازی برویم با دستور
\begin{mybox}[colback=yellow]{دستور}
	\begin{latin}	
		duel new -{}-second-player <player2 username> -{}-rounds <1/3>
	\end{latin}
\end{mybox}
    در صورتی که بازیکن \lr{player 2} وجود نداشت پیام
\begin{mybox}[colback=yellow]{پیغام به کاربر}
	\begin{latin}	
		There is no player with this username
	\end{latin}
\end{mybox}
    چاپ شود و در صورتی که هر یک از بازیکنان دک فعال نداشتند پیغام خطای زیر 
    چاپ شود:
\begin{mybox}[colback=yellow]{پیغام به کاربر}
	\begin{latin}	
		<username> has no active deck
	\end{latin}
\end{mybox}
    که در ابتدا بازیکن اصلی که درخواست بازی کرده است بررسی می‌شود و بعد 
    بازیکنی که از آن درخواست شده.
\\
    در صورتی هم که دک فعال آنها غیرمجاز بود دستور زیر چاپ شود:
\begin{mybox}[colback=yellow]{پیغام به کاربر}
	\begin{latin}	
		<username>’s deck is invalid
	\end{latin}
\end{mybox}
    بعد از این دستور در صورت عدم وجود خطا بازی ای با \lr{deck} فعال هر 
    بازیکن ساخته میشود و وارد بازی می‌شویم.
\subsubsection*{{\titr نمایش صفحه بازی}}
\addcontentsline{toc}{subsubsection}{{\fehrestContent نمایش صفحه بازی}}
بعد از انجام هر دستوری و هر حرکتی وقتی به منوی اصلی بازی رفتیم باید برد 
بازی به صورت زیر نمایش داده شود:

\vspace{.5cm}
\textbf{نکات کلی نمایش صفحه:}
\\
برد بازی باید به گونه‌ای نمایش داده شود نوبت هر بازیکنی باشد فیلد خود در سمت 
پایین نمایش داده شود. یعنی بعد از تغییر نوبت برد بازی باید قرینه نسبت به 
نقطه مرکز بازی شود یا دوران 180 درجه صفحه.
\\
هنگام نمایش صفحه در این فاز فقط لازم است تعداد کارت‌ها در بخش های مختلف زمین 
مثل \lr{graveyard} نمایش داده شود به همراه تعداد کارت‌هایی که در دست حریف 
است 
به همراه کارت‌های روی \lr{field} بازی، و برای کارت‌های \lr{field} بازی نیازی 
به نمایش خود کارت‌ها نیست و به روشی که نشان می‌دهیم فقط نوع قرارگیری کارت‌ها 
را نمایش می‌دهیم.
\\
برای انتخاب یا نمایش کارت‌های هر قسمت از صفحه بازی که بعدا با دستور های آن 
آشنا می شویم نیاز به آشنایی به مختصات و اسم هر قسمت از بخش‌های صفحه بازی 
داریم:
\begin{itemize}
	\item گورستان کارت ها(\lr{graveyard}):
	\\
	با عبارت \lr{graveyard} به \lr{graveyard} خود و \lr{graveyard -{}-opponen}t 
	به \lr{graveyard} حریف اشاره کند.
	\item قسمت کارت های هیولا(\lr{monster card zone}):
	\\
	همانطور که می‌دانید هر بخش 5 کارت دارد. شماره‌گذاری صفحه به صورت زیر است. 
	از این سیستم شماره گذاری برای دستور \lr{select} بعدا به طور کامل آن را 
	توضیح 
	می‌دهیم استفاده می‌شوند.
	\\
	برای شماره گذاری کارت‌های monster زمین خود به صورت زیر عمل می‌کنیم:
	\begin{figure}[h]
		\centering
		\includegraphics[width=.5\paperwidth]{./Resources/CardsNumbering.png}
		\caption{شماره گذاری کارت‌های \lr{monster} زمین خود}
		\label{}
	\end{figure}
	مثلا کارتی که در جایگاه 2 است را به صورت زیر صدا می‌زنیم.
	\begin{mybox}[colback=yellow]{ورودی نمونه}
		\begin{latin}	
			select -{}-monster 2
		\end{latin}
	\end{mybox}
	و چون کارت‌های زمین حریف 180 درجه دوران یافته‌اند شماره‌گذاری آنها به صورت 
	زیر است:
	\begin{figure}[h]
		\centering
		\includegraphics[width=.5\paperwidth]{./Resources/CardsNumbering_1.png}
		\caption{شماره گذاری کارت‌های \lr{monster} زمین حریف}
		\label{}
	\end{figure}
	برای مثال برای سمت راست‌ترین کارت باید
	\begin{mybox}[colback=yellow]{ورودی نمونه}
		\begin{latin}	
			select -{}-monster 5 -{}-opponent 
		\end{latin}
	\end{mybox}
	را صدا بزنیم.
	 \item قسمت کارت‌های جادو و تله (\lr{spell and trap card zone})
	 \\
	 نام‌گذاری کارت‌های این بخش مانند بخش بالا است با یک تفاوت که به جای 
	 \lr{monster-{}-} باید از \lr{spell-{}-} استفاده کرد. (مثلا:  \lr{select -
	 {}-
	 spell 1})
	 
	 \item بخش دسته کارت‌ها (\lr{deck zone})
	 \\
	 که کارت‌های \lr{deck} در آنجا قرار می‌گیرند و در نمایش فقط تعداد آنها 
	 نمایش داده می‌شوند.
	 \item بخش میدان (\lr{field zone})
	 \\
	 که حداکثر یک کارت در آن قرار میگیرد و برای صدا زدن \lr{field zone} خود 
	 یا حریف از 2 دستور \lr{select -{}-field} یا 
	 \lr{select -{}-field -{}-opponent} 
	 استفاده می‌کنیم.
\end{itemize}

\vspace{.5cm}
\textbf{چگونگی نمایش برد:}
در این قسمت چگونگی نمایش کل برد بازی به همراه کارت های درون هر بخش و یک 
مثال و را توضیح می‌دهیم.
\\
برد بازی بعد از اتمام هر عملیات به شکل زیر نمایش داده می‌شود:
\begin{mybox}[colback=yellow]{پیغام به کاربر}
	\begin{latin}	
		TODO: google doc page 17 text here
	\end{latin}
\end{mybox}
(توجه شود که تمام فواصل بالا تب [\lr{tab}] هستند)  
\\
در بخش بالا کارت‌های آبی مربوط به بازیکنی است که در آن لحظه نوبت اوست که با 
تغییر \lr{turn} بازی نیز باید 180 درجه بچرخد و عوض شود.
\\
در بخش بالا به جای \lr{GY} باید تعداد کارت های درون \lr{graveyard} را نمایش 
دهند.
\\
به جای \lr{FZ} اگر در آن قسمت کارتی بود \lr{O} به معنای \lr{Occupied} و اگر 
کارتی نبود \lr{E} به معنای \lr{Empty} را نمایش دهند.
\\
در بخش هیولا یا همان \lr{monster card zone} که در بالا یک مثال از آن می‌بینید به 
ازای هر جایگاهی که خالی است \lr{E} و هر جایگاهی که کارت در آن در حالت دفاعی 
(\lr{defensive}) قرار گرفته و برای حریف دیده می‌شود \lr{DO} به معنی
 \lr{defensive occupied} و اگر برای حریف دیده نمی‌شود یا همان \lr{set} شده است 
 \lr{DH} 
 به معنی \lr{defensive hidden} است. و کارتی که \lr{summon} شده و در حالت حمله 
 است \lr{OO} به 
 معنی \lr{offensive occupied} است.
\\
در بخش تله و جادو (\lr{spell and trap}) به ازای هر کارتی که برای حریف نمایش 
داده می‌شود \lr{O} و هر کارتی که نمایش داده نمی شود \lr{H} بزارید.
\\
در نمایش برد بازی به جای \lr{DN} باید تعداد کارت های باقی مانده در 
\lr{deck}. دوباره تکرار می‌شود که در صفحه‌ی بازی‌، صفحه‌ی دشمن 180 درجه دوران 
یافته و ترتیب کارت ها در هر \lr{Zone} برعکس شده است.

\subsubsection*{{\titr مرور قوانین بازی}}
\addcontentsline{toc}{subsubsection}{{\fehrestContent مرور قوانین بازی}}
\lr{TODO: Soroush Miazane :D}

\subsubsection*{{\titr انتخاب یک کارت}}
\addcontentsline{toc}{subsubsection}{{\fehrestContent انتخاب یک کارت}}
برای انتخاب یک کارت چه از بین کارت‌های خود و کارت های حریف دستورات زیر را 
تعریف می کنیم و بعدا کارهایی که بعد از انتخاب هر کارت می‌توان انجام داد را 
معرفی می‌کنیم.
\\
ابتدا بدانید که بعد از انجام هر دستور مثل \lr{attack} یا \lr{set} یا 
\lr{summon} همانطور که صفحه بازی نمایش داده شده و \lr{update} می شود باید 
کارت \lr{select} شده از حالت \lr{Select} خارج شود.
\\
دستور \lr{select} به صورت زیر انجام می شود:
\begin{mybox}[colback=yellow]{دستور}
	\begin{latin}	
		select <card address>
	\end{latin}
\end{mybox}
و آدرس کارت‌ها همانطور که در قسمت نکات کلی نمایش صفحه به آن اشاره کردیم را 
به طور کامل توضیح می‌دهیم.
\begin{itemize}
	\item
	آدرس کارت‌هایی که در \lr{monster zone} خود بازیکن هستند به صورت
	 \lr{select -{}-monster <number>}
	  است. که \lr{number} ها در بخش های بالاتر داک توضیح داده شده اند.
	\item
	آدرس کارت‌هایی که در \lr{spell and trap card zone} هستند به صورت
	 \lr{-{}-spell <number>}
	 است. 
	\item
	آدرس کارت هایی که در \lr{monster zone} بازیکن حریف هستند (\lr{select} 
	کردن آنها بیشتر برای دیدن اطلاعات آنهاست و کاربرد آن مانند کارت های 
	خود 
	بازیکن زیاد نیست) به صورت \lr{-{}-monster -{}-opponent <number>} است.
	\item
	آدرس کارت هایی که در \lr{spell card zone} حریف هستند به صورت
	 \lr{-{}-spell -{}-opponent <number>}
	  است.
	\item
	آدرس کارتی که در قسمت \lr{field zone} بازیکن است به صورت \lr{field-{}-} 
	است.
	\item
	آدرس کارتی که در قسمت \lr{field zone} حریف است به صورت
	 \lr{-{}-field -{}-opponent}
	  است.
	\item
	آدرس کارتی که در \lr{hand} خود بازیکن است (که طبق قانون حداکثر 6 کارت 
	نیز هستند) به صورت 
	\lr{select -{}-hand <number>}
	 است که \lr{number} در کارت‌های درون دست در هنگام نمایش برای خود از چپ به 
	 راست شماره گذاری می شوند. البته که نیاز به نمایش شماره نیست و از 
	 شماره فقط 
	 برای عملیات انتخاب کردن استفاده می‌شود.
	 \item
	 همانطور که معلوم است نیازی به انتخاب و آدرس‌دهی کارتی که در دست بازیکن 
	 حریف است نداریم؛ زیرا هیچ وقت برای بازیکن دیگر قابل نمایش نیست.	
\end{itemize}
در صورتی که آدرس داده شده \lr{invalid} بود مثلا هیچ یک از حالات بالا نبود یا 
شماره آن نامعتبر بود یا عدد بعد از \lr{my hand} بیش‌تر از کارت‌های درون دست 
فرد بود پیام زیر چاپ شود:
\begin{mybox}[colback=yellow]{پیغام به کاربر}
	\begin{latin}	
		Invalid selection
	\end{latin}
\end{mybox}
در غیر این صورت در دستور \lr{select} اگر در آدرسی که ذکر کردیم کارتی وجود 
داشت پیام
\begin{mybox}[colback=yellow]{پیغام به کاربر}
	\begin{latin}	
		card selected
	\end{latin}
\end{mybox}
چاپ شود و در صورتی که در آن خانه کارتی وجود نداشت پیام 
\begin{mybox}[colback=yellow]{پیغام به کاربر}
	\begin{latin}	
		no card found in the given position
	\end{latin}
\end{mybox}
چاپ شود.
\\
در صورت موفق بودن دستور \lr{select} آن کارت انتخاب شده می‌ماند و بعدا می 
توانیم دستورهایی را اجرا کنیم که که آن‌ها را توضیح می دهیم.
\\
برای پاک‌کردن انتخاب خود از دستور 
\begin{mybox}[colback=yellow]{دستور}
	\begin{latin}	
		select -d
	\end{latin}
\end{mybox}
استفاده می‌کنیم. بعد از اجرای این دستور اگر کارتی انتخاب شده نبود پیام
\begin{mybox}[colback=yellow]{پیغام به کاربر}
	\begin{latin}	
		no card is selected yet
	\end{latin}
\end{mybox}
و در غیر این صورت
\begin{mybox}[colback=yellow]{پیغام به کاربر}
	\begin{latin}	
		card deselected
	\end{latin}
\end{mybox}
چاپ شود.

\subsubsection*{{\titr تغییر فاز های نوبت و دستور \lr{next phase}}}
\addcontentsline{toc}{subsubsection}{{\fehrestContent تغییر فاز های نوبت و 
دستور \lr{next phase}}}
همانطور که می دانید بازی از 6 فاز
 \lr{draw phase} ، \lr{stand by phase} ، \lr{main phase 1} ، \lr{battle phase} و 
 \lr{main phase 2} و \lr{end phase }
 تشکیل شده است.
\\
    در ابتدا لازم به ذکر است که بعد از ورود به هر فاز جدید پیام زیر چاپ شود:
\begin{mybox}[colback=yellow]{پیغام به کاربر}
	\begin{latin}	
		Phase: <phase name>
	\end{latin}
\end{mybox}
    در هر فاز که بودیم با دستور \lr{next phase} به فاز بعدی بازی می رویم.

\vspace{.5cm}
\textbf{فاز کارت گرفتن (\lr{draw phase}):}
در این فاز ابتدا کارت بالای دک را به دست بازیکنی که نوبت اوست اضافه می کند و 
آن را در آخرین جایگاه کارت های آن فرد می گذارد و این پیام را چاپ می‌کند:
\begin{mybox}[colback=yellow]{پیغام به کاربر}
	\begin{latin}	
		new card added to the hand : <card name>
	\end{latin}
\end{mybox}
البته همانطور که در قوانین بازی ذکر شده اگر کارت‌های دک فرد تمام شده باشند و 
نتواند کارتی بردارد بازنده اعلام می‌شود و پیام باخت او نمایش داده می‌شود و 
بازی تمام می‌شود.

\vspace{.5cm}
\textbf{فاز آماده باش (\lr{standby phase}):}

\vspace{.5cm}
\textbf{فاز اصلی اول (\lr{Main Phase 1}):}
در این فاز. ابتدا صفحه بازی نمایش داده می‌شود که در مورد چگونگی آن توضیحات 
لازم را دادیم و بعد از هر دستور دوباره برد بازی نمایش داده می شود. 
\lr{Action}‌ های مربوط به این فاز در ادامه توضیح داده خواهد شد.

\vspace{.5cm}
\textbf{فاز پایانی (\lr{End Phase}):}
    در این فاز علاوه بر نمایش متن اسم فاز که برای همه ی فازها نام بردیم پیام 
    زیر چاپ می‌شود. 
\begin{mybox}[colback=yellow]{پیغام به کاربر}
	\begin{latin}	
	    its <next player nickname>’s turn
	\end{latin}
\end{mybox}

\subsubsection*{{\titr احضار یک کارت هیولا}}
\addcontentsline{toc}{subsubsection}{{\fehrestContent احضار یک کارت هیولا}}
    بعد از اجرای دستور
\begin{mybox}[colback=yellow]{دستور}
	\begin{latin}	
		summon 
	\end{latin}
\end{mybox}
اگر کارتی انتخاب نشده باشد و کارت انتخاب شده(\lr{selected}) نداشته باشیم 
پیام زیر چاپ شود:
\begin{mybox}[colback=yellow]{پیغام به کاربر}
	\begin{latin}	
		no card is selected yet
	\end{latin}
\end{mybox}
و در غیر این صورت اگر کارتی که انتخاب شده است در دست بازیکن نباشد یا 
\lr{monster} مد نظر قابلیت احضار عادی را نداشته باشد یا کارت انتخاب شده 
اصلا 
\lr{monster} نباشد پیام
\begin{mybox}[colback=yellow]{پیغام به کاربر}
	\begin{latin}	
		you can’t summon this card
	\end{latin}
\end{mybox}
نمایش داده می‌شود. همچنین  در صورتی که در فاز اصلی 1 یا 2 نباشیم، پیام
\begin{mybox}[colback=yellow]{پیغام به کاربر}
	\begin{latin}	
		action not allowed in this phase
	\end{latin}
\end{mybox}
نمایش داده می‌شود. همچنین اگر هر 5 خانه \lr{monster card zone} پر باشند دستور
\begin{mybox}[colback=yellow]{پیغام به کاربر}
	\begin{latin}	
		monster card zone is full
	\end{latin}
\end{mybox}
را نمایش می‌دهیم. و اما در صورتی که در این نوبت (\lr{turn}) قبلا یک کارت  
\lr{summon} یا \lr{set} کرده باشد و نتواند دیگر کارتی \lr{summon} کند دستور
\begin{mybox}[colback=yellow]{پیغام به کاربر}
	\begin{latin}	
		you already summoned/set on this turn
	\end{latin}
\end{mybox}
اگر سطح (\lr{level}) هیولایی که قصد احضار آن را داریم کمتر مساوی 4 بود آن 
کارت \lr{summon} شده به صورت کامل احضار کرده و آن کارت را به اولین خانه‌ای 
در 
\lr{monster card zone} که خالی (\lr{Empty}) است ببرید. طبعا نوع قرار گیری در 
آن خانه نیز \lr{OO} است. منظور از اولین خانه، براساس شماره خانه روی صفحه 
بازی است. و پیغام 
\begin{mybox}[colback=yellow]{پیغام به کاربر}
	\begin{latin}	
		summoned successfully
	\end{latin}
\end{mybox}
را چاپ کنید و اگر سطح آن هیولا 5 یا 6 بود اگر هیچ کارتی روی زمین برای 
\lr{tribute} کردن وجود نداشت پیام
\begin{mybox}[colback=yellow]{پیغام به کاربر}
	\begin{latin}	
		there are not enough cards for tribute
	\end{latin}
\end{mybox}
را نمایش داده و در صورت وجود از کاربر یک عدد که نشان دهنده‌ی آدرس کارتی است 
میخواهیم آن را \lr{tribute} کنیم است و در صورتی که در آن خانه هیولایی وجود 
نداشت پیام
\begin{mybox}[colback=yellow]{پیغام به کاربر}
	\begin{latin}	
		there no monsters one this address		
	\end{latin}
\end{mybox}
چاپ شود و در صورت موفقیت پیام
\begin{mybox}[colback=yellow]{پیغام به کاربر}
	\begin{latin}	
		summoned successfully 
	\end{latin}
\end{mybox}
را نمایش دهید و آن کارت \lr{tribute} شده را از زمین برداشته و سپس کارت 
انتخاب شده را بر روی اولین خانه‌ای که \lr{empty} است قرار دهید.
\\
و اگر سطح آن هیولا 7 یا 8 بود اگر حداقل 2 کارت هیولا روی زمین قرار نداشت پیام
\begin{mybox}[colback=yellow]{پیغام به کاربر}
	\begin{latin}	
		there are not enough cards for tribute
	\end{latin}
\end{mybox}
را نمایش دهید و در غیر این صورت 2 عدد از کاربر گرفته که نشان دهنده‌ی آدرس آن 
کارت‌هایی است که می‌خواهیم آنها رو \lr{tribute} کنیم و در صورتی که در حداقل 
یکی از آن 2 خانه هیچ هیولایی وجود نداشت پیام
\begin{mybox}[colback=yellow]{پیغام به کاربر}
	\begin{latin}	
		there no monsters on one of this addresses
	\end{latin}
\end{mybox}
چاپ شود و در صورت موفقیت پیام
\begin{mybox}[colback=yellow]{پیغام به کاربر}
	\begin{latin}	
		summoned successfully
	\end{latin}
\end{mybox}
را نمایش دهید و کارت های روی زمین را برداشته و آن کارت انتخاب شده را 
\lr{summon} کنید.

\subsubsection*{{\titr به کمین گذاشتن یک کارت هیولا (\lr{Set})}}
\addcontentsline{toc}{subsubsection}{{\fehrestContent به کمین گذاشتن یک کارت 
هیولا (\lr{Set})}}
    بعد از اجرای دستور
\begin{mybox}[colback=yellow]{دستور}
	\begin{latin}	
		Set
	\end{latin}
\end{mybox}
اگر کارتی انتخاب نشده باشد و کارت انتخاب شده (\lr{selected}) نداشته باشیم 
پیام زیر چاپ شود:
\begin{mybox}[colback=yellow]{پیغام به کاربر}
	\begin{latin}	
		no card is selected yet
	\end{latin}
\end{mybox}
و در غیر این صورت اگر کارتی که انتخاب شده است در دست (\lr{hand}) بازیکن 
نباشد، پیام زیر چاپ می‌شود:

\begin{mybox}[colback=yellow]{پیغام به کاربر}
	\begin{latin}	
		you can’t set this card
	\end{latin}
\end{mybox}
در غیر این صورت اگر این کارت \lr{monster} باشد (\lr{trap} و \lr{spell} را 
بعدا توضیح می دهیم) و در صورتی که در فاز اصلی 1 یا 2 نباشد پیام
\begin{mybox}[colback=yellow]{پیغام به کاربر}
	\begin{latin}	
		you can’t do this action in this phase
	\end{latin}
\end{mybox}
نمایش داده شود و اما اگر هر 5 خانه ی \lr{monster card zone} پر باشند دستور
\begin{mybox}[colback=yellow]{پیغام به کاربر}
	\begin{latin}	
		monster card zone is full
	\end{latin}
\end{mybox}
را نمایش دهید و اما در صورتی که در این نوبت (\lr{turn}) قبلا یک کارت 
\lr{summon} یا \lr{set} کرده باشد و نتواند دیگر کارتی \lr{summon} کند دستور
\begin{mybox}[colback=yellow]{پیغام به کاربر}
	\begin{latin}	
		you already summoned/set on this turn
	\end{latin}
\end{mybox}
را نمایش دهید و در آخر اگر \lr{set} کردن او به صورت کامل انجام شد آن کارت را 
به اولین (از لحاظ شماره ی آن خانه) خانه‌ای در \lr{monster card zone} که خالی 
(\lr{Empty}) است ببرید (طبعا نوع قرار گیری در آن خانه نیز \lr{DH} است) و 
دستور
\begin{mybox}[colback=yellow]{پیغام به کاربر}
	\begin{latin}	
		set successfully
	\end{latin}
\end{mybox}
را نمایش دهید.

\subsubsection*{{\titr تغییر حالت حمله ای به دفاعی یک هیولا و برعکس 
(\lr{change monster card position})}}
\addcontentsline{toc}{subsubsection}{{\fehrestContent تغییر حالت حمله ای به 
دفاعی یک هیولا و برعکس (\lr{change monster card position})}}
    با دستور
\begin{mybox}[colback=yellow]{دستور}
	\begin{latin}	
		change to attack position
	\end{latin}
\end{mybox}
حالت یک کارتی که \lr{DO} است به یک کارت \lr{OO} تبدیل می‌شود.
\\
و با دستور
\begin{mybox}[colback=yellow]{دستور}
	\begin{latin}	
	    change to defense position
	\end{latin}
\end{mybox}
    حالت یک کارتی که \lr{OO} است به \lr{DO} تبدیل می‌شود.
    اگر کارتی انتخاب نشده باشد و کارت انتخاب شده (\lr{selected}) نداشته 
    باشیم پیام زیر چاپ شود:
\begin{mybox}[colback=yellow]{پیغام به کاربر}
	\begin{latin}	
		no card is selected yet
	\end{latin}
\end{mybox}
و در غیر این صورت اگر کارتی که انتخاب شده است در دست قسمت
 \lr{monster card zone }
 نباشد پیام
\begin{mybox}[colback=yellow]{پیغام به کاربر}
\begin{latin}	
	you can’t change this card position
\end{latin}
\end{mybox}
نمایش داده شود و در صورتی که در فاز اصلی 1 یا 2 یا همان \lr{main phase} 
نباشد پیام
\begin{mybox}[colback=yellow]{پیغام به کاربر}
	\begin{latin}	
		you can’t do this action in this phase
	\end{latin}
\end{mybox}
اگر دستور \lr{change to attack} برای کارتی جز کارتی که حالت آن \lr{DO} است و 
\lr{change to defense} برای کارتی که حالت آن جز \lr{OO} است صدا زده شود باید 
پیام
\begin{mybox}[colback=yellow]{پیغام به کاربر}
	\begin{latin}	
	    this card is already in the wanted position		
	\end{latin}
\end{mybox}
نمایش داده شود و در صورتی که در این \lr{turn} یکبار \lr{position} این کارت 
تغییر کرده باشد نمی‌توان دیگر آن را تغییر داد و پیام
\begin{mybox}[colback=yellow]{پیغام به کاربر}
	\begin{latin}	
		you already changed this card position in this turn		
	\end{latin}
\end{mybox}
نمایش داده شود. در صورت موفقیت پیام زیر چاپ شود و حالت هیولا عوض شود:
\begin{mybox}[colback=yellow]{پیغام به کاربر}
	\begin{latin}	
		monster card position changed successfully
	\end{latin}
\end{mybox}

\subsubsection*{{\titr احضار چرخشی(\lr{flip summon})}}
\addcontentsline{toc}{subsubsection}{{\fehrestContent احضار چرخشی(\lr{flip 
summon})}}
این نوع احضار فقط برای چرخش کارت‌هایی که به صورت \lr{DH} بر روی
 \lr{monster card zone} قرار گرفته‌اند است و با دستور زیر انجام می‌شود:
\begin{mybox}[colback=yellow]{دستور}
	\begin{latin}	
	    flip summon	
	\end{latin}
\end{mybox}
    اگر کارتی انتخاب نشده باشد و کارت انتخاب شده (\lr{selected}) نداشته 
    باشیم پیام زیر چاپ شود
\begin{mybox}[colback=yellow]{پیغام به کاربر}
	\begin{latin}	
		no card is selected yet
	\end{latin}
\end{mybox}
و در غیر این صورت اگر کارتی که انتخاب شده است  در قسمت \lr{monster card zone} 
نباشد پیام
\begin{mybox}[colback=yellow]{پیغام به کاربر}
	\begin{latin}	
		you can’t change this card position	
	\end{latin}
\end{mybox}
نمایش داده شود و در صورتی که در فاز اصلی 1 یا 2 یا همان \lr{main phase} 
نباشد پیام
\begin{mybox}[colback=yellow]{پیغام به کاربر}
	\begin{latin}	
		you can’t do this action in this phase	
	\end{latin}
\end{mybox}
چاپ شود و در غیر این صورت اگر کارت انتخاب شده در حالت \lr{DH} نباشد یا در 
همین دور تازه روی زمین گذاشته شده باشد پیام
\begin{mybox}[colback=yellow]{پیغام به کاربر}
	\begin{latin}	
	    you can’t flip summon this card
	\end{latin}
\end{mybox}
    چاپ شود. در غیر این صورت کارت به حالت \lr{OO} در بیاید و پیام زیر چاپ 
    شود:
\begin{mybox}[colback=yellow]{پیغام به کاربر}
	\begin{latin}	
	    flip summoned successfully		
	\end{latin}
\end{mybox}

\subsubsection*{{\titr حمله کردن یک کارت هیولا به یک هیولا ی دشمن (\lr{attack 
monster})
}}
\addcontentsline{toc}{subsubsection}{{\fehrestContent حمله کردن یک کارت هیولا 
به یک هیولا ی دشمن (\lr{attack monster})
}}
    با دستور
\begin{mybox}[colback=yellow]{دستور}
	\begin{latin}	
	    attack <number>	
	\end{latin}
\end{mybox}
که در آن \lr{number} عددی بین 1 تا 5 است کارت انتخاب شده به \lr{monster} ای 
که در خانه ی \lr{number} قسمت کارت های هیولا (\lr{monster card zone}) دشمن 
است 
حمله می‌کند. (در صورت ارور ندادن)
\\
    اگر کارتی انتخاب نشده باشد و کارت انتخاب شده(selected) نداشته باشیم پیام 
    زیر چاپ شود
\begin{mybox}[colback=yellow]{پیغام به کاربر}
	\begin{latin}	
		no card is selected yet
	\end{latin}
\end{mybox}
و در غیر این صورت اگر کارتی که انتخاب شده است در دست قسمت
 \lr{monster card zone }نباشد پیام
\begin{mybox}[colback=yellow]{پیغام به کاربر}
	\begin{latin}	
		you can’t attack with this card
	\end{latin}
\end{mybox}
نمایش داده شود و در صورتی که در فاز حمله یا همان \lr{battle phase} نباشد 
پیام
\begin{mybox}[colback=yellow]{پیغام به کاربر}
	\begin{latin}	
		you can’t do this action in this phase
	\end{latin}
\end{mybox}
نمایش داده شود و در غیر این صورت در صورتی که این کارت قبلا در این نوبت حمله 
کرده باشد پیام
\begin{mybox}[colback=yellow]{پیغام به کاربر}
	\begin{latin}	
		this card already attacked
	\end{latin}
\end{mybox}
نمایش داده شود و در غیر این صورت در صورتی که در آن خانه ی دشمن کارتی وجود 
نداشته باشد دستور
\begin{mybox}[colback=yellow]{پیغام به کاربر}
	\begin{latin}	
		there is no card to attack here	
	\end{latin}
\end{mybox}
و در غیر این صورت حمله با موفقیت شکل می‌گیرد. حال بر حسب حالت‌های متفاوت 
پیام‌های درست را چاپ می‌کنیم و عمل \lr{attack} را اتجام می‌دهیم.

\begin{itemize}
	\item
	اگر به کارتی از حریف که حالت \lr{OO} (حمله‌ای) داشت حمله کرد؛ اگر قدرت 
	حمله‌ی کارت بازیکن بیشتر از  کارت \lr{monster} حریف بود \lr{LP} حریف کم 
	شده و 
	آن کارت حریف به \lr{graveyard} می‌رود. همچنین پیام زیر چاپ می‌شود:
	\begin{mybox}[colback=yellow]{پیغام به کاربر}
		\begin{latin}	
			your opponent’s monster is destroyed and your opponent receives 
			<damage> battle damage	
		\end{latin}
	\end{mybox}
	        که damage همان اختلاف قدرت حمله‌ی 2 کارت است.
	
	\item
	اگر به کارتی از حریف که حالت \lr{OO} (حمله ای) دارد که قدرت حمله ی برابر 
	با کارت انتخاب شده دارد حمله کند پیام زیر چاپ شده و هر 2 کارت نابود 
	شوند:
	\begin{mybox}[colback=yellow]{پیغام به کاربر}
		\begin{latin}	
		    both you and your opponent monster cards are destroyed and no one 
		    receives damage	
		\end{latin}
	\end{mybox}

	\item
اگر به کارتی از حریف که حالت \lr{OO} (حمله ای) داشت حمله کرد اگر قدرت حمله ی 
کارت بازیکن کمتر از  کارت \lr{monster} حریف بود \lr{LP} بازیکن کم شده و 
کارت 
بازیکن به \lr{graveyard} می رود و پیام زیر چاپ می‌شود:
	\begin{mybox}[colback=yellow]{پیغام به کاربر}
	\begin{latin}	
		    Your monster card is destroyed and you received <damage> battle 
		    damage
	\end{latin}
	\end{mybox}

	\item
اگر به کارتی از حریف که در حالت \lr{DO} بود حمله کرد؛ اگر قدرت دفاع آن کارت 
حریف کمتر از قدرت حمله ی کارت انتخاب شده‌ی بازیکن بود پیام زیر چاپ شود و آن 
کارت نابود شود:
	\begin{mybox}[colback=yellow]{پیغام به کاربر}
	\begin{latin}	
	    the defense position monster is destroyed	
	\end{latin}
	\end{mybox}

	\item
اگر به کارتی از حریف که در حالت \lr{DO} بود حمله کرد؛ اگر قدرت دفاع آن کارت 
حریف برابر قدرت حمله‌ی کارت انتخاب شده‌ی بازیکن بود، اتفاقی نیفتاده و پیام 
زیر چاپ می‌شود:
	\begin{mybox}[colback=yellow]{پیغام به کاربر}
	\begin{latin}	
	    no card is destroyed	
	\end{latin}
	\end{mybox}

	\item
اگر به کارتی از حریف که در حالت \lr{DO} بود حمله کرد؛ اگر قدرت دفاع آن کارت 
حریف بیشتر از قدرت حمله‌ی کارت انتخاب شده ی بازیکن بود پیام زیر چاپ شود و 
از 
\lr{LP} بازیکن کم شود:
	\begin{mybox}[colback=yellow]{پیغام به کاربر}
	\begin{latin}	
	    no card is destroyed and you received <damage> battle damage	
	\end{latin}
	\end{mybox}

	\item
	در صورتی که به یک کارت \lr{DH} حمله شود قوانین چاپ آن مانند حالت‌های 
	\lr{DO} است فقط با یک تفاوت که این بار نام آن کارت را در خط قبل چاپ کند 
	مثلا 
	اگر \lr{attack} کارت بازیکن برابر \lr{defense} کارت حریف بود پیام
	\begin{mybox}[colback=yellow]{پیغام به کاربر}
		\begin{latin}	
			opponent’s monster card was <monster card name> and no card is 
			destroyed
		\end{latin}
	\end{mybox}
	چاپ می‌شود.
\end{itemize}

\subsubsection*{{\titr حمله کردن مستقیم یک کارت هیولا به دشمن (\lr{direct 
attack})}}
\addcontentsline{toc}{subsubsection}{{\fehrestContent حمله کردن مستقیم یک کارت 
هیولا به دشمن \lr{(direct attack})}}
با دستور
\begin{mybox}[colback=yellow]{دستور}
	\begin{latin}	
		attack direct
	\end{latin}
\end{mybox}
    در صورت موفقیت آمیز بودن دستور، به طور مستقیم به دشمن حمله می کند و از 
    \lr{LP} او کم می‌کند.
\\
اگر کارتی انتخاب نشده باشد و کارت انتخاب شده(\lr{selected}) نداشته باشیم 
پیام زیر چاپ شود
\begin{mybox}[colback=yellow]{پیغام به کاربر}
	\begin{latin}	
		no card is selected yet
	\end{latin}
\end{mybox}
و در غیر این صورت اگر کارتی که انتخاب شده است در قسمت \lr{monster card zone }
نباشد پیام
\begin{mybox}[colback=yellow]{پیغام به کاربر}
	\begin{latin}	
		you can’t attack with this card
	\end{latin}
\end{mybox}
نمایش داده شود و در صورتی که در فاز حمله یا همان \lr{battle phase} نباشد 
پیام
\begin{mybox}[colback=yellow]{پیغام به کاربر}
	\begin{latin}	
		you can’t do this action in this phase
	\end{latin}
\end{mybox}
نمایش داده شود و در غیر این صورت در صورتی که این کارت قبلا در این نوبت حمله 
کرده باشد پیام
\begin{mybox}[colback=yellow]{پیغام به کاربر}
	\begin{latin}	
		this card already attacked	
	\end{latin}
\end{mybox}
نمایش داده شود و در غیر این صورت به طور موفقیت‌آمیز حمله رخ می‌دهد و از جان 
حریف کاسته می‌شود و پیام
\begin{mybox}[colback=yellow]{پیغام به کاربر}
	\begin{latin}	
		you opponent receives <damage> battale damage	
	\end{latin}
\end{mybox}
نمایش داده می‌شود.

\subsubsection*{{\titr فعال کردن یک کارت جادو(\lr{activate a spell card})}}
\addcontentsline{toc}{subsubsection}{{\fehrestContent فعال کردن یک کارت 
جادو(\lr{activate a spell card})}}
با دستور
\begin{mybox}[colback=yellow]{دستور}
	\begin{latin}	
	    activate effect	
	\end{latin}
\end{mybox}
اگر کارت انتخاب شده یک \lr{spell} باشد که قبلا \lr{activate} نشده باشد انجام 
می‌شود
\\
    اگر کارتی انتخاب نشده باشد و کارت انتخاب شده (\lr{selected}) نداشته 
    باشیم پیام زیر چاپ شود:

\begin{mybox}[colback=yellow]{پیغام به کاربر}
	\begin{latin}	
		no card is selected yet
	\end{latin}
\end{mybox}
و در غیر این صورت اگر کارتی که انتخاب شده است \lr{spell} نباشد پیام
\begin{mybox}[colback=yellow]{پیغام به کاربر}
	\begin{latin}	
		Activate effect is only for spell cards.	
	\end{latin}
\end{mybox}
نمایش داده شود و در صورتی که در فاز اصلی یا همان \lr{main phase} نباشد پیام
\begin{mybox}[colback=yellow]{پیغام به کاربر}
	\begin{latin}	
		you can’t activate an effect on this turn
	\end{latin}
\end{mybox}
نمایش داده شود و در صورتی که این کارت قبلا \lr{activate} شده باشد (در زمین 
باشد) دستور 
\begin{mybox}[colback=yellow]{پیغام به کاربر}
\begin{latin}	
	you have already activated this card
\end{latin}
\end{mybox}
و اگر کارت انتخاب شده در دست باشد و هر 5 خانه‌ی \lr{spell} زمین پر باشند (در 
صورتی که کارت ما برای فعال شدن نیاز به رفتن روی زمین داشته باشد و مربوط به 
\lr{Field Zone} نباشد، پیام
\begin{mybox}[colback=yellow]{پیغام به کاربر}
\begin{latin}	
	spell card zone is full
\end{latin}
\end{mybox}
نمایش داده شود و اگر آن \lr{spell} شرایطی داشت که نتوانستیم آن را فعال کنیم 
پیام
\begin{mybox}[colback=yellow]{پیغام به کاربر}
	\begin{latin}	
		preparations of this spell are not done yet
	\end{latin}
\end{mybox}
نمایش داده شود و در صورت موفقیت پیام
\begin{mybox}[colback=yellow]{پیغام به کاربر}
	\begin{latin}	
		spell activated
	\end{latin}
\end{mybox}
نمایش داده شود و به اولین خانه ای از \lr{spell card zone} که خالی است برده 
شود. در صورتی که کارت مربوط به \lr{Field Zone} باشد، به \lr{field zone} می 
رود. اگر کارتی از قبل در \lr{field zone} وجود داشته باشد، کارت قبلی به 
\lr{graveyard} برفته و کارت جدید به \lr{field zone} می‌رود. و \lr{activate} 
می‌شود. نوع قرارگیری چنین کارتی \lr{O} است. یعنی به طور کامل توسط دو طرف 
دیده 
می‌شود.
\\
(در مورد چگونگی اعمال \lr{spell} ها و پیام مربوط به کاری که انجام دادند بعدا 
گفته می‌شود)

\subsubsection*{{\titr به کمین گذاشتن یک جادو (\lr{set a spell}):}}
\addcontentsline{toc}{subsubsection}{{\fehrestContent به کمین گذاشتن یک جادو 
(\lr{set a spell})}}
با دستور 
\begin{mybox}[colback=yellow]{دستور}
	\begin{latin}	
	    set 	
	\end{latin}
\end{mybox}
میتوان علاوه بر کمین گذاشتن هیولا یک \lr{spell} هم به کمین گذاشت. در قسمت‌های 
قبلی حالت \lr{monster} بودن را توضیح داده‌ایم. در اینحا همین مسئله را برای 
\lr{Spell} کارت‌ها توضیح می‌دهیم.
\\
اگر کارتی انتخاب نشده باشد و کارت انتخاب شده (\lr{selected}) نداشته باشیم 
پیام زیر چاپ شود:
\begin{mybox}[colback=yellow]{پیغام به کاربر}
	\begin{latin}	
		no card is selected yet
	\end{latin}
\end{mybox}
و در غیر این صورت اگر کارتی که انتخاب شده است در دست (\lr{hand}) بازیکن 
نباشد:
\begin{mybox}[colback=yellow]{پیغام به کاربر}
	\begin{latin}	
		you can’t set this card	
	\end{latin}
\end{mybox}
در غیر این صورت اگر این کارت \lr{Spell} باشد، در صورتی که در فاز اصلی 1 یا 2 
نباشد پیام
\begin{mybox}[colback=yellow]{پیغام به کاربر}
	\begin{latin}	
		you can’t do this action in this phase
	\end{latin}
\end{mybox}
نمایش داده شود و اما اگر هر 5 خانه ی \lr{spell card zone} پر باشند دستور
\begin{mybox}[colback=yellow]{پیغام به کاربر}
	\begin{latin}	
		spell card zone is full
	\end{latin}
\end{mybox}
را نمایش دهید و در آخر اگر \lr{set} کردن او به صورت کامل انجام شد آن کارت را 
به اولین (از لحاظ شماره ی آن خانه) خانه ای در \lr{spell card zone} که خالی 
(\lr{Empty}) است ببرید (طبعا نوع قرار گیری در آن خانه نیز \lr{H} است) و 
دستور
\begin{mybox}[colback=yellow]{پیغام به کاربر}
	\begin{latin}	
		set successfully
	\end{latin}
\end{mybox}
را نمایش دهید.

\subsubsection*{{\titr به کمین گذاشتن یک تله}}
\addcontentsline{toc}{subsubsection}{{\fehrestContent به کمین گذاشتن یک تله}}
پس از انتخاب یک تله، برای به کمین گذاشتنش از دستور زیر استفاده می‌کنیم.
\begin{mybox}[colback=yellow]{دستور}
	\begin{latin}	
		set 
	\end{latin}
\end{mybox}
اگر کارتی انتخاب نشده باشد و کارت انتخاب شده(\lr{selected}) نداشته باشیم 
پیام زیر چاپ شود

\begin{mybox}[colback=yellow]{پیغام به کاربر}
	\begin{latin}	
		no card is selected yet
	\end{latin}
\end{mybox}
و در غیر این صورت اگر کارتی که انتخاب شده است در دست (\lr{hand}) بازیکن 
نباشد
\begin{mybox}[colback=yellow]{پیغام به کاربر}
\begin{latin}	
	you can’t set this card
\end{latin}
\end{mybox}
در غیر این صورت فرض کنید در این بخش این کارت حتما \lr{trap} است. حال در 
صورتی که در فاز اصلی 1 یا 2 نباشد پیام
\begin{mybox}[colback=yellow]{پیغام به کاربر}
\begin{latin}	
	you can’t do this action in this phase
\end{latin}
\end{mybox}
نمایش داده شود. اگر هر 5 خانه ی \lr{spell card zone} پر باشند دستور
\begin{mybox}[colback=yellow]{پیغام به کاربر}
\begin{latin}	
	spell card zone is full	
\end{latin}
\end{mybox}
را نمایش دهید و در آخر اگر \lr{set} کردن او به صورت کامل انجام شد آن کارت را 
به اولین (از لحاظ شماره ی آن خانه) خانه‌ای در \lr{spell card zone} که خالی 
(\lr{Empty}) است ببرید (طبعا نوع قرار گیری در آن خانه نیز \lr{H} است) و 
دستور
\begin{mybox}[colback=yellow]{پیغام به کاربر}
\begin{latin}	
	set successfully
\end{latin}
\end{mybox}
را نمایش دهید.

\subsubsection*{{\titr فعال کردن یک تله یا جادو در نوبت حریف}}
\addcontentsline{toc}{subsubsection}{{\fehrestContent فعال کردن یک تله یا جادو 
در نوبت حریف}}
همانطور که میدانید برای فعال کردن بعضی از \lr{spell} ها و \lr{trap} ها باید 
شرایط خاصی رخ دهد که در بخش \lr{spell} و \lr{trap} با آنها آشنا می‌شوید.
\\
در زمانی که بعضی شرایط فعال سازی بعضی از \lr{trap} ها یا \lr{spell} ها در 
نوبت حریف برقرار شد یا کارت \lr{quick spell} رد زمین حریف \lr{set} شده باشد 
باید همانجا نوبت به بازیکن حریف داده شود و پیام زیر چاپ شود و حتی برد بازی 
دوباره چاپ شود.
\begin{mybox}[colback=yellow]{پیغام به کاربر}
	\begin{latin}	
		now it will be <another player username>’s turn \\
		<show board>
	\end{latin}
\end{mybox}
و پیغام زیر چاپ شود
\begin{mybox}[colback=yellow]{پیغام به کاربر}
	\begin{latin}	
	    do you want to activate your trap and spell?
	\end{latin}
\end{mybox}
در صورتی که جواب \lr{no} باشد دوباره به نوبت خود بازیکن برگردد و پیام 

\begin{mybox}[colback=yellow]{پیغام به کاربر}
	\begin{latin}	
		now it will be <the player username>’s turn \\
		<show board>
	\end{latin}
\end{mybox}
نمایش داده شود و بازی ادامه پیدا کند و در صورتی که \lr{yes}
چاپ شد در آن نوبت فقط حرکت های مربوط به فعال کردن یک \lr{spell} یا \lr{trap} 
مجاز است (که شرایط آن برقرار شده است و به طبع آن کارت روی زمین بوده) و در 
هر حرکت و دستور دیگری انجام شد پیغام زیر چاپ شود.
\begin{mybox}[colback=yellow]{پیغام به کاربر}
	\begin{latin}	
		it’s not your turn to play this kind of moves
	\end{latin}
\end{mybox}
و در صورت دیگر که دستور \lr{activate} مربوط به آن کارت (کارت‌ها) صدا زده شد 
خروجی 
\begin{mybox}[colback=yellow]{پیغام به کاربر}
	\begin{latin}	
		spell/trap activated
	\end{latin}
\end{mybox}
چاپ شود.

\subsubsection*{{\titr احضار آیینی:/ (\lr{Ritual Summon})}}
\addcontentsline{toc}{subsubsection}{{\fehrestContent احضار آیینی:/ (\lr{Ritual 
Summon})}}
در ابتدا به این نکته اشاره کنم که هنگام فعال کردن یک \lr{ritual spell} اگر 
هیچ \lr{ritual} کارتی در دست برای وارد کردن به بازی وجود نداشت یا مجموع 
سطوح 
هیچ زیر مجموعه‌ای از کارت‌های زمین با سطح یک کارت \lr{ritual monster} ای که 
در 
دست قرار دارد برابر نمی‌شد پیغام خطای زیر به جای پیغام موفقیت در فعال شدن 
جادوی آن کارت نمایش داده شود و کارت همچنان به شکل فعال نشده باقی بماند.
\begin{mybox}[colback=yellow]{پیغام به کاربر}
	\begin{latin}	
		there is no way you could ritual summon a monster
	\end{latin}
\end{mybox}
در صورت فعال شدن یک \lr{ritual spell} یا به هر طریقی که یک \lr{ritual summon} 
صورت بگیرد، تا زمانی که کارت مربوطه به درستی روی زمین قرار داده نشده، اگر 
هر دستور دیگری وارد شود، باید پیام زیر نوشته شود:
\begin{mybox}[colback=yellow]{پیغام به کاربر}
	\begin{latin}	
	    you should ritual summon right now	
	\end{latin}
\end{mybox}
و در صورتی که کارت درست (\lr{ritual monster}) انتخاب شد و دستور \lr{summon} 
زده شد باید از کاربر آدرس کارت‌هایی که قرار است برای ورود \lr{Ritual} قربانی 
شوند، پرسیده شود. اگر سطح آن‌ کارت‌ها متناسب با کارتی که قرار است وارد زمین 
بشود نبود، باید پیام خطای زیر چاپ شود:
\begin{mybox}[colback=yellow]{پیغام به کاربر}
	\begin{latin}	
		selected monsters levels don’t match with ritual monster
	\end{latin}
\end{mybox}
و دوباره منتظر انتخاب ترتیب درستی از کارت‌ها باشیم. در مورد \lr{Cancel} کردن 
کل این بخش در قسمت مربوطه توضیح داده شده است.
\\
در صورت درست بودن \lr{ritual summon} ابتدا حالت آن کارت (\lr{attacking} یا 
\lr{defensive}) بودن آن پرسیده شده و به یکی از 2 حالت \lr{OO} یا \lr{DO} 
گذاشته می‌شود و سپس پیام:
\begin{mybox}[colback=yellow]{پیغام به کاربر}
	\begin{latin}	
	    summoned successfully	
	\end{latin}
\end{mybox}
    نمایش داده شود و کارهای مربوطه انجام شود.

\subsubsection*{{\titr احضار خاص (\lr{Special Summon})}}
\addcontentsline{toc}{subsubsection}{{\fehrestContent احضار خاص (\lr{Special 
Summon})}}
بعضی از کارت ها ی \lr{monster effect} یا کارت های دیگر ویژگی دارند که میتوان 
با فعال کردن آنها \lr{special summon} انجام داد و یک هیولای دیگر را در آن 
دست 
\lr{summon} کرد. در صورتی که کارت هیولایی در دست بازیکن وجود نداشت، امکان 
فعالسازی آن \lr{spell} وجود نخواهد داشت و پیام 
\begin{mybox}[colback=yellow]{پیغام به کاربر}
	\begin{latin}	
		there is no way you could special summon a monster
	\end{latin}
\end{mybox}
چاپ می‌شود.
\\
مانند بخش \lr{ritual summon} بعد از فعال کردن آن کارت باید حتما یک هیولا 
\lr{summon} شود و دستورات دیگر انجام نمی‌شوند و در صورت نوشتن دستورات دیگر 
پیام
\begin{mybox}[colback=yellow]{پیغام به کاربر}
	\begin{latin}	
		you should special summon right now
	\end{latin}
\end{mybox}
نوشته شود.

\subsubsection*{{\titr نشان دادن گورستان(\lr{show graveyard})}}
\addcontentsline{toc}{subsubsection}{{\fehrestContent نشان دادن 
گورستان(\lr{show graveyard})}}
با دستور 
\begin{mybox}[colback=yellow]{دستور}
	\begin{latin}	
		show graveyard
	\end{latin}
\end{mybox}
لیست تمامی کارت‌های درون \lr{graveyard} نوشته می‌شود و طریقه‌ی نمایش آن هم به 
صورت زیر است:
\begin{mybox}[colback=yellow]{پیغام به کاربر}
	\begin{latin}	
		1. card name: <cardname>\\
		2. card name: <cardname>\\
		...
	\end{latin}
\end{mybox}
ترتیب نمایش کارت‌ها از قدیمی‌ترین کارت ارسال  است. در صورت خالی بودن گورستان 
پیغام
\begin{mybox}[colback=yellow]{پیغام به کاربر}
	\begin{latin}	
		graveyard empty
	\end{latin}
\end{mybox}
نمایش داده شود. با دستور
\begin{mybox}[colback=yellow]{دستور}
	\begin{latin}	
		back
	\end{latin}
\end{mybox}
به بازی باز می‌گردیم.

\subsubsection*{{\titr نمایش یک کارت(\lr{view a card})}}
\addcontentsline{toc}{subsubsection}{{\fehrestContent نمایش یک کارت(\lr{view a 
card})}}
با دستور 
\begin{mybox}[colback=yellow]{دستور}
	\begin{latin}	
		view <selected/number>
	\end{latin}
\end{mybox}
    در صورتی که گزینه‌ی \lr{selected} انتخاب شود اگر کارتی \lr{select} نشده 
    بود پیام
\begin{mybox}[colback=yellow]{پیغام به کاربر}
	\begin{latin}	
		no card is selected yet
	\end{latin}
\end{mybox}
نمایش داده شود و در صورتی که کارتی انتخاب شده بود اطلاعات آن کارت که در بخش 
های قبل گیم‌پلی گفتیم به همان فرمت نمایش داده شود.
\\
و اگر درون دستور \lr{show graveyard} بودیم و آن لیست را نشان می‌دادیم با 
انتخاب شماره‌ی آن کارت در لیست اطلاعات آن کارت نمایش داده می‌شود. برای 
کارتهای 
\lr{graveyard} دستور \lr{select} نداریم و فقط امکان \lr{view} کردن آن‌ها وجود 
دارد. ضمنا تنها زمانی می‌توان از این دستور استفاده کرد که در وضعیت \lr{show} 
باشیم و هرجای دیگری \lr{view <number>} آمد پیام   
\begin{mybox}[colback=yellow]{پیغام به کاربر}
	\begin{latin}	
		invalid command 
	\end{latin}
\end{mybox}
نمایش داده می‌شود. همچنین اگر این دستور معتبر بود ولی شماره داده شده معتبر 
نبود پیام
\begin{mybox}[colback=yellow]{پیغام به کاربر}
	\begin{latin}	
	    invalid number	
	\end{latin}
\end{mybox}
نمایش داده می‌شود. در صورت درست بودن اطلاعات کارت نمایش داده شود.


\subsubsection*{{\titr زنجیر (\lr{Chain})}}
\addcontentsline{toc}{subsubsection}{{\fehrestContent زنجیر (\lr{Chain})}}
با زنجیر در مقررات بازی آشنا شدید در تمام دستورات بالا اگر باعث شد که نوبت 
به حریف برود و زنجیری فعال کند آن \lr{action} را انجام نمی‌دهید و پیام 
مربوطه 
را انجام نمی دهید و در صورت اتمام زنجیر به صورت برعکس \lr{action} های 
مربوطه 
را انجام می‌دهید.

\subsubsection*{{\titr نکات کلی}}
\addcontentsline{toc}{subsubsection}{{\fehrestContent نکات کلی}}
دقت کنید در هر کدام از دستوراتی که در چند مرحله از کاربر ورودی گرفته می‌شود 
با زدن دستور \lr{cancel} کل دستور متوقف می‌شود. (مثلا شماره خانه‌ها برای 
\lr{tribute summon})
\\
    هر زمان دستور 
\begin{mybox}[colback=yellow]{دستور}
	\begin{latin}	
	    surrender 	
	\end{latin}
\end{mybox}
به عنوان ورودی قرار گرفت بازیکن، از آن بازی انصراف داده و بازنده می‌شود. و 
طبق قوانین \lr{win} و \lr{lose} و پیغام‌های آن که می‌گوییم یک باخت برای 
بازیکن 
تسلیم شده ثبت می‌شود.
\\
    در صورتی که فردی برنده بازی شد:
\begin{mybox}[colback=yellow]{پیغام به کاربر}
	\begin{latin}	
		<username> won the game and the score is:  <score1>-<score2>
	\end{latin}
\end{mybox}
نمایش داده شود و در صورتی که بازیکنی کل \lr{match} (متشکل از چند بازی پشت سر 
هم) را برد علاوه بر پیام بالا پیام زیر نیز نمایش داده شود:
\begin{mybox}[colback=yellow]{پیغام به کاربر}
	\begin{latin}	
		<username> won the whole match with score: <score1>-<score2>
	\end{latin}
\end{mybox}

\subsection*{{\titr هوش مصنوعی}}
\addcontentsline{toc}{subsection}{{\fehrestContent هوش مصنوعی}}
یک ویژگی مهم بازی‌های رایانه‌ای، استفاده از هوش مصنوعی در آن‌هاست. این بازی هم 
از این قاعده مستثنی نیست و برای آن به یک هوش مصنوعی نیاز داریم؛ البته 
لزومی 
ندارد که این هوش مصنوعی پیچیده و پیشرفته باشد و کافیست یک هوش مصنوعی بسیار 
ساده و حداقلی پیاده‌سازی کنید که توانایی انجام بازی را داشته باشد و در هر 
نوبت، اگر امکانش وجود داشت کارتی را به زمین بازی اضافه کند و با کارت‌های 
زمین، حرکات مناسب را انجام دهد.
\\
یک استراتژی پیشنهادی برای پیاده‌سازی هوش مصنوعی در زیر آمده است اما 
پیاده‌سازی آن اجباری نیست و صرفا به عنوان یک پیشنهاد است. شما می‌توانید این 
استراتژی را بهبود دهید و استراتژی‌های خودتان را پیاده سازی کنید تا هوش 
مصنوعی قدرتمند‌تری داشته باشید! 
\\
توجه کنید که در زمان تحویل پروژه، به هوش‌های مصنوعی خوب نمره امتیازی تعلق 
خواهد گرفت.
\\
استراتژی پیشنهادی:
\begin{itemize}
	\item 
	هر کارت تله را در صورت امکان به زمین اضافه کند.
	
	\item
	هر کارت تله یا اسپل که قابلیت اجرایی دارد را به شرطی که به نفعش باشد و 
	یا ضرر نکند اجرا کند.
	\begin{itemize}
		\item این که چه زمانی این حرکت به نفعش است را از دیدگاه‌های گوناگون 
		می‌توان بررسی کرد. به عنوان مثال می‌توانید این که مقدار علامت‌دار اختلاف 
		امتیاز 
		کارت‌هایش و کارت‌های حریف بیشتر شود را نفع در نظر بگیرید؛ به شرط 
		این‌که با 
		فعال‌سازی آن کارت، نبازد.
		\item
		در صورت نیاز به فعال کردن تله در نوبت حریف هم به این بند توجه شود.
	\end{itemize}

	\item
	همیشه از میان هیولا‌ها، آن را که امتیاز بیشتری دارد و توان احضارش را دارد 
	احضار کند. 
	\begin{itemize}
		\item 
		اگر آن هیولا به قربانی نیاز داشت، کارت‌هایی را در زمین که کمترین 
		امتیاز را دارند قربانی کند.
	\end{itemize}
	
	\item
	برای حمله، به ترتیب از هیولایی که بیشترین مقدار حمله را دارد به کارت‌هایی 
	که رو به بالا (\lr{face up}) هستند در صورت پیروزی (کمتر بودن امتیاز 
	آن‌ها از 
	مقدار حمله آن کارت) حمله کند و سپس به همین ترتیب به کارت‌هایی که رو به 
	پایین 
	(\lr{face down}) هستند حمله کند.
\end{itemize}

\subsection*{{\titr حالت تقلب / دیباگ}}
\addcontentsline{toc}{subsection}{{\fehrestContent حالت تقلب / دیباگ}}
احتمالا در بازی های زیادی دیدید (شنیدید) که یک سری کدهای تقلب وجود دارد که 
بازی را از حالت عادی خارج می کند، نمونه ای از این کد ها مثلا در بازی 
\lr{GTA} 
وجود دارد، می‌تواند باعث شود تمامی اسلحه ها برای بازیکن در دسترس باشد، یا 
در 
مثال دیگر می تواند باعث شوند بازیکن هیچگاه جان خودش را از دست ندهد.
\\
در پروژه شما نیز این قابلیت باید پیاده‌سازی شود. در این حالت پروژه شما باید 
بتواند کارهایی که در حالت عادی نمی‌توانید انجام دهد را  پیاده‌سازی کند، 
نمونه‌ای از این کار ها را می‌توان افزایش پول به مقدار دلخواه، برداشت کارت 
بیش 
از حد مجاز، جلوگیری از بین رفتن کارت‌ها در طی بازی و … باشد.
\\
در اینجا نمونه‌ای از دستورات پیشنهادی را برای شما آورده‌ایم، شما نیز می‌توانید 
خودتان دستورات دیگری برای این حالت توسعه دهنده در زمینه‌های دیگر استفاده 
کنید. 
\\
افزایش پول به مقدار دلخواه:
\begin{mybox}[colback=yellow]{دستور}
	\begin{latin}	
		increase money  [Money] -{}-force
	\end{latin}
\end{mybox}
انتخاب کارت اضافه تر از حد مجاز:
\begin{mybox}[colback=yellow]{دستور}
	\begin{latin}	
		use card [card name] -{}-force
	\end{latin}
\end{mybox}
    افزایش جان در هنگام بازی:
\begin{mybox}[colback=yellow]{دستور}
	\begin{latin}	
		increase LP [amount] -{}-force
	\end{latin}
\end{mybox}
    بردن بازی:
\begin{mybox}[colback=yellow]{دستور}
	\begin{latin}	
		win game -{}-force
	\end{latin}
\end{mybox}
شما نیز می‌توانید به سلیقه خودتان دستورات تقلب‌های دیگری مانند، از بین بردن 
نوبت حریف، افزایش و کاهش قدرت‌های هیولاها، تغییرات در گورستان و … نیز اضافه 
کنید.

\subsection*{{\titr ایمپورت و اکسپورت دیتابیس }}

یکی از کارهای دیگری که پروژه شما باید قابلیتش را داشته باشد، توانایی JSON کردن کارت شماست، در واقع با دادن یک سری دستورات پروژه شما باید قادر باشد که کارت های شما را (با تمامی اطلاعاتشان) به صورت جیسون در یک فایل بریزد (به عبارتی دیگر serialize کند)، و بتواند آن را از فایل‌ها نیز بخواند (به عبارتی دیگر deserialize کند).

ایمپورت کردن یک کارت:

\begin{mybox}[colback=yellow]{دستور}
	\begin{latin}	
		import card [card name]
	\end{latin}
\end{mybox}

اکسپورت کردن کارت:
\begin{mybox}[colback=yellow]{دستور}
	\begin{latin}	
	export card [card name]
	
	\end{latin}
\end{mybox}

\subsection*{{\titr معرفی کارت‌ها }}
\addcontentsline{toc}{subsection}{{\fehrestContent لاگ نوشتن}}
قطعا می‌دانید که هیولاها جز مهمی از این بازی هستند، در این جا می‌خواهیم یک سری 
از هیولا ها با ویژگی‌های آن‌ها را به شما معرفی کنیم:


\tcbset{skin=bicolor,colback=LightGreen,colframe=DarkGreen,
	colbacklower=LimeGreen!75!LightGreen,
	width=1\textwidth,before=,after=\hfill,equal height group=bicolg,
	left=1mm,right=1mm,top=1mm,bottom=1mm,middle=1mm , valign lower=center , valign=center , varwidth boxed title=center , halign lower=center , halign title=center , halign upper=center}








\newcommand{\spell}[5]{
	\begin{tcolorbox}[adjusted title=\lr{#1}]
		\includegraphics[width = 40mm]{#2}
		
		جادو یا تله: #3
		
		نوع: #4
		
		\tcblower
		
		#5
		
	\end{tcolorbox}
	
}


\newcommand{\monster}[7]{
	\begin{tcolorbox}[adjusted title=\lr{#1}]
		\includegraphics[width = 40mm]{#2}
		
		نوع هیولا: \lr{#3} \quad سطح: {#4}
		
		قدرت حمله: #5 \quad قدرت دفاع: #6
		\tcblower
		
		#7
		
	\end{tcolorbox}
	
}





\subsubsection*{{\titr کارت‌های هیولا}}
\addcontentsline{toc}{subsubsection}{{\fehrestContent کارت‌های هیولا}}

\monster{Command knight}{Resources/monsters/CK.png}{Effect}{4}{1000}{1000}{
	این هیولا یک \lr{continuous effect monster} است که هنگامی که در زمین به صورت آشکار و face up قرار گرفته باشد به قدرت حمله‌ی همه ی کارت‌هایی که روی زمین قرار گرفته‌اند 400 واحد اضافه می‌کند و تا زمانی که کارت هیولای دیگری در زمین باشد نمی‌توان به این کارت حمله کرد(اگر \lr{effect} اش در حال اجرا باشد)
	
}





\monster{Axe Raider}{Resources/monsters/AxeWarrior.jpg}{Warrior}{4}{1700}{1150}{
	بدون توضیحات اضافه
}



\monster{Horn Imp}{Resources/monsters/HI.jpg}{Fiend}{4}{1300}{1000}{
	بدون توضیحات اضافه
}

\monster {Yomi Ship}{Resources/monsters/YS.png}{Aqua}{3}{800}{1400}{
	وقتی یک هیولا بهش حمله کند و نابودش کند، خود همان هیولا نیز می شود.
}


\monster{Silver Fang}{Resources/monsters/SF.jpg}{Beast}{3}{1200}{800}{
	بدون توضیحات اضافه
}

\monster{Suijin}{Resources/monsters/suijin.png}{Aqua}{7}{2500}{2400}{
	وقتی این کارت رو است، فقط برای یک بار می توانید اگر یک هیولا به این کارت حمله کرده، حمله آن هیولا را فقط برای همان دفعه صفر کنید. 
}


\monster{Fireyarou}{Resources/monsters/FireY.png}{Pyro}{4}{1300}{1000}{
	بدون توضیحات اضافه
}


\monster{Curtain of Dark Ones}{Resources/monsters/CODO.png}{Spellcaster}{2}{600}{500}{
	بدون توضیحات اضافه
}


\monster{Feral Imp}{Resources/monsters/FI.png}{Fiend}{4}{1300}{1400}{
	بدون توضیحات اضافه
}

\monster{Dark Magician}{Resources/monsters/DM.jpg}{Spellcaster}{7}{2500}{2100}{
	بدون توضیحات اضافه
}

\monster{Wattkid}{Resources/monsters/WK.jpg}{Thunder}{3}{1000}{500}{
	بدون توضیحات اضافه
}

\monster{Baby Dragon}{Resources/monsters/BabyDragon.jpg}{Dragon}{3}{1200}{700}{
	بدون توضیحات اضافه
}

\monster{Hero of the East}{Resources/monsters/hote.png}{Warrior}{3}{1100}{1000}{
	بدون توضیحات اضافه
}

\monster{Battle Warrior}{Resources/monsters/BW.jpg}{Warrior}{3}{700}{1000}{
	بدون توضیحات اضافه
}


\monster{Crawling dragon}{Resources/monsters/CrawlingDragon.png}{Dragon}{5}{1600}{1400}{
	بدون توضیحات اضافه
}

\monster{Flame Manipulator}{Resources/monsters/FM.jpg}{Spellcaster}{3}{900}{100}{
	بدون توضیحات اضافه
}

\monster{Blue-Eyes White Dragon}{Resources/monsters/BEUD.png}{Dragon}{8}{3000}{2500}{
	بدون توضیحات اضافه
}

\monster{Crab Turtle}{Resources/monsters/crab.png}{Beast-Warrior}{7}{2550}{2500}{
	قابل فراخوانی با کارت مخصوص فراخوانی آیینی (Ritual)
}

\monster{Skull Guardian}{Resources/monsters/Skull.jpg}{Warrior}{7}{2050}{2500}{
	قابل فراخوانی با کارت مخصوص فراخوانی آیینی (Ritual)
}

\monster{Slot Machine}{Resources/monsters/sm.png}{Machine}{7}{2000}{2300}{
	بدون توضیحات اضافه
}


\monster{Haniwa}{Resources/monsters/Haniwa.jpg}{Rock}{2}{500}{500}{
	بدون توضیحات اضافه
}

\monster{Man-Eater Bug}{Resources/monsters/MEB.jpg}{Insect}{2}{450}{600}{
	موقعی که برگردد می تواند یکی از هیولا های حریف را نابود کند
}


\monster{Gate Guardian}{Resources/monsters/GG.jpg}{Insect}{2}{3750}{3400}{
	موقعی که برگردد می تواند یکی از هیولا های حریف را نابود کند
}

\monster{Scanner}{Resources/monsters/Scanner.png}{Machine}{1}{?}{?}{
یک بار به ازای هر نوبت می تواند تبدیل بشود به یکی از کارت های حریف که از بازی خارج شده است. در این حالحمله و دفاع و اسم و تمامی خاصیت‌های این شخصیت تا آخر آن نوبت با کارت انتخاب شده حریف یکی می‌شود.
}

\monster{Bitron}{Resources/monsters/Bitron.png}{Cyberse}{2}{200}{2000}{
	بدون توضیحات اضافه
}


\monster{Bitron}{Resources/monsters/Marshmallon.png}{Cyberse}{2}{300}{500}{
این کارت در نبرد عادی نمی‌تواند نابود شود.
زمانی که به این کارت حمله می‌شود، پس از محاسبه ضربه‌ی وارد شده به صاحب کارت، اگر کارت به پشت (face down) بود، از جان حمله کننده ۱۰۰۰ واحد کم خواهد شد.

}



\monster{Beast King Barbaros}{Resources/monsters/BKB.png}{Beast-Warrior}{8}{3000}{1200}{
می‌توانید این کارت را بدون قربانی دو هیولای دیگر به صورت عادی احضار کنید اما قدرت حمله‌ی آن به ۱۹۰۰ کاهش خواهد یافت. همچنین می‌توانید این کارت را با قربانی سه هیولا فقط احضار کنید (نه اینکه به پشت در زمین بگذارید)، در این صورت تمام کارت‌هایی که حریف کنترل می‌کند نابود خواهد شد.
}



\monster{Texchanger}{Resources/monsters/Texchanger.png}{Cyberse}{1}{100}{100}{
	تنها یکبار در هر نوبت، اگر این کارت مورد حمله واقع شود، حمله خنثی می‌شود و صاحب کارت می‌تواند یک هیولای عادی (بدون قابلیت خاص) از نوع Cyberse را از دست، دک و یا گورستان خود به صورت ویژه احضار کند.
}


\monster{Leotron}{Resources/monsters/Leo.png}{Cyberse}{4}{2000}{0}{
	بدون توضیحات اضافه
}

\monster{The Calculator}{Resources/monsters/Calc.png}{Thunder}{2}{?}{0}{
	قدرت حمله‌ی این کارت برابر مجموع سطح هیولا‌های رو به بالای‌ صاحب کارت ضربدر ۳۰۰ است.
}


\monster{Alexandrite Dragon}{Resources/monsters/Alex.png}{Dragon}{4}{2000}{100}{
	بدون توضیحات اضافه
}


\monster{Mirage Dragon}{Resources/monsters/MD.png}{Dragon}{4}{1600}{600}{
زمانی که این کارت روی زمین باشد، حریف نمی‌تواند هیچ کارت تله‌ای را فعال کند.
}


\monster{Herald of Creation}{Resources/monsters/HOC.png}{Spellcaster}{4}{1800}{600}{
تنها یکبار در هر نوبت، می‌توانید با حذف یک کارت، یک هیولای سطح ۷ و یا بالاتر را از گورستان خود به دست خود منتقل کنید.

}

\monster{Exploder Dragon }{Resources/monsters/ED.png}{Dragon}{3}{1000}{0}{
اگر این کارت در نبرد کشته شود و به گورستان منتقل شود، کارتی که او را نابود کرده است را نابود می‌کند و در این میان، جان هیچ‌کدام از بازیکنان کم نخواهد شد و تنها کارت‌ها از بین می‌روند
}

\monster{Warrior Dai Grepher }{Resources/monsters/WDG.png}{Warrior}{4}{1700}{1600}{
	بدون توضیحات اضافه
}


\monster{Dark Blade }{Resources/monsters/DB.png}{Warrior}{4}{1800}{1500}{
	بدون توضیحات اضافه
}


\monster{Wattaildragon  }{Resources/monsters/wat.png}{Dragon}{6}{2500}{1700}{
	بدون توضیحات اضافه
}


\monster{Terratiger, the Empowered Warrior}{Resources/monsters/TEMW.png}{Warrior}{4}{1800}{1200}{
زمانی که این کارت به صورت عادی احضار شده باشد، صاحب کارت می‌تواند یک هیولای عادی سطح ۴ یا پایین‌تر را در حالت تدافعی احضار کند.	
}


\monster {The Tricky}{Resources/monsters/TT.png}{Spellcaster}{5}{2000}{1200}{
اگر این کارت در دست شما موجود باشد، می‌توانید آن را با حذف یک کارت به صورت ویژه احضار کنید.
	
}


\monster{Terratiger, the Empowered Warrior}{Resources/monsters/TEMW.png}{Warrior}{4}{1800}{1200}{
	زمانی که این کارت به صورت عادی احضار شده باشد، صاحب کارت می‌تواند یک هیولای عادی سطح ۴ یا پایین‌تر را در حالت تدافعی احضار کند.	
}

\monster{Spiral Serpent}{Resources/monsters/SS.png}{Sea Serpent}{8}{2900}{2900}{
	بدون توضیحات اضافه	
}











\subsubsection*{{\titr کارت های \lr{Spell} و \lr{Trap}}}
\addcontentsline{toc}{subsubsection}{{\fehrestContent کارت های \lr{Spell} و 
\lr{Trap}}}

\spell{Monster Reborn}{Resources/spells/MR.png}{جادو}{عادی - محدود}{
	با این کارت شما می‌توانید یک از هیولاها را از گورستان خود یا حریف انتخاب کنید و آن را به صورت احضار ویژه به بازی برگردانید.
}


\spell{Terraforming}{Resources/spells/T.png}{جادو}{عادی - محدود}{
یک عدد کارت Field Spell  از دکی که در اختیار دارید، به دست خود اضافه کنید.
}

\spell{Pot of Greed}{Resources/spells/POG.png}{جادو}{عادی - محدود}{
	 عدد کارت جدید از دک وارد دست خود بکنید.
}

\spell{Pot of Greed}{Resources/spells/POG.png}{جادو}{عادی - محدود}{
	2 عدد کارت جدید از دک وارد دست خود بکنید.
}

\spell{Raigeki}{Resources/spells/R.png}{جادو}{عادی - محدود}{
	تمام هیولاهایی که حریفتان کنترل می کند را نابود کنید.
}


\spell{Change of Heart}{Resources/spells/coh.png}{جادو}{عادی - محدود}{
یکی از هیولاهایی که حریفتان کنترل میکند را مورد هدف قرار دهید، تا انتهای این فاز از بازی آن را به کنترل خود درآورید.
}



\spell{Harpie's Feather Duster}{Resources/spells/hf.png}{جادو}{عادی - محدود}{
	تمام کارت ‌ای اسپل و ترپ که حریفتان کنترل می کند را نابود کنید.
}



\spell{Swords of Revealing Light}{Resources/spells/sor.png}{جادو}{عادی - نامحدود}{
	پس از آنکه این کارت را فعال کردید، در زمین بازی باقی خواهد ماند اما باید آن را در پایان مرحله ی سوم نوبت حریف نابود کند. زمانی که این کارت فعال می شود:
	
	اگر حریف شما هیولا رو به پایینی را کنترل می کند، باید تمام این کارت ها را به حالت رو به بالا تغییر دهید. تا زمانی که این کارت رو به بالا در زمین بازی قرار دارد، کارت هیولای حریف نمی تواند اعلام حمله کند.
	
}



\spell{Dark Hole}{Resources/spells/DH.png}{جادو}{عادی - نامحدود}{
تمام کارت‌های هیولا موجود در زمین بازی را نابود کنید.
}



\spell{Supply Squad}{Resources/spells/SS.png}{جادو}{عادی - نامحدود}{
	اگر هیولا (هیولاهایی) که شما کنترل می کنید در جنگ یا در اثر استفاده شدن یک کارت دیگر نابود شده‌اند، می توانید یکبار در هر نوبت خود یک کارت جدید از دک وارد دست خود بکنید..
}


\spell{Spell Absorption}{Resources/spells/sa.png}{جادو}{عادی - نامحدود}{
هر زمانی که یک کارت اسپل فعال سازی شده است،  به محض  resolve  شدن آن 500 LP دریافت کنید.
}



\spell{Messenger of peace}{Resources/spells/mop.png}{جادو}{Continuous  - نامحدود}{
هیولا هایی که ATK معادل 1500 یا بیشتر را دارند نمی توانند اقدام به حمله کنند. یکبار در هر نوبت از بازی، در طول فاز \lr{Standby}  به میزان $100$ واحد از LP خود را  بپردازید یا این کارت را نابود کنید.
}



\spell{Twin Twisters}{Resources/spells/tt.png}{جادو}{\lr{Quick Play} - نامحدود}{
یکی از کارت ها را دور بریزید؛ سپس حداکثر 2 کارت اسپل یا ترپ موجود در زمین بازی را مورد هدف قرار دهید و نابودشان کنید.
}



\spell{Mystical space typhoon}{Resources/spells/ms.png}{جادو}{\lr{Quick Play} - نامحدود}{
یک کارت اسپل یا ترپ موجود در زمین بازی را مورد هدف قرار داده و سپس نابودش کنید.
}

\spell{Ring of Defense}{Resources/spells/rod.png}{جادو}{\lr{Quick Play} - نامحدود}{
	هرگاه یک کارت ترپ فعال شده که اثر تخریب کننده دارد، آن اثر را 0 کنید.
}

\spell{Yami}{Resources/spells/Y.png}{جادو}{\lr{Field} - نامحدود}{
تمام  هیولاها ی موجود در زمین بازی که از نوع  Fiend  و  Spellcaster  هستند، 200 ATK/DEF  دریافت میکنند؛ همچنین تمام هیولا های  از نوع   Fairy که در زمین بازی قرار دارند، 200 ATK/Def  از دست می دهند.
}

\spell{Forest}{Resources/spells/F.png}{جادو}{\lr{Field} - نامحدود}{
تمام  هیولاهای موجود در زمین بازی که از نوع  Insect/Beast/Beast-Worrior  هستند، 200 ATK/Def  دریافت می‌کنند.
}


\spell{Closed Forest}{Resources/spells/CF.png}{جادو}{\lr{Field} - نامحدود}{
تمام  هیولاهای Beast-Type  که خودتان کنترل می کنید، به ازای هر هیولای موجود در گورستان،  100 واحد ATK  دریافت می کنند.
}

\spell{UMIIRUKA}{Resources/spells/U.png}{جادو}{\lr{Field} - نامحدود}{
	برای تمام کارت های هیولا از نوع  Aqua به میزان 500 امتیاز به ATK اضافه کرده و 400 امتیاز از DEF کم کنید.
}

\spell{Sword of Dark Destruction}{Resources/spells/sd.png}{جادو}{\lr{Equip} - نامحدود}{
	کارت های هیولای مجهز شده (equipped) به این کارت که از نوع Fiend باشند، 400 واحد ATK دریافت می‌کنند و 200 واحد از DEF آن‌ها کم می‌شود.
}

\spell{Black Pendant}{Resources/spells/bp.png}{جادو}{\lr{Equip} - نامحدود}{
کارت های هیولای مجهز شده (equipped) به این کارت، 500 واحد ATK دریافت می‌کنند.
}

\spell{United We Stand}{Resources/spells/uws.png}{جادو}{\lr{Equip} - نامحدود}{
کارت های هیولای مجهز شده (equipped) به این کارت، به ازای هر هیولایی که توسط شما کنترل می شود و در زمین بازی به صورت رو به بالا قرار گرفته است 800 واحد ATK/DEF دریافت می کنند. 
}

\spell{Magnum Shield}{Resources/spells/ms.png}{جادو}{\lr{Equip} - نامحدود}{
فقط هیولا هایی از نوع Warrior  را مجهز می کند. تاثیر این کارت با توجه به حالت قرار گرفتن کارت هیولای مورد نظر در زمین بازی است:

کارت هیولا در حالت حمله باشد:  آن کارت دقیقا معادل مقدار DEF، امتیاز ATK دریافت می کند.

کارت هیولا در حالت دفاع باشد:  آن کارت دقیقا معادل مقدار ATK، امتیاز DEF دریافت می کند.

}

\spell{Advanced Ritual Art}{Resources/spells/ARA.png}{جادو}{\lr{Ritual} - نامحدود}{
با این کارت می‌توانید یک هیولای آیینی (\lr{Ritual Monster}) را احضار آیینی (\lr{Ritual Summon}) کنید.

شما باید تعدادی کارت‌های هیولا از نوع Normal در زمیین انتخاب کنید که مجموع سطحشان برابر با سطح کارت هیولای Ritual قصد احضارش را دارید باشد. سپس این کارت‌هارا از به Graveyard بفرستید و هیولای آیینی مد نظر را احضار کنید.

	
}

\subsubsection*{{\titr کارت‌های تله}}
\addcontentsline{toc}{subsubsection}{{\fehrestContent کارت‌های تله}}


\spell{Mirror Force}{Resources/traps/mf.png}{تله}{عادی - نامحدود}{
زمانی که کارت هیولای حریف، اعلام به حمله می کند: تمام مانستر‌های Attack Position  حریف را نابود کنید.
}


\spell{Magic Cylinder}{Resources/traps/MC.png}{تله}{عادی - نامحدود}{
	زمانی که کارت هیولای حریف، اعلام به حمله می‌کند: هیولا مهاجم را هدف قرار دهید، حمله را خنثی کنید و در صورت انجام این کار، معادل ATK خود به حریف آسیب بزنید.
}


\spell{Mind Crush}{Resources/traps/mcr.png}{تله}{عادی - نامحدود}{
نام یک کارت را اعلام کنید؛ اگر آن کارت در دست حریف بود، تمام نسخه های آن را باید کنار بگذارد، در غیر این صورت باید برای یک کارت تصادفی این کنار گذاشتن را اعمال کند.
}



\spell{Trap Hole}{Resources/traps/th.png}{تله}{عادی - نامحدود}{
زمانی که حریف شما یک هیولا با حداقل ATK:1000   را به صورت Normal  یا Flip Summons به کار می‌گیرد: آن هیولا را هدف قرار داده و نابودش کنید.
}


\spell{Torrential Tribute}{Resources/traps/tt.png}{تله}{عادی - نامحدود}{
	زمانی که کارت هیولای حریف، اعلام به حمله می کند: تمام مانستر‌های Attack Position  حریف را نابود کنید.
}



\spell{Time Seal}{Resources/traps/ts.jpg}{تله}{عادی - محدود}{
	زمانی که کارت هیولای حریف، اعلام به حمله می کند: تمام مانستر‌های Attack Position  حریف را نابود کنید.
}

\spell{Negate Attack}{Resources/traps/na.png}{تله}{عادی - نامحدود}{
هنگامی که هیولای حریف حمله خود را اعلام می کند: هیولای مهاجم را هدف قرار دهید. حمله را نفی کنید ، سپس مرحله نبرد را خاتمه دهید.
}


\spell{Solemn Warning }{Resources/traps/sw.png}{تله}{عادی - نامحدود}{
هنگامی که یک هیولا احضار می شود ، یا وقتی کارت spell/trap یا اثر هیولا فعال می شود که شامل افکت ویژه احضار یک هیولا است: 2000 LP  پرداخت کنید؛ احضار یا فعال سازی را نفی کنید ، و اگر این کار را انجام دادید ، آن را نابود کنید.
}


\spell{Magic Jammer }{Resources/traps/mj.png}{تله}{عادی - نامحدود}{
هنگامی که کارت spell فعال می شود: 1 کارت را کنار بگذارید. فعال سازی را نفی کنید و در صورت انجام این کار، آن را از بین ببرید.
}



\spell{Call of the Haunted }{Resources/traps/coh.png}{تله}{عادی - نامحدود}{
یک کارت را از Graveyard خود به زمین بازی احضار کنید.
}




\end{document}







