\documentclass[]{article}
\usepackage{graphicx}
\usepackage[svgnames]{xcolor} 
\usepackage{fancyhdr}
\usepackage{fancyvrb}
\usepackage{tocloft}
\usepackage[hidelinks]{hyperref}
\usepackage{enumitem}
\usepackage[many]{tcolorbox}
\usepackage{listings }
%\usepackage[a4paper, total={6in, 8in} , top = 2cm,bottom = 4cm]{geometry}
\usepackage[a4paper, total={6in, 8in}]{geometry}
\usepackage{afterpage}
\usepackage{amssymb}
\usepackage{pdflscape}
\usepackage{textcomp}
\usepackage{xecolor}
\usepackage{rotating}
\usepackage[Kashida]{xepersian}
\usepackage[T1]{fontenc}
\usepackage{tikz}
\usepackage[utf8]{inputenc}
\usepackage{PTSerif} 
\usepackage{seqsplit}
\usepackage{changepage}


\usepackage{listings}
\usepackage{xcolor}
\usepackage{sectsty}

\setcounter{secnumdepth}{0}
 
\definecolor{codegreen}{rgb}{0,0.6,0}
\definecolor{codegray}{rgb}{0.5,0.5,0.5}
\definecolor{codepurple}{rgb}{0.58,0,0.82}
\definecolor{backcolour}{rgb}{0.95,0.95,0.92}
\definecolor{blanchedalmond}{rgb}{1.0, 0.92, 0.8}
\definecolor{brilliantlavender}{rgb}{0.96, 0.73, 1.0}
 
\NewDocumentCommand{\codeword}{v}{
\texttt{\textcolor{blue}{#1}}
}
\lstset{language=java,keywordstyle={\bfseries \color{blue}}}

\lstdefinestyle{mystyle}{
    backgroundcolor=\color{backcolour},   
    commentstyle=\color{codegreen},
    keywordstyle=\color{magenta},
    numberstyle=\tiny\color{codegray},
    stringstyle=\color{codepurple},
    basicstyle=\ttfamily\normalsize,
    breakatwhitespace=false,         
    breaklines=true,                 
    captionpos=b,                    
    keepspaces=true,                 
    numbers=left,                    
    numbersep=5pt,                  
    showspaces=false,                
    showstringspaces=false,
    showtabs=false,                  
    tabsize=2
}

\lstset{style=mystyle}

 \settextfont[BoldFont={XB Zar bold.ttf}]{XB Zar.ttf}


\setlatintextfont[Scale=1.0,
 BoldFont={LiberationSerif-Bold.ttf}, 
 ItalicFont={LiberationSerif-Italic.ttf}]{LiberationSerif-Regular.ttf}





\newcommand{\inputsample}[1]{
    ~\\
    \textbf{ورودی نمونه}
    ~\\
    \begin{tcolorbox}[breakable,boxrule=0pt]
        \begin{latin}
            \large{
                #1
            }
        \end{latin}
    \end{tcolorbox}
}

\newcommand{\outputsample}[1]{
    ~\\
    \textbf{خروجی نمونه}

    \begin{tcolorbox}[breakable,boxrule=0pt]
        \begin{latin}
            \large{
                #1
            }
        \end{latin}
    \end{tcolorbox}
}

\newtcolorbox{mybox}[2][]{colback=red!5!white,
colframe=red!75!black,fonttitle=\bfseries,
colbacktitle=red!85!black,enhanced,
attach boxed title to top center={yshift=-2mm},
title=#2,#1}

\newenvironment{changemargin}[2]{%
\begin{list}{}{%
\setlength{\topsep}{0pt}%
\setlength{\leftmargin}{#1}%
\setlength{\rightmargin}{#2}%
\setlength{\listparindent}{\parindent}%
\setlength{\itemindent}{\parindent}%
\setlength{\parsep}{\parskip}%
}%
\item[]}{\end{list}}


\definecolor{foldercolor}{RGB}{124,166,198}
\definecolor{sectionColor}{HTML}{ff5e0e}
\definecolor{subsectionColor}{HTML}{008575}

\definecolor{listColor}{HTML}{00d3b9}

\definecolor{umlrelcolor}{HTML}{3c78d8}

\definecolor{subsubsectionColor}{HTML}{3c78d8}

\defpersianfont\authorFont[Scale=0.9]{XB Zar bold.ttf}

\defpersianfont\titr[Scale=1.5]{Lalezar-Regular.ttf}

\defpersianfont\fehrest[Scale=1.2]{Lalezar-Regular.ttf}

\defpersianfont\fehrestTitle[Scale=3.0]{Lalezar-Regular.ttf}

\defpersianfont\fehrestContent[Scale=1.2]{XB Zar bold.ttf}


\sectionfont{\color{sectionColor}}  % sets colour of sections
\subsectionfont{\color{subsectionColor}}  % sets colour of sections
\subsubsectionfont{\color{subsubsectionColor}}


\renewcommand{\labelitemii}{$\circ$}


\renewcommand{\baselinestretch}{1.1}


\renewcommand{\contentsname}{فهرست}

\renewcommand{\cfttoctitlefont}{\fehrestTitle}


\renewcommand\cftsecfont{\color{sectionColor}\fehrestContent\selectfont}
\renewcommand\cftsubsecfont{\color{subsectionColor}\fehrestContent\selectfont}
\renewcommand\cftsubsubsecfont{\color{subsubsectionColor}\fehrestContent\selectfont}
%\renewcommand{\cftsecpagefont}{\color{sectionColor}}

\setlength{\parskip}{1.2pt}

\begin{document}


%%% title pages
\begin{titlepage}
\begin{center}

\textbf{ \Huge{به نام خدا} }
        
\vspace{0.2cm}

\includegraphics[width=0.4\textwidth]{sharif1.png}\\
\vspace{0.2cm}
\textbf{ \Huge{\emph درس برنامه‌سازی پیشرفته} }\\
\vspace{0.25cm}
\textbf{ \Large{ فاز اول پروژه} }
\vspace{0.2cm}
       
 
      \large \textbf{دانشکده مهندسی کامپیوتر}\\\vspace{0.1cm}
    \large   دانشگاه صنعتی شریف\\\vspace{0.2cm}
       \large   ﻧﯿﻢ سال دوم 00-99 \\\vspace{0.10cm}
      \noindent\rule[1ex]{\linewidth}{1pt}
استاد:\\
    \textbf{{دکتر محمدامین فضلی}}



    \vspace{0.20cm}

   مهلت ارسال:\\
    \textbf{{۲۳ اردیبهشت - }}
    \textbf{{ساعت 23:59:59}}

    \vspace{0.10cm}
مسئول پروژه:\\
    \textbf{\authorFont{امیرمهدی نامجو}}
    
        \vspace{0.10cm}
مسئول فاز اول:\\
    \textbf{\authorFont{عرشیا اخوان}}
    
        \vspace{0.10cm}
طراحان فاز اول:\\
    \textbf{\authorFont{سروش جهان‌زاد، متین شجاع، امیرصدرا عبدالهی، امیرمهدی کوششی، یاسمین گلزار، امیرحسین هادیان }}
    
        \vspace{0.05cm}
مسئولین تنظیم مستند:\\
    \textbf{\authorFont{سروش جهان‌زاد و پارسا محمدیان}}
    

\end{center}
\end{titlepage}
%%% title pages


%%% header of pages
\newpage
\pagestyle{fancy}
\fancyhf{}
\fancyfoot{}
\cfoot{\thepage}
\lhead{فاز اول}
\rhead{\includegraphics[width=0.1\textwidth]{sharif.png}\\
دانشکده مهندسی کامپیوتر
}
\chead{پروژه برنامه‌سازی پیشرفته}
%%% header of pages
\renewcommand{\headrulewidth}{2pt}

\KashidaOff



\tableofcontents

\newpage

 \Large \textbf{\\\\
}


\section*{{\titr نکات قابل توجه}}
\addcontentsline{toc}{section}{{\fehrestContent نکات قابل توجه}}
\begin{itemize}
\item
پس از اتمام این فاز، در گیت خود یک تگ با ورژن \lr{"v1.0.0"} بزنید. در روز تحویل حضوری این tag بررسی خواهد شد و کدهای پس از آن نمره‌ای نخواهد گرفت. برای اطلاعات بیش‌تر در مورد شیوه ورژن‌گذاری، می‌توانید به
 \href{https://semver.org/}{\textcolor{blue}{\underline{این لینک}}}
 مراجعه کنید. البته برای این پروژه صرفا رعایت کردن همان ورژن گفته شده کافیست، اما خوب‌ است که با منطق ورژن‌بندی هم آشنا بشوید.

\item
در روز تحویل حضوری مشارکت تمام اعضای تیم در پروژه بررسی خواهد‌ شد و در صورت عدم مشارکت بعضی از اعضا، نمره‌ی ایشان برای آن فاز پروژه "صفر" لحاظ می‌گردد. مشارکت، با توجه به commit های افراد تیم در مخزن گیت‌هاب پروژه بررسی می‌شود.

\item
در هر فاز می‌توانید سه روز تاخیر به ازای کسر نمره داشته‌ باشید که به ازای هر روز آن، ۱۰ درصد از نمرهٔ آن فاز را از دست خواهید‌ داد. در مجموع سه‌فاز پروژه، سه روز تاخیر نیز بخشیده خواهد‌ شد.

\item
به ازای هر ساعتی که پروژه را زودتر تحویل دهید، ۱۵ دقیقه به مهلت تاخیر بدون کسر نمره شما اضافه خواهد‌ شد. این مقدار حداکثر یک روز خواهد‌ بود که در صورت ارسال ۴ روز زودتر از ددلاین به شما تعلق خواهد گرفت. \textbf{بنابراین ددلاین‌های پروژه تحت هیچ شرایطی تمدید نخواهد‌ شد.} توصیه می‌شود با برنامه‌ریزی مناسب به ددلاین‌های درس پایبند باشید.

\item
در صورت کشف تقلب از هریک از تیم‌ها، برای بار اول منفی نمرهٔ آن فاز برای آن تیم ثبت می‌شود و برای بار دوم، نمرهٔ منفی کل پروژه برای تیم لحاظ خواهد‌ شد که معادل مردود شدن در درس است.
\end{itemize}

\newpage

\section*{{\titr مقدمه}}
\addcontentsline{toc}{section}{{\fehrestContent مقدمه}}

\subsection*{{\titr اهداف پروژه}}

\addcontentsline{toc}{subsection}{{\fehrestContent اهداف قابل توجه}}

\begin{itemize}

\item
هدف این پروژه، طراحی یک بازی کارتی مشابه عنوان \lr{Yu-Gi-Oh!} است. فاز اول به شکل عمده به منطق بازی اختصاص دارد.


\item
در این فاز از پروژه، طراحی شی‌ءگرای بازی و جداسازی صحیح منطق بخش‌های مختلف از یکدیگر مورد نظر است.

\item
یکی از اهداف پروژه، آشنایی با برخی ابزارها و Pattern های استاندارد برنامه‌نویسی، مثل ابزار مدیریت و تعریف پروژه‌ٔ 
\href{https://en.wikipedia.org/wiki/Apache_Maven}{\textcolor{blue}{\underline{\lr{Apache Maven}}}}، 
است.

\item
آشنایی با سیستم مدیریت نسخه \lr{Git} و کار تیمی بر روی پروژه بر بستر یک مخزن \lr{Github}، یکی از اهداف مهم پروژه است. در این مورد توصیه می‌شود تغییرات خود را در دوره‌های کوتاه مدت \lr{commit} کنید.

\end{itemize}

\subsection*{{\titr کلیات پروژه}}
\addcontentsline{toc}{subsection}{{\fehrestContent کلیات پروژه}}

در این فاز، صرفا منطق پروژه، بدون پیاده‌سازی گرافیک یا معماری شبکهٔ آن، باید پیاده‌سازی شود. نحوهٔ ارتباط با کاربر نیز از طریق واسط کاربری کنسول است. توجه داشته باشید که در فاز سوم پروژه، باید سیستم را طبق یک معماری سرور-کلاینت طراحی کنید. در معماری اکثر بازی‌هایی که به این شکل انجام می‌شوند، منطق بازی در سرور و مستقل از واسط کاربری سمت کلاینت و کاربر است. هر چند برای فاز اول نیاز به این مسئله ندارید ولی خوب‌ است که از الآن طراحی خوبی داشته باشید که قسمت‌های مختلف پروژه نظیر منطق اصلی انجام بازی، وابستگی اساسی به بخش‌هایی نظیر واسط کاربری نداشته باشد.

در ادامهٔ مستند، موجودیت‌ها، نمای کلی رابط کاربری سیستم، نقش‌ها و دستورات لازم شرح داده‌شده است.

\begin{enumerate}[label={نکته \arabic*:}]
\item
 تمامی اطلاعات، اعم از اطلاعات کاربران، کارت‌ها و... باید در خارج از برنامه (مثلا روی فایل) ذخیره شوند و پس از \lr{terminate} شدن برنامه و اجرای مجدد آن، بصورت خودکار اطلاعات قبلی خوانده شود و قابل دسترسی باشد. برای این کار می‌توانید از ابزارهای کار با \lr{Json} در جاوا، مثل
  \href{https://www.tutorialspoint.com/gson/gson_quick_guide.htm}{\textcolor{blue}{\lr{Gson}}}
   استفاده‌ کنید.

\item
شما باید پروژه‌ی خود را بر بستر ابزار \lr{Apache Maven} پیاده‌سازی کنید. همچنین برای اضافه کردن کتابخانه‌های مورد نیاز، \lr{dependency} های مربوطه را به فایل \lr{pom.xml} اضافه کنید.




   

\item
در هر جایی از پروژه می‌توانید هرگونه خلاقیتی را به‌کار ببرید. با این حال توجه کنید که خواسته‌های واضح پروژه بایستی انجام شوند و سیستم ورودی گرفتن و خروجی دادن شما باید مطابق جزییات گفته شده در این مستند باشد.


\item
در مستند بعضی از دستور‌هایی که مشاهده می‌کنید فرمتی به شکل زیر دارند:

\begin{mybox}[colback=yellow]{دستور}
	
	
	\begin{latin}
		
	user login --username <username> --password <password>
		
	\end{latin}
	
\end{mybox}

در چنین مواردی که شامل پارامتر‌هایی هستند که با
\lr{--}
یعنی دو کاراکتر دش، مشخص شده‌اند، باید بتوانیم از دستور با هر ترتیبی استفاده کنیم. یعنی دستور زیر

\begin{mybox}[colback=yellow]{دستور}
	
	
	\begin{latin}
		
		user login --password <password> --username <username> 
		
	\end{latin}
	
\end{mybox}

هم دستور معتبری است.

همچنین به عنوان نمره امتیازی، می‌توانید حالت مخفف شده را هم برای آنان پیاده‌سازی کنید. نحوه مخفف سازی به صورت تک حرفی با و با یک نماد -
خواهد بود و نحوه انتخاب حروف برعهده خودتان است. مثلا برای نمونه بالا چنین عبارتی می‌تواند پیشنهاد خوبی باشد:

\begin{mybox}[colback=yellow]{دستور}
	
	
	\begin{latin}
		
		user login -p <password> -u <username> 
		
	\end{latin}
	
\end{mybox}


\end{enumerate}




\newpage


\section*{{\titr معرفی بازی}}
\addcontentsline{toc}{section}{{\fehrestContent معرفی بازی}}


\textbf{این دو عبارت زیر مربوط به داک سال قبلن. صرفا برای این که قالب نوشتن این مدل دستورات که label خاص کنارشون هست یا توی حاشیه خاص هستن دستتون باشه، گذاشتم. این خط رو پاک کنید بعدا. }TODO

\begin{itemize}[label=$\blacksquare$]
	\item
	نام کاربری، نام، نام‌خانوادگی، ایمیل، شماره تلفن، رمز عبور 
\end{itemize}



\begin{mybox}[colback=yellow]{دستور}
	
	
	\begin{latin}
		
		create account [type] [username]
		
	\end{latin}
	
\end{mybox}

\section*{{\titr توضیح بخش‌های مختلف پروژه}}
\addcontentsline{toc}{section}{{\fehrestContent توضیح بخش‌های مختلف پروژه}}



\subsection*{{\titr کاربر و منوها}}
\addcontentsline{toc}{subsection}{{\fehrestContent کاربر و منوها}}



\subsection*{{\titr معرفی ساختار کارت‌ها}}
\addcontentsline{toc}{subsection}{{\fehrestContent معرفی ساختار کارت‌ها}}

\subsection*{{\titr دک و دسته‌کارت‌ها}}
\addcontentsline{toc}{subsection}{{\fehrestContent دک و دسته‌کارت‌ها}}

\subsection*{{\titr فروشگاه}}
\addcontentsline{toc}{subsection}{{\fehrestContent فروشگاه}}

\subsection*{{\titr گیم‌پلی بازی}}
\addcontentsline{toc}{subsection}{{\fehrestContent گیم‌پلی بازی}}


\subsection*{{\titr هوش مصنوعی}}
\addcontentsline{toc}{subsection}{{\fehrestContent هوش مصنوعی}}

\subsection*{{\titr حالت توسعه‌دهنده}}
\addcontentsline{toc}{subsection}{{\fehrestContent حالت توسعه‌دهنده}}


\subsection*{{\titr لاگ نوشتن}}
\addcontentsline{toc}{subsection}{{\fehrestContent لاگ نوشتن}}


\subsection*{{\titr لیست کارت‌ها}}
\addcontentsline{toc}{subsection}{{\fehrestContent لیست کارت‌ها}}




\subsubsection*{{\titr کارت‌های هیولا}}
\addcontentsline{toc}{subsubsection}{{\fehrestContent کارت‌های هیولا}}

\subsubsection*{{\titr کارت‌های جادو}}
\addcontentsline{toc}{subsubsection}{{\fehrestContent کارت‌های جادو}}

\subsubsection*{{\titr کارت‌های تله}}
\addcontentsline{toc}{subsubsection}{{\fehrestContent کارت‌های تله}}






\end{document}







