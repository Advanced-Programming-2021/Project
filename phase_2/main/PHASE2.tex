\documentclass[]{article}
\usepackage{graphicx}
\usepackage[svgnames]{xcolor} 
\usepackage{fancyhdr}
\usepackage{fancyvrb}
\usepackage{forest}
\usepackage{tocloft}
\usepackage[hidelinks]{hyperref}
\usepackage{enumitem}
\usepackage[many]{tcolorbox}
\usepackage{listings }
\usepackage[a4paper, total={6in, 8in} , top = 2cm,bottom = 4cm]{geometry}
%\usepackage[a4paper, total={6in, 8in}]{geometry}
\usepackage{afterpage}
\usepackage{amssymb}
\usepackage{pdflscape}
\usepackage{textcomp}
\usepackage{xecolor}
\usepackage{rotating}
\usepackage[Kashida]{xepersian}
\usepackage[T1]{fontenc}
\usepackage{tikz}
\usepackage[utf8]{inputenc}
\usepackage{PTSerif} 
\usepackage{seqsplit}
\usepackage{changepage}


\usepackage{listings}
\usepackage{xcolor}
\usepackage{sectsty}

\setcounter{secnumdepth}{0}
 
\definecolor{codegreen}{rgb}{0,0.6,0}
\definecolor{codegray}{rgb}{0.5,0.5,0.5}
\definecolor{codepurple}{rgb}{0.58,0,0.82}
\definecolor{backcolour}{rgb}{0.95,0.95,0.92}
\definecolor{blanchedalmond}{rgb}{1.0, 0.92, 0.8}
\definecolor{brilliantlavender}{rgb}{0.96, 0.73, 1.0}
 
\NewDocumentCommand{\codeword}{v}{
\texttt{\textcolor{blue}{#1}}
}
\lstset{language=java,keywordstyle={\bfseries \color{blue}}}

\lstdefinestyle{mystyle}{
    backgroundcolor=\color{backcolour},   
    commentstyle=\color{codegreen},
    keywordstyle=\color{magenta},
    numberstyle=\tiny\color{codegray},
    stringstyle=\color{codepurple},
    basicstyle=\ttfamily\normalsize,
    breakatwhitespace=false,         
    breaklines=true,                 
    captionpos=b,                    
    keepspaces=true,                 
    numbers=left,                    
    numbersep=5pt,                  
    showspaces=false,                
    showstringspaces=false,
    showtabs=false,                  
    tabsize=2
}

\lstset{style=mystyle}

 \settextfont[BoldFont={XB Zar bold.ttf}]{XB Zar.ttf}


\setlatintextfont[Scale=1.0,
 BoldFont={LiberationSerif-Bold.ttf}, 
 ItalicFont={LiberationSerif-Italic.ttf}]{LiberationSerif-Regular.ttf}





\newcommand{\inputsample}[1]{
    ~\\
    \textbf{ورودی نمونه}
    ~\\
    \begin{tcolorbox}[breakable,boxrule=0pt]
        \begin{latin}
            \large{
                #1
            }
        \end{latin}
    \end{tcolorbox}
}

\newcommand{\outputsample}[1]{
    ~\\
    \textbf{خروجی نمونه}

    \begin{tcolorbox}[breakable,boxrule=0pt]
        \begin{latin}
            \large{
                #1
            }
        \end{latin}
    \end{tcolorbox}
}

\newtcolorbox{mybox}[2][]{colback=red!5!white,
colframe=red!75!black,fonttitle=\bfseries,
colbacktitle=red!85!black,enhanced,
attach boxed title to top center={yshift=-2mm},
title=#2,#1}

\newenvironment{changemargin}[2]{%
\begin{list}{}{%
\setlength{\topsep}{0pt}%
\setlength{\leftmargin}{#1}%
\setlength{\rightmargin}{#2}%
\setlength{\listparindent}{\parindent}%
\setlength{\itemindent}{\parindent}%
\setlength{\parsep}{\parskip}%
}%
\item[]}{\end{list}}


\definecolor{foldercolor}{RGB}{124,166,198}
\definecolor{sectionColor}{HTML}{ff5e0e}
\definecolor{subsectionColor}{HTML}{008575}

\definecolor{listColor}{HTML}{00d3b9}

\definecolor{umlrelcolor}{HTML}{3c78d8}

\definecolor{subsubsectionColor}{HTML}{3c78d8}

\defpersianfont\authorFont[Scale=0.9]{XB Zar bold.ttf}

\defpersianfont\titr[Scale=1.5]{Lalezar-Regular.ttf}

\defpersianfont\fehrest[Scale=1.2]{Lalezar-Regular.ttf}

\defpersianfont\fehrestTitle[Scale=3.0]{Lalezar-Regular.ttf}

\defpersianfont\fehrestContent[Scale=1.2]{XB Zar bold.ttf}


\sectionfont{\color{sectionColor}}  % sets colour of sections
\subsectionfont{\color{subsectionColor}}  % sets colour of sections
\subsubsectionfont{\color{subsubsectionColor}}


\renewcommand{\labelitemii}{$\circ$}


\renewcommand{\baselinestretch}{1.1}


\renewcommand{\contentsname}{فهرست}

\renewcommand{\cfttoctitlefont}{\fehrestTitle}


\renewcommand\cftsecfont{\color{sectionColor}\fehrestContent\selectfont}
\renewcommand\cftsubsecfont{\color{subsectionColor}\fehrestContent\selectfont}
\renewcommand\cftsubsubsecfont{\color{subsubsectionColor}\fehrestContent\selectfont}
%\renewcommand{\cftsecpagefont}{\color{sectionColor}}

\setlength{\parskip}{1.2pt}

\begin{document}


%%% title pages
\begin{titlepage}
\begin{center}

\textbf{ \Huge{به نام خدا} }
        
\vspace{0.2cm}

\includegraphics[width=0.4\textwidth]{sharif1.png}\\
\vspace{0.2cm}
\textbf{ \Huge{\emph درس برنامه‌سازی پیشرفته} }\\
\vspace{0.25cm}
\textbf{ \Large{ فاز دوم پروژه} }
\vspace{0.2cm}
       
 
      \large \textbf{دانشکده مهندسی کامپیوتر}\\\vspace{0.1cm}
    \large   دانشگاه صنعتی شریف\\\vspace{0.2cm}
       \large   ﻧﯿﻢ سال دوم 00-99 \\\vspace{0.10cm}
      \noindent\rule[1ex]{\linewidth}{1pt}
استاد:\\
    \textbf{{دکتر محمدامین فضلی}}



    \vspace{0.20cm}

   مهلت ارسال:\\
    \textbf{{}}
    \textbf{{}}

    \vspace{0.10cm}
مسئول پروژه:\\
    \textbf{\authorFont{}}
    
        \vspace{0.10cm}
مسئول فاز دوم:\\
    \textbf{\authorFont{}}
    
        \vspace{0.10cm}
طراحان فاز دوم:\\
    \textbf{\authorFont{}}
    
        \vspace{0.05cm}
مسئولین تنظیم مستند:\\
    \textbf{\authorFont{}}
    

\end{center}
\end{titlepage}
%%% title pages


%%% header of pages
\newpage
\pagestyle{fancy}
\fancyhf{}
\fancyfoot{}
\cfoot{\thepage}
\lhead{فاز دوم}
\rhead{\includegraphics[width=0.1\textwidth]{sharif.png}\\
دانشکده مهندسی کامپیوتر
}
\chead{پروژه برنامه‌سازی پیشرفته}
%%% header of pages
\renewcommand{\headrulewidth}{2pt}

\KashidaOff



\tableofcontents

\newpage

 \Large \textbf{\\\\
}


\section*{{\titr نکات قابل توجه}}
\addcontentsline{toc}{section}{{\fehrestContent نکات قابل توجه}}
\begin{itemize}
\item
پس از اتمام این فاز، در گیت خود یک تگ با ورژن \lr{"v1.0.0"} بزنید. در روز تحویل حضوری این tag بررسی خواهد شد و کدهای پس از آن نمره‌ای نخواهد گرفت. برای اطلاعات بیش‌تر در مورد شیوه ورژن‌گذاری، می‌توانید به
 \href{https://semver.org/}{\textcolor{blue}{\underline{این لینک}}}
 مراجعه کنید. البته برای این پروژه صرفا رعایت کردن همان ورژن گفته شده کافیست، اما خوب‌ است که با منطق ورژن‌بندی هم آشنا بشوید.

\item
در روز تحویل حضوری مشارکت تمام اعضای تیم در پروژه بررسی خواهد‌ شد و در صورت عدم مشارکت بعضی از اعضا، نمره‌ی ایشان برای آن فاز پروژه "صفر" لحاظ می‌گردد. مشارکت، با توجه به commit های افراد تیم در مخزن گیت‌هاب پروژه بررسی می‌شود.

\item
در هر فاز می‌توانید سه روز تاخیر به ازای کسر نمره داشته‌ باشید که به ازای هر روز آن، ۱۰ درصد از نمرهٔ آن فاز را از دست خواهید‌ داد. در مجموع سه‌فاز پروژه، سه روز تاخیر نیز بخشیده خواهد‌ شد.

\item
به ازای هر ساعتی که پروژه را زودتر تحویل دهید، ۱۵ دقیقه به مهلت تاخیر بدون کسر نمره شما اضافه خواهد‌ شد. این مقدار حداکثر یک روز خواهد‌ بود که در صورت ارسال ۴ روز زودتر از ددلاین به شما تعلق خواهد گرفت. \textbf{بنابراین ددلاین‌های پروژه تحت هیچ شرایطی تمدید نخواهد‌ شد.} توصیه می‌شود با برنامه‌ریزی مناسب به ددلاین‌های درس پایبند باشید.

\item
در صورت کشف تقلب از هریک از تیم‌ها، برای بار اول منفی نمرهٔ آن فاز برای آن تیم ثبت می‌شود و برای بار دوم، نمرهٔ منفی کل پروژه برای تیم لحاظ خواهد‌ شد که معادل مردود شدن در درس است.


\item
شما باید \textbf{حداقل} برای یکی از فرایندهای موجود در پروژه، \lr{Unit Test} بنویسید. همچنین برای \lr{code coverage} بالاتر از ۷۰ درصد، به صورت خطی با درصد \lr{coverage}، نمره‌ٔ امتیازی تعلق خواهد گرفت.

\end{itemize}

\newpage

\section*{{\titr مقدمه}}
\addcontentsline{toc}{section}{{\fehrestContent مقدمه}}

\section*{{\titr نکات}}
\addcontentsline{toc}{section}{{\fehrestContent نکات}}
در طراحی گرافیک خود نکات زیر را در نظر داشته باشید:

\begin{itemize}
    \item در طراحی بازی یکی از نکاتی که باید در نظر داشت \lr{user friendly} بودن محیط بازی می‌باشد به این معنی که در استفاده از آن ابهامی وجود نداشته باشد و به سادگی بتوان فهمید که هر عنصر بیانگر چه چیزی است یکی از را‌ه‌های رسیدن به آن حذف هرگونه عنصر اضافی و همچنین استفاده از \lr{sprite} های مناسب برای هر قسمت است.
    \item استفاده از فونت و رنگ مناسب در محیط بازی اهمیت دارد به این صورت که نوشته‌ها باید کاملا واضح و خوانا باشند برای مثال استفاده از رنگ متنی که تفاوت زیادی با رنگ \lr{background} ندارد خوانایی متن را کاهش می‌دهد.
    \item در هر بخش بعضا یک چینش برای آن بخش پیشنهاد داده شده است اما هیچ اجباری برای طراحی صفحه به آن صورت نیست و می‌توانید خواسته‌های هر بخش را هر طور که مایلید پیاده‌سازی کنید اما دقت کنید که تمام قابلیت‌های خواسته شده باید در پیا‌ده‌سازی شما وجود داشته باشد.
\end{itemize}

\section*{{\titr جزئیات}}
\addcontentsline{toc}{section}{{\fehrestContent جزئیات}}

\subsection*{{\titr منوها}}
\addcontentsline{toc}{subsection}{{\fehrestContent منوها}}

\begin{itemize}
    \item تمامی منو‌های پیاده‌سازی شده باید از منوی‌ اصلی در دسترس باشند و همچنین امکان برگشت از آنها به منوی اصلی باید وجود داشته باشد.
    \item تمامی‌ منو‌ها باید دارای \lr{background} مناسب باشند.
    \item اگر در حین انجام هر عملیاتی به خطا برخوردید باید این خطا به طور مناسب نشان داده شود.خطا را با توجه به نوع آن می‌توانید در متنی زیر دکمه‌ی انجام عملیات،  زیر \lr{element} ورودی کاربر و یا در قالب  یک \lr{pop up} را نشان دهید.
    \item \textbf{امتیازی}: تغییر شکل نشان‌گر موس در بازی نمره امتیازی دارد. 
\end{itemize}

\subsection*{{\titr منو اصلی}}
\addcontentsline{toc}{subsection}{{\fehrestContent منو اصلی}}

\begin{itemize}
    \item منوی اصلی ما می‌باشد که در این منو باید امکان ورود به منو‌های دیگر وجود داشته باشد.
    \item \lr{\textbf{Logout}}: دکمه‌ای تحت این عنوان وجود داشته باشد تا با کلیک روی آن از حساب کاربری خارج شده و به منوی \lr{register/login} منتقل شویم.
\end{itemize}

\subsection*{{\titr دسته کارت}}
\addcontentsline{toc}{subsection}{{\fehrestContent دسته کارت}}
\begin{itemize}
    \item \textbf{نمایش لیست دک‌های کاربر}: در این صفحه همه دک های کاربر و تعداد کارت های داخل آنها نشان داده می شود. دک فعال باید در این صفحه مشخص باشد.
    \item \textbf{حذف یا ویرایش یک دک}: برای هر دک باید امکان حذف یا ویرایش آن وجود داشته باشد با کلیک روی دکمه‌ی حذف دک حذف می‌شود و با کلیک روی دکمه ویرایش وارد صفحه مربوط به آن دک می شویم.
    \item \textbf{گزینه ساخت دک جدید}: در این صفحه باید دکمه ای برا ایجاد دک جدید وجود داشته که در صورت انتخاب آن نام دک جدید را از کاربر بگیرد و در صورت معتبر بودن نام  یک دک ساخته شود و کاربر پیامی تحت عنوان \lr{'Deck created successfully'} مشاهده کند.
    \item \textbf{انتخاب یک دک به عنوان دک فعال}: کاربر باید بتواند یکی از دک های خود را به عنوان دک فعال انتخاب کند و یا دک فعالش را تغییر دهد می‌توانید در قالب یک دکمه در کنار حذف و ویرایش یک دک این قابلیت را داشته باشید یا در صفحه دک بتوانید آن را به عنوان یک دک فعال انتخاب کنید.
    \item \textbf{جزئیات دک}: با کلیک روی دک باید جزئیات آن نشان داده شود، برای این کار می‌توانید یک صفحه جزئیات دک داشته باشید. برای هر دک نام آن، کارت‌های آن نمایش داده می‌شود که کارت‌های اصلی و فرعی باید مجزا نشان داده شوند همچنین کاربر باید بتواند کارت یک دک را حذف کند یا به دک کارت اضافه کند.
    \item برای اضافه کردن کارت به دک پس از کلیک روی دکمه مورد نظر باید امکان نمایش همه کارت های کاربر وجود داشته باشد تا اگر خواست بتواند کارتی را در صورت امکان به دک خود اضافه کند.س
    \item \textbf{امتیازی}: در هر دو صفحه بالا اگر برای حذف هر دک یا هر کارت به جای یک دکمه، یک \lr{icon} سطل زباله داشته باشید و با \lr{drag \& drop} کردن شیء مورد  نظر، آن را حذف کنید به شما نمره امتیازی تعلق می گیرد.
    \item \textbf{نمایش کارت‌های داخل دک با نگه داشتن موس روی آن (امتیازی)}:
    اگر با نگه داشتن موس روی هر دک کارت های آن دک را نمایش دهید به شما نمره امتیازی تعلق می گیرد.    
\end{itemize}

\subsection*{{\titr فروشگاه}}
\addcontentsline{toc}{subsection}{{\fehrestContent فروشگاه}}

\begin{itemize}
    \item \textbf{نمایش لیست کارت‌ها}: باید عکس هر کارت که شامل مشخصات هر‌کدام، که بستگی به نوع کارت دارد، نشان داده شود.
    \item \textbf{خرید کارت}: در کنار هر کارت باید گزینه خرید وجود داشته باشد که در صورت کلیک روی آن قیمت کارت از حساب کابر کم شده و کارت به لیست کارت های کاربر اضافه شود.بعد از نهایی شدن خرید کارت و اضافه شدن آن به لیست کارت های کاربر باید پیامی با عنوان \lr{'Card successfully added'}  به کاربر نشان داده شود.
    \item \textbf{تعداد موجود از کارت}: در صورتی که کاربر یک یا چند بار یک کارت را خریده باشد یا به عبارت دیگر موجودی کارت ۰ نباشد، تعداد موجود از کارت برای هر کارت خریداری شده نشان داده شود.
    \item \textbf{امتیازی}: در صورتی که قیمت کارتی از موجودی کاربر بیشتر بود دکمه‌ی \lr{buy} کارت غیر‌فعال باشد.
\end{itemize}

\subsection*{{\titr جدول امتیازات}}
\addcontentsline{toc}{subsection}{{\fehrestContent جدول امتیازات}}

\begin{itemize}
    \item \lr{\textbf{Scoreboard}}:  لیست ۲۰ کاربری که امتیازشان از بقیه بیشتر است نشان داده شود که در هر سطر، شماره سطر،  نام مستعار کاربر و امتیاز او وجود دارد و اگر کاربر در آن لیست وجود دارد رنگ سطر کاربر با بقیه فرق داشته باشد.
\end{itemize}


\subsection*{{\titr پروفایل}}
\addcontentsline{toc}{subsection}{{\fehrestContent پروفایل}}
\begin{itemize}
    \item \textbf{عکس پروفایل}: برای هر کاربر یک عکس پروفایل وجود دارد که با ساخت اکانت یکی از عکس‌های \lr{default} برای پروفایل کاربر انتخاب می‌شود.
    \item \textbf{نام کاربری}: نام کاربری کاربر باید در این منو در دسترس باشد اما قابلیت تغییر آن وجود ندارد.
    \item \textbf{نام مستعار}: نام مستعار کاربر باید نشان داده شده و کاربر می‌تواند نام مستعار خود را تغییر دهد.
    \item \textbf{تغییر رمز عبور}: برای تغییر رمز عبور باید رمز قبلی و رمز جدید را به عنوان ورودی از کاربر گرفته و در صورت اعتبارسنجی عملیات انجام شود.
    \item \textbf{امتیازی}: کاربر بتواند با اپلود عکس جدید، عکس پروفایل‌ خود را عوض کند.
\end{itemize}

\subsection*{{\titr ثبت‌نام و ورود}}
\addcontentsline{toc}{subsection}{{\fehrestContent ثبت‌نام و ورود}}
\begin{itemize}
    \item \lr{\textbf{Register}}: سه ورودی نام کاربری، نام مستعار و رمز از کاربر گرفته شده و در صورت عدم وجود خطا حساب کاربری ساخته می‌شود، پس از ساخته شدن حساب کاربری باید کاربر را از این موضوع مطلع کنیم برای مثال می‌توانید در زیر دکمه‌ی ثبت‌نام با رنگ سبز عبارت \lr{'user created successfully'} را نمایش دهید.
    \item حساب کاربری باید یک عکس پروفایل داشته باشد، با ساختن حساب کاربری باید از یک سری عکس‌های از پیش تعیین شده به صورت رندوم یکی را انتخاب کرده و به عنوان عکس پروفایل حساب کاربری قرار دهیم.
    \item \lr{\textbf{Login}}: با دریافت نام کاربری و رمز از کاربر در صورت اعتبارسنجی، باید عملیات ورود انجام شده و از این منو به منوی اصلی بازی منتقل شویم.
\end{itemize}

\subsection*{{\titr منوها}}
\addcontentsline{toc}{subsection}{{\fehrestContent منوها}}
\begin{itemize}
    \item 
\end{itemize}

\subsection*{{\titr منوها}}
\addcontentsline{toc}{subsection}{{\fehrestContent منوها}}



\end{document}







