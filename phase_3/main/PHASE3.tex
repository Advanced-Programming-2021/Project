\documentclass[]{article}
\usepackage{graphicx}
\usepackage[svgnames]{xcolor} 
\usepackage{fancyhdr}
\usepackage{fancyvrb}
\usepackage{forest}
\usepackage{tocloft}
\usepackage[hidelinks]{hyperref}
\usepackage{enumitem}
\usepackage[many]{tcolorbox}
\usepackage{listings }
\usepackage[a4paper, total={6in, 8in} , top = 2cm,bottom = 4cm]{geometry}
%\usepackage[a4paper, total={6in, 8in}]{geometry}
\usepackage{afterpage}
\usepackage{amssymb}
\usepackage{pdflscape}
\usepackage{textcomp}
\usepackage{xecolor}
\usepackage{rotating}
\usepackage[Kashida]{xepersian}
\usepackage[T1]{fontenc}
\usepackage{tikz}
\usepackage[utf8]{inputenc}
\usepackage{PTSerif} 
\usepackage{seqsplit}
\usepackage{changepage}


\usepackage{listings}
\usepackage{xcolor}
\usepackage{sectsty}

\setcounter{secnumdepth}{0}
 
\definecolor{codegreen}{rgb}{0,0.6,0}
\definecolor{codegray}{rgb}{0.5,0.5,0.5}
\definecolor{codepurple}{rgb}{0.58,0,0.82}
\definecolor{backcolour}{rgb}{0.95,0.95,0.92}
\definecolor{blanchedalmond}{rgb}{1.0, 0.92, 0.8}
\definecolor{brilliantlavender}{rgb}{0.96, 0.73, 1.0}
 
\NewDocumentCommand{\codeword}{v}{
\texttt{\textcolor{blue}{#1}}
}
\lstset{language=java,keywordstyle={\bfseries \color{blue}}}

\lstdefinestyle{mystyle}{
    backgroundcolor=\color{backcolour},   
    commentstyle=\color{codegreen},
    keywordstyle=\color{magenta},
    numberstyle=\tiny\color{codegray},
    stringstyle=\color{codepurple},
    basicstyle=\ttfamily\normalsize,
    breakatwhitespace=false,         
    breaklines=true,                 
    captionpos=b,                    
    keepspaces=true,                 
    numbers=left,                    
    numbersep=5pt,                  
    showspaces=false,                
    showstringspaces=false,
    showtabs=false,                  
    tabsize=2
}

\lstset{style=mystyle}

 \settextfont[BoldFont={XB Zar bold.ttf}]{XB Zar.ttf}


\setlatintextfont[Scale=1.0,
 BoldFont={LiberationSerif-Bold.ttf}, 
 ItalicFont={LiberationSerif-Italic.ttf}]{LiberationSerif-Regular.ttf}





\newcommand{\inputsample}[1]{
    ~\\
    \textbf{ورودی نمونه}
    ~\\
    \begin{tcolorbox}[breakable,boxrule=0pt]
        \begin{latin}
            \large{
                #1
            }
        \end{latin}
    \end{tcolorbox}
}

\newcommand{\outputsample}[1]{
    ~\\
    \textbf{خروجی نمونه}

    \begin{tcolorbox}[breakable,boxrule=0pt]
        \begin{latin}
            \large{
                #1
            }
        \end{latin}
    \end{tcolorbox}
}

\newtcolorbox{mybox}[2][]{colback=red!5!white,
colframe=red!75!black,fonttitle=\bfseries,
colbacktitle=red!85!black,enhanced,
attach boxed title to top center={yshift=-2mm},
title=#2,#1}

\newenvironment{changemargin}[2]{%
\begin{list}{}{%
\setlength{\topsep}{0pt}%
\setlength{\leftmargin}{#1}%
\setlength{\rightmargin}{#2}%
\setlength{\listparindent}{\parindent}%
\setlength{\itemindent}{\parindent}%
\setlength{\parsep}{\parskip}%
}%
\item[]}{\end{list}}


\definecolor{foldercolor}{RGB}{124,166,198}
\definecolor{sectionColor}{HTML}{ff5e0e}
\definecolor{subsectionColor}{HTML}{008575}

\definecolor{listColor}{HTML}{00d3b9}

\definecolor{umlrelcolor}{HTML}{3c78d8}

\definecolor{subsubsectionColor}{HTML}{3c78d8}

\defpersianfont\authorFont[Scale=0.9]{XB Zar bold.ttf}

\defpersianfont\titr[Scale=1.5]{Lalezar-Regular.ttf}

\defpersianfont\fehrest[Scale=1.2]{Lalezar-Regular.ttf}

\defpersianfont\fehrestTitle[Scale=3.0]{Lalezar-Regular.ttf}

\defpersianfont\fehrestContent[Scale=1.2]{XB Zar bold.ttf}


\sectionfont{\color{sectionColor}}  % sets colour of sections
\subsectionfont{\color{subsectionColor}}  % sets colour of sections
\subsubsectionfont{\color{subsubsectionColor}}


\renewcommand{\labelitemii}{$\circ$}


\renewcommand{\baselinestretch}{1.1}


\renewcommand{\contentsname}{فهرست}

\renewcommand{\cfttoctitlefont}{\fehrestTitle}


\renewcommand\cftsecfont{\color{sectionColor}\fehrestContent\selectfont}
\renewcommand\cftsubsecfont{\color{subsectionColor}\fehrestContent\selectfont}
\renewcommand\cftsubsubsecfont{\color{subsubsectionColor}\fehrestContent\selectfont}
%\renewcommand{\cftsecpagefont}{\color{sectionColor}}

\setlength{\parskip}{1.2pt}

\begin{document}


%%% title pages
\begin{titlepage}
\begin{center}

\textbf{ \Huge{به نام خدا} }
        
\vspace{0.2cm}

\includegraphics[width=0.4\textwidth]{sharif1.png}\\
\vspace{0.2cm}
\textbf{ \Huge{\emph درس برنامه‌سازی پیشرفته} }\\
\vspace{0.25cm}
\textbf{ \Large{ فاز سوم پروژه} }
\vspace{0.2cm}
       
 
      \large \textbf{دانشکده مهندسی کامپیوتر}\\\vspace{0.1cm}
    \large   دانشگاه صنعتی شریف\\\vspace{0.2cm}
       \large   ﻧﯿﻢ سال دوم 00-99 \\\vspace{0.10cm}
      \noindent\rule[1ex]{\linewidth}{1pt}
استاد:\\
    \textbf{{دکتر محمدامین فضلی}}



    \vspace{0.20cm}

   مهلت ارسال:\\
    \textbf{}
    \textbf{}

    \vspace{0.10cm}
مسئول پروژه:\\
    \textbf{\authorFont{امیرمهدی نامجو}}
    
        \vspace{0.10cm}
مسئول فاز دوم:\\
    \textbf{\authorFont{}}
    
        \vspace{0.10cm}
طراحان فاز دوم:\\
    \textbf{\authorFont{}}
    
        \vspace{0.05cm}
مسئولین تنظیم مستند:\\
    \textbf{\authorFont{پارسا محمدیان و سروش جهان‌زاد}}
    

\end{center}
\end{titlepage}
%%% title pages


%%% header of pages
\newpage
\pagestyle{fancy}
\fancyhf{}
\fancyfoot{}
\cfoot{\thepage}
\lhead{فاز سوم}
\rhead{\includegraphics[width=0.1\textwidth]{sharif.png}\\
دانشکده مهندسی کامپیوتر
}
\chead{پروژه برنامه‌سازی پیشرفته}
%%% header of pages
\renewcommand{\headrulewidth}{2pt}

\KashidaOff



\tableofcontents

\newpage

 \Large \textbf{\\\\
}


\section*{{\titr نکات قابل توجه}}
\addcontentsline{toc}{section}{{\fehrestContent نکات قابل توجه}}
\begin{itemize}
\item
پس از اتمام این فاز، در گیت خود یک تگ با ورژن \lr{"v3.0.0"} بزنید. در روز تحویل حضوری این tag بررسی خواهد شد و کدهای پس از آن نمره‌ای نخواهد گرفت. برای اطلاعات بیش‌تر در مورد شیوه ورژن‌گذاری، می‌توانید به
 \href{https://semver.org/}{\textcolor{blue}{\underline{این لینک}}}
 مراجعه کنید. البته برای این پروژه صرفا رعایت کردن همان ورژن گفته شده کافیست، اما خوب‌ است که با منطق ورژن‌بندی هم آشنا بشوید.

\item
در روز تحویل حضوری مشارکت تمام اعضای تیم در پروژه بررسی خواهد‌ شد و در صورت عدم مشارکت بعضی از اعضا، نمره‌ی ایشان برای آن فاز پروژه "صفر" لحاظ می‌گردد. مشارکت، با توجه به commit های افراد تیم در مخزن گیت‌هاب پروژه بررسی می‌شود.

\item
در هر فاز می‌توانید سه روز تاخیر به ازای کسر نمره داشته‌ باشید که به ازای هر روز آن، ۱۰ درصد از نمرهٔ آن فاز را از دست خواهید‌ داد. در مجموع سه‌فاز پروژه، سه روز تاخیر نیز بخشیده خواهد‌ شد.

\item
به ازای هر ساعتی که پروژه را زودتر تحویل دهید، ۱۵ دقیقه به مهلت تاخیر بدون کسر نمره شما اضافه خواهد‌ شد. این مقدار حداکثر یک روز خواهد‌ بود که در صورت ارسال ۴ روز زودتر از ددلاین به شما تعلق خواهد گرفت. \textbf{بنابراین ددلاین‌های پروژه تحت هیچ شرایطی تمدید نخواهد‌ شد.} توصیه می‌شود با برنامه‌ریزی مناسب به ددلاین‌های درس پایبند باشید.

\item
در صورت کشف تقلب از هریک از تیم‌ها، برای بار اول منفی نمرهٔ آن فاز برای آن تیم ثبت می‌شود و برای بار دوم، نمرهٔ منفی کل پروژه برای تیم لحاظ خواهد‌ شد که معادل مردود شدن در درس است.


\end{itemize}

\newpage

\section*{{\titr مقدمه}}
\addcontentsline{toc}{section}{{\fehrestContent مقدمه}}

همانطور که میدونید، در دو فاز قبلی بخش اصلی مورد نیاز برای اجرای بازی توسط یک کامپیوتر مستقل را پیاده‌سازی کردیم. یعنی بخش‌های مربوط به منطق و گرافیک بازی که برای کامل بودن بازی کافیه. پس در این بخش قراره چه کنیم؟ قراره قابلیت‌هایی رو به بازیمون اضافه کنیم که بشه بر بستر اینترنت هم بازی را اجرا و با بقیه بازیکنان از راه دور بازی کرد. همچنین این فاز بخش‌های امتیازی متنوع و زیادی داره که برای جبران نمرات از دست‌ رفته‌ی دو فاز قبلی می‌تونید ازشون استفاده کنید. در ادامه بخش‌های اجباری و امتیازی را به تفکیک می‌تونید ببینید.

\section*{{\titr بخش‌های اجباری}}
\addcontentsline{toc}{section}{{\fehrestContent بخش‌های اجباری}}

\subsection*{{\titr احراز هویت}}
\addcontentsline{toc}{subsection}{{\fehrestContent احراز هویت}}

فرض کنید می‌خواهید وارد گروهی شوید که برای افرادی که در آن ثبت‌نام می‌کنند قابلیت‌های ویژه‌ای در نظر گرفته شده است. شما ثبت‌نام کرده و مشخصات شما توسط مسئولین گروه ذخیره می‌شود، سپس به شما کارت عضویت داده می‌شود شما با این کارت احراز هویت شده و از شرایط خاص گروه بهره‌مند می‌شوید.

احراز هویت در برنامه‌نویسی نیز خاصیت مشابهی دارد شما پس از ثبت‌نام در سایتی دارای حساب کاربری شده و می‌توانید از ویژگی‌های خاص سایت مورد نظر استفاده کنید.

سوالی که مطرح می‌شود این است که احراز هویت به چه شکل صورت می‌گیرد؟

در ابتدا کاربر باید ثبت‌نام کند. به صورت عادی از کاربران نام‌کاربری و رمز عبور خواسته می‌شود (اطلاعات فرد)، کاربر پس از تایید ثبت‌نام دارای حساب کاربری شده و می‌تواند با نام‌کاربری و رمز عبور وارد حساب خود شود (کارت عضویت).

در برنامه‌‌ها سرور برای پردازش هر ریکوئست از سمت کلاینت نیاز به شناخت کاربر درخواست‌دهنده دارد.

عموما به دو شکل سرور متوجه کاربر حال حاضر می‌شود.

\begin{itemize}
    \item بر پایه‌ی سشن: \\
    در این روش سرور برای هر کاربری که لاگین می‌کند سشنی ساخته و رکوئست‌هایی که از آن دستگاه به سرور می‌آیند را در سشن مورد نظر پردازش می‌کند و متوجه کاربر درخواست دهنده می‌شود.
    \item بر پایه‌ی توکن: \\
    در این روش پس از لاگین‌کردن کاربر، سرور یک \lr{token} به کلاینت ارسال می‌کند. از این پس کلاینت با هر درخواست این توکن را به سمت سرور می‌فرستد تا سرور متوجه کاربر درخواست‌کننده شود.
\end{itemize}

پیشنهاد ما: استفاده از احراز هویت \lr{token-based}

از لینک‌های زیر می‌توانید برای مطالعه بیش‌تر استفاده کنید:

\begin{itemize}
    \item \href{https://dev.to/thecodearcher/what-really-is-the-difference-between-session-and-token-based-authentication-2o39}{تفاوت‌های این دو نوع}
    \item \href{https://sherryhsu.medium.com/session-vs-token-based-authentication-11a6c5ac45e4}{این دو در برابر هم}
    \item \href{https://www.okta.com/identity-101/what-is-token-based-authentication/}{احراز هویت بر پایه توکن}
\end{itemize}

\subsection*{{\titr اتاق گفت‌وگو (\lr{Chat Room})}}
\addcontentsline{toc}{subsection}{{\fehrestContent اتاق گفت‌وگو (\lr{Chat Room})}}

همانطور که در فاز یک اشاره شده بود، در این فاز قصد اضافه کردن اتاق گفت‌وگو به بازی خود را داریم.

هدف اصلی ما اضافه کردن اتاقی عمومی به برنامه به منظور صحبت و گفت‌وگو میان تمام افرادی است که در برنامه ثبت‌نام کرده‌اند.

\begin{tcolorbox}[colback=green!5!white,colframe=green!75!black,title=\textbf{نکته}]
    اضافه کردن «گفت‌و‌گوی خصوصی» و یا «گفت‌وگو در حین بازی» جزو خواسته‌های اصلی این قسمت نیست و هم‌چنین نمره امتیازی نیز نخواهد داشت؛ اما برای جذاب شدن هر چه بیشتر برنامه خود می‌توانید در صورت داشتن وقت و حوصله کافی :) این ویژگی‌ها را نیز به برنامه خود اضافه کنید.
\end{tcolorbox}

در بخش اجباری تنها 
اتاق گفت‌وگوی عمومی(چت گلوبال \lr{Global Chat})
وجود دارد.

\subsection*{{\titr تابلوی امتیازات (\lr{Scoreboard})}}
\addcontentsline{toc}{subsection}{{\fehrestContent تابلوی امتیازات (\lr{Scoreboard})}}

پیاده‌سازی تابلوی امتیازات باید به گونه‌ای باشد که اطلاعات آن در سمت سرور نگه داشته شده و هر بار کلاینت برای دیدن نتایج به سرور درخواست بدهد.

\subsection*{{\titr بازی دو نفره}}
\addcontentsline{toc}{subsection}{{\fehrestContent بازی دو نفره}}
در این بازی، بازیکنان باید بتوانند با یکدیگر به صورت‌های متفاوت (۱ راند و ۳ راند) با یکدیگر مسابقه دهند. به بیان دیگر، همان انجام بازی دو نفره که در فاز ۱ پیاده‌سازی کردید، در اینجا باید توسط کلاینت‌های متفاوت امکان پذیر باشد و هر بازیکن اتفاقاتی که می‌خواهد صورت بگیرد را در نوبت خودش به سمت سرور می‌فرستد و در صورت مجاز بودن خواسته‌ی بازیکن آن اتفاق عملی شود.

همچنین واضح است که بازیکنی که نوبتش نیست نیز باید به صفحه‌ی بازی دسترسی داشته باشد و به طور همزمان کارهایی که حریف انجام می‌دهد را ببیند. البته روشن است اگر بخش‌هایی از صفحه‌ی بازی لازم است برای حریف محدود باشد در اینجا نیز باید آن را رعایت کنید. مثلا کارت‌هایی که در دست حریف قرار دارند، نباید قابل مشاهده باشند.

\subsection*{{\titr لابی}}
\addcontentsline{toc}{subsection}{{\fehrestContent لابی}}
برای اینکه بازیکنان بتوانند همدیگر را برای مسابقه دادن با هم پیدا کنند باید یک لابی طراحی کنید در لابی بازیکنان می‌توانند با یکدیگر تعامل داشته باشند. (می‌توانید قسمت چت عمومی را نیز به لابی اضافه کنید) یکی از ویژگی‌های لابی اینست که بازیکن می‌تواند درخواست بازی دونفره با توجه به نوع بازی مد نظرش بدهد و در ادامه در حالت انتظار می‌ماند و اگر شخص دیگری هم درخواست مشابه با او را داشت سرور برای آنها یک بازی می‌سازد و آنها را به هم متصل می‌کند. توجه کنید، اگر چندین نفر هم زمان درخواست مشابه داشتند آن‌ها به صورت رندم به هم متصل می‌کند.

روشن است که بازیکنان باید بتوانند به اختیار خودشان از حالت انتظار خارج شوند.

\subsection*{{\titr فروشگاه}}
\addcontentsline{toc}{subsection}{{\fehrestContent فروشگاه}}
در این فاز فروشگاه بازی باید به سرور منتقل شود و تمامی کارت‌هایی که در شاپ وجود دارند از یک ظرفیت محدود برخوردار باشند. اگر بازیکنی کارتی را خرید، ظرفیت آن در شاپ یکی کم می‌شود. در صورتی که ظرفیت کارتی تمام شد هیچ بازیکنی، دیگر نمی‌تواند از آن کارت بخرد. تمامی تغییراتی که روی اقلام شاپ اتفاق می‌افتد باید در سرور ذخیره شود. از طرفی بازیکن‌ها می توانند کارت‌هایی که دارند را بفروشند و در این صورت ظرفیت آن کارت در شاپ یکی بیشتر می‌شود. همچنین برای فروشگاه باید یک پنل ادمین ساده برای مدیریت آن ایجاد شود. از طریق این پنل ادمین همچنین باید بتوان خرید و فروش یک کارت خاص را ممنوع کرد. به عنوان مثال اگر کارت \lr{X} به مقدار ۵تا در فروشگاه موجود باشد ولی از طریق پنل ادمین ممنوع شده باشد، اگر چه کارت در فروشگاه نمایش داده می‌شود اما باید یک علامت ممنوعه روی آن زده شده و امکان خریدن آن وجود نداشته باشد. از طریق این پنل کارهای ساده‌ی دیگر مانند افزودن یا کم کردن تعداد یک کارت باید قابل انجام باشد.




\section*{{\titr بخش‌های امتیازی}}
\addcontentsline{toc}{section}{{\fehrestContent بخش‌های امتیازی}}

\end{document}
