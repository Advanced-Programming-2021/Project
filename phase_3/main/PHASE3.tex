\documentclass[]{article}
\usepackage{graphicx}
\usepackage[svgnames]{xcolor} 
\usepackage{fancyhdr}
\usepackage{fancyvrb}
\usepackage{forest}
\usepackage{tocloft}
\usepackage[hidelinks]{hyperref}
\usepackage{enumitem}
\usepackage[many]{tcolorbox}
\usepackage{listings }
\usepackage[a4paper, total={6in, 8in} , top = 2cm,bottom = 4cm]{geometry}
%\usepackage[a4paper, total={6in, 8in}]{geometry}
\usepackage{afterpage}
\usepackage{amssymb}
\usepackage{pdflscape}
\usepackage{textcomp}
\usepackage{xecolor}
\usepackage{rotating}
\usepackage[Kashida=on,KashidaXBFix=on]{xepersian}
\usepackage[T1]{fontenc}
\usepackage{tikz}
\usepackage[utf8]{inputenc}
\usepackage{PTSerif} 
\usepackage{seqsplit}
\usepackage{changepage}


\usepackage{listings}
\usepackage{xcolor}
\usepackage{sectsty}

\setcounter{secnumdepth}{0}
 
\definecolor{codegreen}{rgb}{0,0.6,0}
\definecolor{codegray}{rgb}{0.5,0.5,0.5}
\definecolor{codepurple}{rgb}{0.58,0,0.82}
\definecolor{backcolour}{rgb}{0.95,0.95,0.92}
\definecolor{blanchedalmond}{rgb}{1.0, 0.92, 0.8}
\definecolor{brilliantlavender}{rgb}{0.96, 0.73, 1.0}
 
\NewDocumentCommand{\codeword}{v}{
\texttt{\textcolor{blue}{#1}}
}
\lstset{language=java,keywordstyle={\bfseries \color{blue}}}

\lstdefinestyle{mystyle}{
    backgroundcolor=\color{backcolour},   
    commentstyle=\color{codegreen},
    keywordstyle=\color{magenta},
    numberstyle=\tiny\color{codegray},
    stringstyle=\color{codepurple},
    basicstyle=\ttfamily\normalsize,
    breakatwhitespace=false,         
    breaklines=true,                 
    captionpos=b,                    
    keepspaces=true,                 
    numbers=left,                    
    numbersep=5pt,                  
    showspaces=false,                
    showstringspaces=false,
    showtabs=false,                  
    tabsize=2
}

\lstset{style=mystyle}

 \settextfont[BoldFont={XB Zar bold.ttf}]{XB Zar.ttf}


\setlatintextfont[Scale=1.0,
 BoldFont={LiberationSerif-Bold.ttf}, 
 ItalicFont={LiberationSerif-Italic.ttf}]{LiberationSerif-Regular.ttf}





\newcommand{\inputsample}[1]{
    ~\\
    \textbf{ورودی نمونه}
    ~\\
    \begin{tcolorbox}[breakable,boxrule=0pt]
        \begin{latin}
            \large{
                #1
            }
        \end{latin}
    \end{tcolorbox}
}

\newcommand{\outputsample}[1]{
    ~\\
    \textbf{خروجی نمونه}

    \begin{tcolorbox}[breakable,boxrule=0pt]
        \begin{latin}
            \large{
                #1
            }
        \end{latin}
    \end{tcolorbox}
}

\newtcolorbox{mybox}[2][]{colback=red!5!white,
colframe=red!75!black,fonttitle=\bfseries,
colbacktitle=red!85!black,enhanced,
attach boxed title to top center={yshift=-2mm},
title=#2,#1}

\newenvironment{changemargin}[2]{%
\begin{list}{}{%
\setlength{\topsep}{0pt}%
\setlength{\leftmargin}{#1}%
\setlength{\rightmargin}{#2}%
\setlength{\listparindent}{\parindent}%
\setlength{\itemindent}{\parindent}%
\setlength{\parsep}{\parskip}%
}%
\item[]}{\end{list}}


\definecolor{foldercolor}{RGB}{124,166,198}
\definecolor{sectionColor}{HTML}{ff5e0e}
\definecolor{subsectionColor}{HTML}{008575}

\definecolor{listColor}{HTML}{00d3b9}

\definecolor{umlrelcolor}{HTML}{3c78d8}

\definecolor{subsubsectionColor}{HTML}{3c78d8}

\defpersianfont\authorFont[Scale=0.9]{XB Zar bold.ttf}

\defpersianfont\titr[Scale=1.5]{Lalezar-Regular.ttf}

\defpersianfont\fehrest[Scale=1.2]{Lalezar-Regular.ttf}

\defpersianfont\fehrestTitle[Scale=3.0]{Lalezar-Regular.ttf}

\defpersianfont\fehrestContent[Scale=1.2]{XB Zar bold.ttf}


\sectionfont{\color{sectionColor}}  % sets colour of sections
\subsectionfont{\color{subsectionColor}}  % sets colour of sections
\subsubsectionfont{\color{subsubsectionColor}}


\renewcommand{\labelitemii}{$\circ$}


\renewcommand{\baselinestretch}{1.1}


\renewcommand{\contentsname}{فهرست}

\renewcommand{\cfttoctitlefont}{\fehrestTitle}


\renewcommand\cftsecfont{\color{sectionColor}\fehrestContent\selectfont}
\renewcommand\cftsubsecfont{\color{subsectionColor}\fehrestContent\selectfont}
\renewcommand\cftsubsubsecfont{\color{subsubsectionColor}\fehrestContent\selectfont}
%\renewcommand{\cftsecpagefont}{\color{sectionColor}}

\setlength{\parskip}{1.2pt}

\begin{document}


%%% title pages
\begin{titlepage}
\begin{center}

\textbf{ \Huge{به نام خدا} }
        
\vspace{0.2cm}

\includegraphics[width=0.4\textwidth]{sharif1.png}\\
\vspace{0.2cm}
\textbf{ \Huge{\emph درس برنامه‌سازی پیشرفته} }\\
\vspace{0.25cm}
\textbf{ \Large{ فاز سوم پروژه} }
\vspace{0.2cm}
       
 
      \large \textbf{دانشکده مهندسی کامپیوتر}\\\vspace{0.1cm}
    \large   دانشگاه صنعتی شریف\\\vspace{0.2cm}
       \large   ﻧﯿﻢ سال دوم 00-99 \\\vspace{0.10cm}
      \noindent\rule[1ex]{\linewidth}{1pt}
استاد:\\
    \textbf{{دکتر محمدامین فضلی}}



    \vspace{0.20cm}

   مهلت ارسال:\\
    \textbf{شنبه ۲۶ تیر ۱۴۰۰}
    \textbf{ساعت ۲۳:۵۹:۵۹}

    \vspace{0.10cm}
مسئول پروژه:\\
    \textbf{\authorFont{امیرمهدی نامجو}}
    
        \vspace{0.10cm}
مسئول فاز سوم:\\
    \textbf{\authorFont{علیرضا شاطری}}
    
        \vspace{0.10cm}
طراحان فاز سوم:\\
    \textbf{\authorFont{مازیار شمسی‌پور، مجید طاهرخانی، احسان موفق و محمدجواد هزاره }}
    
        \vspace{0.05cm}
مسئولین تنظیم مستند:\\
    \textbf{\authorFont{پارسا محمدیان و سروش جهان‌زاد}}
    

\end{center}
\end{titlepage}
%%% title pages


%%% header of pages
\newpage
\pagestyle{fancy}
\fancyhf{}
\fancyfoot{}
\cfoot{\thepage}
\lhead{فاز سوم}
\rhead{\includegraphics[width=0.1\textwidth]{sharif.png}\\
دانشکده مهندسی کامپیوتر
}
\chead{پروژه برنامه‌سازی پیشرفته}
%%% header of pages
\renewcommand{\headrulewidth}{2pt}

\KashidaOff



\tableofcontents

\newpage

 \Large \textbf{\\\\
}


\section*{{\titr نکات قابل توجه}}
\addcontentsline{toc}{section}{{\fehrestContent نکات قابل توجه}}
\begin{itemize}
\item
پس از اتمام این فاز، در گیت خود یک تگ با ورژن \lr{"v3.0.0"} بزنید. در روز تحویل حضوری این tag بررسی خواهد شد و کدهای پس از آن نمره‌ای نخواهد گرفت. برای اطلاعات بیش‌تر در مورد شیوه ورژن‌گذاری، می‌توانید به
 \href{https://semver.org/}{\textcolor{blue}{\underline{این لینک}}}
 مراجعه کنید. البته برای این پروژه صرفا رعایت کردن همان ورژن گفته شده کافیست، اما خوب‌ است که با منطق ورژن‌بندی هم آشنا بشوید.

\item
در روز تحویل حضوری مشارکت تمام اعضای تیم در پروژه بررسی خواهد‌ شد و در صورت عدم مشارکت بعضی از اعضا، نمره‌ی ایشان برای آن فاز پروژه "صفر" لحاظ می‌گردد. مشارکت، با توجه به commit های افراد تیم در مخزن گیت‌هاب پروژه بررسی می‌شود.

\item
در دو فاز اول می‌توانید سه روز تاخیر به ازای کسر نمره داشته‌ باشید که به ازای هر روز آن، ۱۰ درصد از نمرهٔ آن فاز را از دست خواهید‌ داد. در مجموع سه‌فاز پروژه، سه روز تاخیر نیز بخشیده خواهد‌ شد. \textbf{توجه کنید که به دلیل نزدیک بودن به مهلت ارسال نمرات، امکان تاخیر برای فاز سوم وجود ندارد.}


\item
در صورت کشف تقلب از هریک از تیم‌ها، برای بار اول منفی نمرهٔ آن فاز برای آن تیم ثبت می‌شود و برای بار دوم، نمرهٔ منفی کل پروژه برای تیم لحاظ خواهد‌ شد که معادل مردود شدن در درس است.


\end{itemize}

\newpage

\section*{{\titr مقدمه}}
\addcontentsline{toc}{section}{{\fehrestContent مقدمه}}

همانطور که می‌دانید، در دو فاز قبلی بدنه‌ی اصلی مورد نیاز برای اجرای بازی روی یک کامپیوتر را پیاده‌سازی کردیم؛ یعنی بخش‌های مربوط به منطق و گرافیک بازی که برای کامل بودن بازی کافی هستند. پس در این بخش می‌خواهیم چه کنیم؟ قرار است قابلیت‌هایی را به بازی‌مان اضافه کنیم تا بشود بر بستر اینترنت هم بازی را اجرا و با بقیه بازیکنان از راه دور بازی کرد. همچنین این فاز بخش‌های امتیازی متنوع و زیادی دارد که برای جبران نمرات از دست‌ رفته در دو فاز قبلی می‌توانید از آن‌ها استفاده کنید. در ادامه بخش‌های اصلی و امتیازی را به صورت جداگانه می‌توانید ببینید.

\section*{{\titr بخش‌های اصلی}}
\addcontentsline{toc}{section}{{\fehrestContent بخش‌های اصلی}}

\subsection*{{\titr معماری \lr{Client-Server}}}
\addcontentsline{toc}{subsection}{{\fehrestContent معماری \lr{Clinet-Server}}}

معماری کارخواه-کارگزار یا \lr{Client-Server} و معماری همتا به همتا \lr{Peer-to-Peer} معروف‌ترین معماری‌های شبکه‌های کامپیوتری هستند. در معماری \lr{Client-Server}، کاربران معمولی کلاینت نامیده شده و هر کدام از آن‌ها درخواست‌هایی را برای سرور ارسال می‌کنند. سرور به منابع و اطلاعات اصلی برنامه دسترسی دارد و پردازش‌های اصلی داده‌ها در آن انجام شده و در نهایت نتیجه به شکل مناسبی به کلاینت اطلاع داده می‌شود.

برای این فاز پروژه، توصیه می‌شود از معماری کلاینت سرور استفاده نمایید. به این شکل که سرور، اطلاعات اصلی نظیر بازی‌های در حال انجام، فروشگاه و... را در اختیار داشته و بسته به درخواست‌هایی که برای آن ارسال می‌شود، پاسخ مناسب را برای هر کلاینت ارسال می‌کند. به بیان دیگر، بخش عمده منطق برنامه باید در سمت سرور رسیدگی شود.

برای آشنایی بیش‌تر با این معماری و نحوه پیاده‌سازی آن، می‌توانید به کارگاه شبکه که پیش‌تر برگزار شده است، مراجعه نمایید.




\subsection*{{\titr احراز هویت}}
\addcontentsline{toc}{subsection}{{\fehrestContent احراز هویت}}

فرض کنید در دنیای واقعی می‌خواهید وارد گروهی شوید که برای اعضایش قابلیت‌های ویژه‌ای در نظر گرفته شده است. شما ثبت‌نام می‌کنید و مسئولین گروه مشخصات شما را یادداشت می‌کنند؛ سپس به شما کارت عضویت داده می‌شود. شما با این کارت احراز هویت می‌شوید و می‌توانید از شرایط خاص گروه بهره‌مند شوید.

احراز هویت در برنامه‌نویسی نیز خاصیت مشابهی دارد. شما پس از ثبت‌نام و احراز هویت در سایت مورد نظر، دارای حساب کاربری می‌شوید و می‌توانید از ویژگی‌های خاص آن سایت استفاده کنید.

حال سوالی که مطرح می‌شود این است که احراز هویت به چه شکل صورت می‌گیرد؟

در ابتدا کاربر باید ثبت‌نام کند. به صورت عادی از کاربران نام‌کاربری و رمز عبور خواسته می‌شود (اطلاعات فرد). پس از تایید ثبت‌نام، کاربر دارای حساب کاربری شده و می‌تواند با نام‌کاربری و رمز عبور وارد حساب خود شود (کارت عضویت).

در برنامه‌‌های رایانه‌ای، سرور برای پردازش هر درخواست (\lr{request}) از سمت کلاینت، نیاز دارد که کاربر درخواست‌دهنده را بشناسد.

عموما از دو روش زیر سرور متوجه هویت کاربر حاضر می‌شود.


\begin{itemize}
    \item بر پایه‌ی نشست (\lr{session}): \\
    در این روش سرور برای هر کاربری که وارد می‌شود (\lr{login} می‌کند) یک نشست می‌سازد و درخواست‌هایی که از آن دستگاه به سرور می‌آیند را در نشست مورد نظر پردازش می‌کند و از این راه متوجه هویت کاربر درخواست‌دهنده می‌شود.
    \item بر پایه‌ی توکن (\lr{token}): \\
    در این روش پس از ورود کاربر، سرور یک توکن (\lr{token}) به کلاینت ارسال می‌کند. از این پس کلاینت در کنار هر درخواست این توکن را هم به سمت سرور می‌فرستد تا سرور متوجه هویت کاربر درخواست‌کننده شود.
\end{itemize}

ما به شما پیشنهاد می‌کنیم که از احراز هویت بر پایه‌ی توکن (\lr{token-based authorization}) استفاده کنید.

از لینک‌های زیر می‌توانید برای مطالعه بیش‌تر استفاده کنید:

\begin{itemize}
    \item \href{https://dev.to/thecodearcher/what-really-is-the-difference-between-session-and-token-based-authentication-2o39}{\textcolor{blue}{\underline{تفاوت‌های این دو روش احراز هویت}}}
    \item \href{https://sherryhsu.medium.com/session-vs-token-based-authentication-11a6c5ac45e4}{\textcolor{blue}{\underline{بررسی این دو روش در برابر هم}}}
    \item \href{https://www.okta.com/identity-101/what-is-token-based-authentication/}{\textcolor{blue}{\underline{احراز هویت بر پایه‌ی توکن}}}
\end{itemize}

\subsection*{{\titr اتاق گفت‌وگو (\lr{Chat Room})}}
\addcontentsline{toc}{subsection}{{\fehrestContent اتاق گفت‌وگو (\lr{Chat Room})}}

همانطور که در فاز یک اشاره شده بود، در این فاز قصد داریم ویژگی اتاق گفت‌وگو را به بازی خود اضافه کنیم.

هدف اصلی ما اضافه کردن اتاقی عمومی به برنامه است تا تمام افرادی که در برنامه ثبت‌نام کرده‌اند بتوانند در آن صحبت و گفت‌وگو کنند. در بخش اصلی تنها باید
اتاق گفت‌وگوی عمومی (چت همگانی یا \lr{Global Chat})
را پیاده‌سازی کنید.
\\

\begin{tcolorbox}[colback=green!5!white,colframe=green!75!black,title=\textbf{نکته}]
    اضافه کردن ویژگی «گفت‌و‌گوی خصوصی» و یا «گفت‌وگو در حین بازی» جزو خواسته‌های اصلی این قسمت نیست و نمره امتیازی نیز نخواهد داشت؛ اما برای جذاب‌تر شدن هرچه بیشتر برنامه می‌توانید در صورت داشتن وقت و حوصله کافی :) این ویژگی‌ها را نیز به برنامه خود اضافه کنید.
\end{tcolorbox}


\subsection*{{\titr تابلوی امتیازات (\lr{Scoreboard})}}
\addcontentsline{toc}{subsection}{{\fehrestContent تابلوی امتیازات (\lr{Scoreboard})}}

پیاده‌سازی تابلوی امتیازات باید به گونه‌ای باشد که اطلاعات آن در سمت سرور نگهداری شود و هر بار کلاینت برای دیدن نتایج به سرور درخواست بدهد.

\subsection*{{\titr بازی دو نفره}}
\addcontentsline{toc}{subsection}{{\fehrestContent بازی دو نفره}}
در این نوع از بازی، بازیکنان باید بتوانند به صورت‌های مختلف (شامل ۱ دور و ۳ دور) با یکدیگر مسابقه دهند. به بیان دیگر، در این بخش همان بازی دو نفره که در فاز ۱ پیاده‌سازی کردید باید در دو کلاینت مختلف به صورت هم‌زمان اجرا شوند و هر بازیکن در یک کلاینت جداگانه به بازی وارد شود و بازی را انجام دهد. در این حالت هر کلاینت اتفاقات مدنظر بازیکن را در نوبت خودش به سمت سرور می‌فرستد و در صورت مجاز بودن خواسته‌ها، آن اتفاقات عملی می‌شوند و سرور پاسخ مناسب را در صورت نیاز به کلاینت‌ها می‌فرستد.

همچنین واضح است که بازیکنان در نوبت حریف نیز می‌توانند به صفحه‌ی بازی دسترسی داشته باشد و به طور همزمان کارهایی که حریف انجام می‌دهد را ببیند؛ البته قوانین اصلی باید رعایت شوند و بخش‌هایی از صفحه‌ی بازی که فقط در دسترس خود بازیکن است باید از حریف پنهان باشند. مثلا کارت‌هایی که در دست حریف قرار دارند، نباید قابل مشاهده باشند.

\subsection*{{\titr لابی}}
\addcontentsline{toc}{subsection}{{\fehrestContent لابی}}
برای اینکه بازیکنان بتوانند همدیگر را برای مسابقه دادن پیدا کنند به یک لابی نیاز داریم. در لابی بازیکنان می‌توانند با یکدیگر تعامل داشته باشند (می‌توانید قسمت چت عمومی را نیز به لابی اضافه کنید). ویژگی دیگر لابی این است که هر بازیکن می‌تواند برای بازی دونفره درخواست بدهد و نوع بازی مد نظرش را در آن درخواست مشخص کند. پس از آن، اگر شخص دیگری هم درخواستی مشابه او داشت سرور برای آنها یک بازی می‌سازد و آنها را به هم متصل می‌کند. 

توجه کنید تا زمانی که درخواست مشابهی وجود نداشت، بازیکن در حالت انتظار باقی می‌ماند، مگر این که به اختیار خودش از حالت انتظار خارج شود. همچنین اگر چندین نفر به طور هم‌زمان درخواست‌های مشابه داشتند، سرور آن‌ها را به صورت تصادفی به یکدیگر متصل می‌کند.

\subsection*{{\titr فروشگاه}}
\addcontentsline{toc}{subsection}{{\fehrestContent فروشگاه}}
در این فاز فروشگاه بازی باید به سرور منتقل شود و تمامی تغییراتی که روی اقلام فروشگاه رخ می‌دهد در سرور ذخیره می‌شوند. همچنین تمامی کارت‌هایی که در فروشگاه موجودی مشخصی خواهند داشت و اگر بازیکنی کارتی را خرید، موجودی آن کارت در فروشگاه یک واحد کم می‌شود. در صورتی که موجودی کارتی تمام شد، هیچ بازیکنی دیگر نمی‌تواند آن کارت را بخرد. از طرفی بازیکن‌ها می توانند کارت‌هایی که دارند را بفروشند و در این صورت موجودی آن کارت‌ها در فروشگاه به تعداد فروخته‌شده بیشتر می‌شود.

علاوه بر این، باید برای فروشگاه یک پنل ادمین ساده ایجاد کنید تا بتوان آن را مدیریت کرد. از طریق این پنل می‌شود خرید و فروش کارت‌های به خصوص را ممنوع کرد. به عنوان مثال اگر از کارت \lr{X} به مقدار ۵تا در فروشگاه موجود باشد ولی از پنل مدیریت خریدش ممنوع شده باشد، کارت همچنان در فروشگاه نمایش داده می‌شود اما یک علامت ممنوعه روی آن زده می‌شود تا نشانگر این وضعیت باشد و امکان خرید آن وجود نخواهد داشت. کارهای ساده‌ی دیگر مانند افزودن یا کم کردن موجودی یک کارت نیز از این پنل در دسترس هستند.


\section*{{\titr بخش‌های امتیازی}}
\addcontentsline{toc}{section}{{\fehrestContent بخش‌های امتیازی}}

\subsection*{{\titr اتاق گفت‌وگو (\lr{Chat Room})}}
\addcontentsline{toc}{subsection}{{\fehrestContent اتاق گفت‌وگو (\lr{Chat Room})}}

علاوه بر اتاق گفت‌و‌گوی عمومی، می‌توانید بخش‌های زیر را هم به صورت امتیازی پیاده‌سازی کنید.

\begin{itemize}
	\item قابلیت حذف پیام
	\item  قابلیت ویرایش (\lr{edit}) پیام
	\item قابلیت پاسخ دادن (\lr{reply}) پیام
	\item قابلیت سنجاق کردن (\lr{pin}) پیام
	\item نمایش آواتار ارسال کننده پیام
	\item نمایش اطلاعات کاربری فرد با کلیک بر روی آواتار او
	\item نمایش تعداد افراد آنلاین (آنلاین به این معنا که در حال حاضر در برنامه لاگین کرده و حضور دارند، نه این‌که لزوما در صفحه مربوط به اتاق گفت‌وگو باشند!).
\end{itemize}

\subsection*{{\titr دستاورد‌ها (\lr{Achievements})}}
\addcontentsline{toc}{subsection}{{\fehrestContent دستاورد‌ها (\lr{Achievements})}}

بازیکنان در طول اجرای بازی مهارت‌های مختلفی بدست خواهند آورد و تجربیات و کنش‌های مختلفی را پشت سر خواهند گذاشت. به پاس قدردانی از زحمات آن‌ها، نشان‌هایی برای آن‌ها در نظر بگیرید که اگر موفق به انجام کنش مورد نظر شدند، به آن‌ها تعلق بگیرد و در پروفایل کاربری آن‌ها درج شود. این کنش‌ها را در ادامه مطرح می‌کنیم.

هم‌چنین علاوه بر در نظر گرفتن نشان می‌توانید وابسته به سختی این کنش‌ها، مقداری پاداش برای هر عمل در نظر بگیرید. این پاداش می‌تواند به صورت اهدای یک کارت به بازیکن و یا اهدای مقداری پول و یا امتیاز به او باشد.

\textbf{\emph{تعریف کنش‌ها}}

در ادامه تعدادی کنش مطرح می‌شود که مقداردهی به پارامترهای آن‌ها پیشنهادی است اما به ذوق و سلیقه خودتان می‌توانید هر عدد معقولی بگذارید.

\begin{itemize}
	\item دنباله‌ای متوالی از پیروزی‌ها به تعداد مناسب‌ (مثلا ۵، ۱۰، ۱۵ و ...)
	\item پیروزی در تعداد مناسبی‌ (مثلا ۵) مسابقه فقط با استفاده از کارت‌های هیولا
	\item دنباله‌ای متوالی از پیروزی‌ها با محدودیت کاهش سلامتی مثلا سه برد متوالی با سلامتی بیش از ۳/۴ مقدار اولیه
	\item پیروزی در تعداد محدودی نوبت
	\item دنباله‌ای متوالی از شکست‌ها به تعداد مناسب
	\item اولین - دهمین - صدمین و ... پیام اتاق گفت‌وگو
	\item احضار کارت به صورت آیینی (\lr{Ritual Summon})
	\item ساخت اولین زنجیره
\end{itemize}


\begin{tcolorbox}[colback=green!5!white,colframe=green!75!black,title=\textbf{نکته}]
	دقت کنید که برخی از موارد بالا را می‌توان به صورت لحظه‌ای \lr{(real time)} پیاده‌سازی کرد؛ به این معنا که بلافاصله بعد از برآورده شدن شرط مربوطه، دستاورد به کاربر تعلق بگیرد و پیام مناسب نمایش داده شود؛ پیاده‌سازی این موارد به صورت لحظه‌ای نمره امتیازی خواهد داشت. امتیازدهی به کنش‌ها نیز متناسب با سختی پیاده‌سازی آن‌ها خواهد بود.
\end{tcolorbox}

\subsection*{{\titr بخش تلویزیون (\lr{TV})}}
\addcontentsline{toc}{subsection}{{\fehrestContent بخش تلویزیون (\lr{TV})}}

صفحه‌ای را فرض کنید که دارای سه قسمت (\lr{tab}) پخش زنده، باز‌پخش و پخش‌ بازی‌های بازیکنان برتر باشد.
از آنجا که منطق کلی هر بخش با بخش دیگر یکسان است و همه شامل نمایش یک بازی می‌شوند، می‌توان همه را در یک صفحه با نام تلویزیون (\lr{TV}) گنجاند. در ادامه به شرح هر سه بخش می‌پردازیم.
\\

\begin{itemize}
	\item \textbf{\emph{پخش زنده‌ی بازی‌ها}}
\end{itemize}
هدف این قسمت به‌اشتراک‌گذاری برخط بازی‌هاست و در واقع مانند استریم بازی‌ها عمل می‌کند. هر کاربری که در حال بازی کردن باشد می‌تواند پخش زنده‌ی بازی خود را به اشتراک بگذارد.
روش‌ پیشنهادی ما برای این بخش ثبت گزارش (\lr{Log}) است. این روش به صورت زیر قابل اجراست.

ابتدا حالت اولیه بازی (شامل کارت‌های هر بازیکن و اسپل‌ها و ...) ذخیره می‌شود. 
سپس هر حرکتی که رخ بدهد ذخیره می‌شود. به عبارت دیگر پس از هر اتفاق، جزئیات آن در قالب مشخصی ذخیره می‌شود. سپس چون حالت اولیه‌ی بازی را می‌دانیم برای هر کاربری که قصد مشاهده‌ی پخش زنده را داشته باشد یک نمونه از بازی می‌سازیم و تمامی حرکات را براساس این گزارش‌ها اجرا می‌کنیم تا به حالت فعلی بازی برسیم. پس از آن نیز به طور مداوم با استفاده از ادامه‌ی گزارش‌ها پخش زنده ادامه پیدا می‌کند.
\\

\begin{itemize}
	\item \textbf{\emph{مشاهده‌ی دوباره‌ی بازی}}
\end{itemize}
کاربر می‌تواند در این بخش هر بازی خود را دوباره مشاهده و بررسی کند. 
به کمک روش ثبت \lr{Log} می‌توان هر بازی را ذخیره کرد و در هر حرکت، به حرکت‌های قبلی و بعدی در بازی دست یافت. همچنین این حرکت‌ها می‌توانند به صورت خودکار و مانند یک فیلم طبیعی، به صورت خودکار و پشت سر هم پخش شوند.
\\

\begin{itemize}
	\item \textbf{\emph{مشاهده بازی‌های ضبط شده‌ی بازیکنان برتر}}
\end{itemize}
بازی بازیکنان برتر ذخیره می‌شود و بعد‌ها نیز می‌توان این بازی‌ها را مشاهده کرد. این بخش هم به کمک ثبت \lr{Log} می‌تواند پیاده‌سازی شود.
این بازی‌ها در کنار اسکوربرد قابل‌دسترسی هستند. نگهداری چند بازی اخیر بازیکنان برتر کافیست و نیازی نیست بازی‌ها تا ابد نگهداری شوند.

\subsection*{{\titr امتیازدهی به بازی‌ها}}
\addcontentsline{toc}{subsection}{{\fehrestContent امتیازدهی به بازی‌ها}}

برای این بخش باید قابلیتی در نظر بگیرید تا بینندگان بتوانند به بازی‌های برتر که در بخش تلویزیون نشان داده می‌شوند امتیاز بدهند. برای این بخش نکات زیر را در نظر داشته باشید.
\begin{itemize}
	\item داشتن سیستم امتیازدهی ساده
	\item به منظور قابل اطمینان بودن این امتیاز، باید به نوعی صلاحیت فرد امتیاز دهنده احراز شود؛ مثلا حتما باید ۷۵ درصد از نوبت‌های بازی را تماشا کرده باشد (یا به هر روش دیگری که صلاح می‌دانید!)
\end{itemize}

\subsection*{{\titr بازی دو نفره}}
\addcontentsline{toc}{subsection}{{\fehrestContent بازی دو نفره}}

\textbf{\emph{ساختن هوشمند بازی (\lr{Matchmaking}):}}
با این قابلیت، سرور در هنگام ساخت بازی جدید و متصل کردن بازیکنان به یکدیگر به گونه‌ای عمل می‌کند که بازیکنانی با فاصله‌ی امتیازی قابل توجه به یکدیگر متصل نشوند و اگر تا مدتی در صف انتظار بودند و شخص مناسبی پیدا نشد به یکدیگر متصل شوند.

دقت کنید که باید برای تحویل این ویژگی، سناریویی را از پیش در نظر بگیرید. به عبارت دیگر در هنگام تحویل حضوری پروژه شخص تحویل‌گیرنده متوجه سه موضوع زیر شود و در غیر این صورت، امتیاز این بخش را به صورت کامل دریافت نخواهید کرد.

\begin{itemize}
	\item سرور بازیکنان با فاصله‌ی امتیازی زیاد را با تاخیر به هم متصل می‌کند.
	\item سرور بازیکنانی که امتیاز نزدیک به هم دارند را بدون وقفه به هم متصل می‌کند.
	\item اگر بازیکنان \lr{A} و \lr{B} با فاصله‌ی امتیازی قابل توجه در حال انتظار باشند و در این بین بازیکن C که امتیاز نزدیک به A دارد درخواست بازی داد، سرور A را به C متصل می‌کند و B در حال انتظار می‌ماند.
\end{itemize}

\subsection*{{\titr محدودیت زمانی هر نوبت}}
\addcontentsline{toc}{subsection}{{\fehrestContent محدودیت زمانی هر نوبت}}

برای هر نوبت (\lr{turn}) محدودیت زمانی خاصی در نظر بگیرید. اگر بازیکنی که نوبت اوست تا پایان آن زمان حرکت خود را انجام ندهد و زمان نوبت به پایان برسد بازی را می بازد و بازیکن رقیب برنده می‌شود.

\subsection*{{\titr امکان بازگشت به بازی بعد از قطع شدن اتصال کلاینت با سرور}}
\addcontentsline{toc}{subsection}{{\fehrestContent امکان بازگشت به بازی بعد از قطع شدن اتصال کلاینت با سرور}}

اگر در میانه ی یک نوبت ارتباط بازیکنی که نوبت اوست با سرور قطع شود، تا زمانی که دوباره به سرور وصل شود زمان نوبت ثابت نگه داشته می شود و بعد از اتصال دوباره زمان شروع به حرکت می‌کند. البته برای مدت این قطع شدن نیز باید سقفی در نظر بگیرید و اگر پس از گذشت آن زمان، آن بازیکن به بازی برنگشته بود بازنده‌ی آن بازی شمرده خواهد شد و رقیبش برنده می‌شود.

\subsection*{{\titr مزایده در فروشگاه}}
\addcontentsline{toc}{subsection}{{\fehrestContent مزایده در فروشگاه}}

یکی دیگر از ویژگی‌های امتیازی که می‌توانید در این فاز به بازی اضافه کنید مزایده است. این ویژگی به این صورت است که هر بازیکن کارتی را از میان کارت‌هایی که در اختیار دارد انتخاب می‌کند و پس از تعیین قیمت اولیه، آن را به صورت مزایده برای فروش می‌گذارد. بازیکنان می‌توانند بیشترین هزینه‌ای را که تا به حال برای آن کارت پیشنهاد شده است را ببینند و در صورت تمایل، مبلغ بیشتری را برای آن پیشنهاد دهد. در نهایت پس از سپری شدن زمان مشخص شده برای اعتبار آن مزایده، بازیکنی که بیشترین پیشنهاد را برای کارت داده باشد برنده‌ی مزایده می‌شود و با پرداخت آن مبلغ صاحب کارت می‌شود. 

در مورد نحوه‌ی کسر اعتبار از بازیکن برنده در ادامه توضیحاتی ‌داده شده است. برای همه‌ی مزایده‌ها مدت زمان مشخصی در نظر بگیرید. همچنین هر بار که یک بازیکن قیمت پیشنهادی را افزایش می‌دهد اگر زمان باقی‌مانده از یک مقدار مشخص دیگر کمتر بود، برابر آن مقدار قرار داده می‌شود. انتخاب این حداقل زمان و همچنین مدت زمان مزایده‌ها به سلیقه‌ی خودتان خواهد بود.

برای پیاده‌سازی این بخش علاوه بر اضافه کردن محلی برای شروع مزایده توسط هر بازیکن باید بخشی به بازی اضافه کنید که بازیکنان بتوانند مزایده‌های فعال را ببیند و وارد آن‌ها شود.
\\

\textbf{\emph{ بخش اضافه کردن مزایده }}

در این بخش بازیکن به کارت‌های در اختیارش دسترسی دارد و می‌تواند یکی از آنها را انتخاب کند. پس از انتخاب کارت تنها با مشخص کردن یک قیمت اولیه بازیکن می‌تواند مزایده را شروع کند.
\\

\textbf{\emph{ بخش مشاهده مزایده‌های فعال }}

باید در منو‌های هر بازیکن صفحه‌ای باشد که لیست تمامی مزایده‌های فعال همراه با آخرین قیمت پیشنهاد شده، زمان باقی‌مانده و عنوان کارت قابل مشاهده باشد. بازیکنان در این صفحه می‌توانند پیشنهاد خود برای افزایش قیمت کارت‌ها را ارائه دهند. توجه کنید که با افزایش قیمت پیشنهادی کارت‌ها، از اعتبار بازیکن به اندازه‌ی قیمتی که پیشنهاد می‌دهد کم می‌شود. همچنین اعتبار بازیکنی که پیشنهاد قبلی را داده بود به او برگردانده می‌شود. توجه کنید که این صفحه باید یک دکمه برای رفرش کردن قیمت‌ها و زمان‌های باقی مانده داشته باشد (در صورتی که بخش امتیازی «نیاز نداشتن به رفرش» را پیاده‌سازی نمی‌کنید).

\subsection*{{\titr نمایش بازیکنان آنلاین در اسکوربرد}}
\addcontentsline{toc}{subsection}{{\fehrestContent نمایش بازیکنان آنلاین در اسکوربرد}}

وضعیت آنلاین بودن بازیکنان باید در صفحه‌ی امتیازات (اسکوربرد یا \lr{scoreboard}) مشخص شود. برای این‌کار باید لیستی از سوکت‌هایی که بازیکنان با آن به سرور متصل هستند نگه‌داری کنید (احتمالا در بخش‌های دیگر پروژه نیز به این نیاز پیدا می‌کنید) و هربار که شخصی درخواست مشاهده‌ی اسکوربرد را داشت سرور باید بازیکنانی که سوکت‌ آنها باز است را آنلاین نشان دهد. همچنین باید در اسکوربرد یک دکمه‌ی رفرش قرار دهید که با استفاده از آن اطلاعات موجود از جمله وضعیت آنلاین بودن بازیکنان بروزرسانی شود.

\subsection*{{\titr امکان ارسال دعوت‌نامه برای بازی دو نفره}}
\addcontentsline{toc}{subsection}{{\fehrestContent امکان ارسال دعوت‌نامه برای بازی دو نفره}}

با این ویژگی، بازیکنان می‌توانند برای یکدیگر دعوت‌نامه بفرستند. این دعوت‌نامه به این صورت است که بازیکنان هنگام ارسال درخواست برای شروع بازی، علاوه بر مشخص کردن نوع، رقیب مورد نظر برای مسابقه را هم می‌توانند مشخص کنند. پس از این کار، روند مانند قبل است و منتظر پاسخ سرور می‌مانند. اگر بازیکنی که به او درخواست داده می‌شود این درخواست را بپذیرد سرور آن دو را به هم متصل می‌کند و اگر پیشنهاد را رد کند به بازیکن درخواست‌دهنده اطلاع داده می‌شود و از حالت انتظار خارج می‌شود.

همچنین مانند قبل، بازیکن باید خودش نیز توانایی خارج شدن از حالت انتظار را داشته باشد. در این حالت با توجه به سلیقه‌ی خودتان می‌توانید درخواست داده شده را حذف کنید یا به صورت دیگری عمل کنید، مثلا اگر بازیکنی که دعوت شده بود درخواست را قبول کند به او خبر داده شود که آن دعوتنامه منقضی شده است.

\subsection*{{\titr نیاز نداشتن به رفرش}}
\addcontentsline{toc}{subsection}{{\fehrestContent نیاز نداشتن به رفرش}}
در بخش مشاهده‌ی مزایده‌های فعال و اسکوربرد اگر امکانی را فراهم کنید که بدون نیاز به دکمه‌ی رفرش، اطلاعات صفحه به صورت خودکار بروزرسانی شوند امتیاز این بخش را دریافت می‌کنید و دیگر نیازی به پیاده‌سازی دکمه‌ی رفرش نخواهید داشت. به عبارت دیگر شرایط زیر باید فراهم شوند.

\begin{itemize}
	\item  \textbf{\emph{در بخش مزایده‌های فعال: }}
	اگر بازیکن A در صفحه‌ی مزایده‌ها باشد و بازیکن B مزایده‌ی جدیدی را به وجود بیاورد، بازیکن A بدون این‌که کاری انجام دهد مزایده‌ی اضافه شده را می‌بیند. همچنین اگر بازیکن C قیمت جدیدی را برای مزایده‌ای پیشنهاد داد بازیکن A متوجه می‌شود.
	\item  \textbf{\emph{در بخش اسکوربرد:}}
	اگر بازیکن A درصفحه‌ی اسکوربرد باشد و بازیکن B به بازی وارد شود بازیکن A باید متوجه این تغییر شود. همچنین اگر امتیاز بازیکن C افزایش یابد این تغییر باید در آن صفحه مشهود باشد.
\end{itemize}

\subsection*{{\titr استفاده از پایگاه‌داده (\lr{Database})}}
\addcontentsline{toc}{subsection}{{\fehrestContent استفاده از پایگاه داده (\lr{Database})}}

پایگاه داده مجموعه‌ای از داده‌ها است که به صورت منظم و با ساختار خاصی درون رایانه نگهداری می‌شوند. پایگاه‌های داده معمولا توسط سامانه‌های مدیریت پایگاه داده (\lr{Database Management Systems} یا DBMS ها) کنترل می‌شوند. شما می‌توانید پایگاه داده‌ی خود را درون یک فایل به صورت ساده نگهداری کنید اما مهندسان بسیاری روی DBMS ها کار کرده‌اند و با توجه به شرایط کامپیوترهای موجود، DBMS هایی را توسعه داده‌اند که عملیات های معمول مانند جستجو، دریافت داده‌هایی با شرایط خاص و تغییر در پایگاه داده را با سرعت بالا و به راحتی انجام دهند. در ادامه با برخی از انواع DBMS ها و نحوه‌ی استفاده از آن‌ها آشنا می‌شویم.
\\
\begin{itemize}
\item \textbf{\emph{ پایگاه‌داده‌های رابطه‌ای (\lr{RDBMS}) }}
\end{itemize}

این خانواده از DBMS ها بسیار وسیع و در عین حال پرکاربرد هستند. RDBMS ها از زبان درخواست ساخت‌یافته (\lr{Structured Query Language} یا \lr{SQL}) استفاده می‌کنند. در این نوع از پایگاه‌های داده، داده‌ها در جدول‌هایی با ویژگی‌های مشخص ذخیره می‌شوند. با استفاده از این جدول‌ها می‌توانیم عملیاتی را مانند انتخاب (Select)، درج (Insert)، به روزرسانی (Update) و حذف (Delete) به همراه سایر ابزارهای مورد نیاز برای مدیریت پایگاه داده‌ها در کنار هم داشته باشیم. RDBMS های معروف فراوانی وجود دارند و بسته به نیاز، می‌توان از هرکدام از آنها استفاده کرد. MySQL و PostgreSQL و SQLite از جمله معروف‌ترین RDBMS ها هستند.

\href{https://www.tutorialspoint.com/sqlite/sqlite_java.htm}{\textcolor{blue}{\underline{در این لینک میتوانید استفاده SQLite را در همراهی با جاوا مشاهده کنید.}}}
\\

\begin{itemize}
\item \textbf{\emph{ پایگاه‌داده‌های غیررابطه‌ای (\lr{NoSQL}) }}
\end{itemize}

با گسترش تکنولوژی و در نتیجه نیاز‌های بشر، دیگر RDBMS ها پاسخگوی همه‌ی نیازها نبودند و برای برخی مفاهیم، نیاز به نوع دیگری از DBMS ها حس می‌شد. در نتیجه‌ی این نیاز‌ها DBMS های دیگری مانند NoSQL ها ساخته شدند. این نوع از پایگاه‌های داده بر پایه‌ی جدول‌ها و روابط نیستند و از زبان SQL پشتیبانی نمی‌کنند. در عوض، قابلیت های دیگری مانند نگهداری پایگاه داده در چند سرور مجزا، جستجو در گراف و ... را می‌توان در این پایگاه‌های داده مشاهده کرد. مواردی مانند Redis و Cassandra و MongdoDB و ElasticSearch از دیتابیس های شناخته شده و پرکاربرد NoSQL هستند.

\href{https://www.tutorialspoint.com/mongodb/mongodb_java.htm}{\textcolor{blue}{\underline{در این لینک میتوانید استفاده MongoDB را در همراهی با جاوا مشاهده کنید.}}}
\\

در این فاز شما محدودیتی در استفاده از پایگاه‌های داده ندارید و میتوانید با توجه به نیاز و علاقه‌تان، از هر نوعی از پایگاه‌های داده که می‌خواهید استفاده کنید.
\\

موفق باشید!
\end{document}
